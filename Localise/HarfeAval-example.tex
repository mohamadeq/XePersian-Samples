% !TEX TS-program = XeLaTeX
% Commands for running this example:
% 	 xelatex HarfeAval-example
% End of Commands
\documentclass{article}
\pagestyle{empty}
\usepackage{xepersian}
\راحت
\ات‌حرف
\تر\کادرو@به@{\کادرو  to}
\بگذار\چپ‌گرد\llap
\تر\حرف‌اول#1{{\بند\درکادر\@فر\کادراست{#1}\بعد@\ارتفاع\@فر \بیفزابر\بعد@ \عمق\@فر
       \تقسیم\بعد@ \فاصله‌کرسی \@شماقت‌آ\بعد@ \عام\بیفزابر\@شماقت‌آ \@فر 
       \عام\درکادر\@فر\کادرو@به@\ارتفاع\کادرشمع{\کادر\@فر\هردوو}}%
       \بعدازسطر -\@شماقت‌آ  \تورفتگی‌ثابت- \عرض\@فر 
       \عمق\@فر=\@فر \بدون‌تورفتگی
       \چپ‌گرد{\کادر\@فر}
       \فاصله‌خالی‌راندیده‌بگیر}
\تر\حرف‌اول#1{\بند\درکادر\@فر\کادراست{#1}\بعد@\ارتفاع\@فر \بیفزابر\بعد@\عمق\@فر
       \@بعدقت‌آ\بعد@ \تقسیم\بعد@\فاصله‌کرسی \@شماقت‌آ\بعد@ \بیفزابر\@شماقت‌آ\یک@
       \بعد@\@شماقت‌آ\فاصله‌کرسی \بیفزابر\بعد@-\@بعدقت‌آ \تقسیم\بعد@\دو@
       \درکادر\@فر\کادرو@به@\ارتفاع\کادرشمع
                        {\فاصله‌و\بعد@ \کادر\@فر\هردوو}\عمق\@فر\@فر
       \بعدازسطر -\@شماقت‌آ  \تورفتگی‌ثابت -\عرض\@فر  
       \بدون‌تورفتگی\چپ‌گرد{\کادر\@فر}
       \فاصله‌خالی‌راندیده‌بگیر}
\تر\حرف‌اول#1{\بند\درکادر\@فر\کادراست{#1}\بعد@\ارتفاع\@فر \بیفزابر\بعد@\عمق\@فر
       \تقسیم\بعد@\فاصله‌کرسی \@شماقت‌آ\بعد@ \بیفزابر\@شماقت‌آ\یک@
%%       \درکادر\@فر\کادرو@به@\ارتفاع\کادرشمع{\کادر\@فر\هردوو}\عمق\@فر\@فر
       \درکادر\@فر\کادرو@به@0.618em{\کادر\@فر\هردوو}\عمق\@فر\@فر
       \بعدازسطر -\@شماقت‌آ  \تورفتگی‌ثابت- \عرض\@فر  
       \بدون‌تورفتگی\چپ‌گرد{\کادر\@فر}
       \فاصله‌خالی‌راندیده‌بگیر}
\ات‌دیگر
\شروع{نوشتار}
جزوه حاضر راهنمای  استفاده از نرم‌افزار تک‌پارسی است. با این نرم‌افزار می‌توان متنهای مختلف، به‌ویژه متنهای حاوی فرمول و علایم ریاضی را با کیفیت بسیار عالی حروف‌چینی کرد. فرمانهای تک‌پارسی که لابلای متن وروی قرار داده می‌شوند، طریقه حروف‌چینی و صفحه‌بندی مطالب را مشخص می‌کنند. مدت‌زمانی که صرف این کار می‌شود بیشتر از وقت مورد نیاز برای تایپ کردن آن روی ماشین تحریر نیست. در واقع از آنجا که می‌توان متنهای کامپیوتری را بسیار آسان و سریع تغییر داد، حروف‌چینی مطالب با استفاده از نرم‌افزار تک‌پارسی در کل به زمان بسیار کمتری نیاز دارد.

\پرش‌بلند
\حرف‌اول{\قلم‌عادی\بزرگ‌تر\شمایل‌سیاه ج}زوه حاضر راهنمای  استفاده از نرم‌افزار تک‌پارسی است. با این نرم‌افزار می‌توان متنهای مختلف، به‌ویژه متنهای حاوی فرمول و علایم ریاضی را با کیفیت بسیار عالی حروف‌چینی کرد. فرمانهای تک‌پارسی که لابلای متن وروی قرار داده می‌شوند، طریقه حروف‌چینی و صفحه‌بندی مطالب را مشخص می‌کنند. مدت‌زمانی که صرف این کار می‌شود بیشتر از وقت مورد نیاز برای تایپ کردن آن روی ماشین تحریر نیست. در واقع از آنجا که می‌توان متنهای کامپیوتری را بسیار آسان و سریع تغییر داد، حروف‌چینی مطالب با استفاده از نرم‌افزار تک‌پارسی در کل به زمان بسیار کمتری نیاز دارد.
\پایان{نوشتار}