\clearpage
\phantomsection
\addcontentsline{toc}{chapter}{مقدمه}
\chapter*{مقدمه}\markboth{مقدمه}{مقدمه}

در سال ۱۹۵۱ فان رایت\LTRfootnote{\lr{von Wright}} کتابی با عنوان «مقاله‌ای در منطق موجهات» \citep{Wright1951} منتشر کرد. هینتیکا\LTRfootnote{\lr{Hintikka}} با ایده‌هایی که از این مقاله گرفته بود در سال ۱۹۶۲ کتابی با عنوان «دانش و باور، مقدمه‌ای بر منطق مبتنی بر این دو مفهوم» \citep{Hintikka1962} به چاپ رساند. وی در این کتاب به کمک مفهوم جهان‌های ممکن مدلی برای دانش و باور ارائه کرد و به همین دلیل بسیاری او را پدر منطق شناختی می‌دانند. هدف اصلی او واکاوی مفهومی دانش و باور بود ولی پس از او عبارت «شناخت» در محدوده‌ای فراتر به کار گرفته شد، اعم از باور و هر روشی که یک عامل می‌تواند دانشی را بدست آورد. در اواخر دهه‌ی ۱۹۷۰ منطق شناختی مورد توجه دانشمندان فعال در شاخه‌هایی مانند هوش مصنوعی، فلسفه و نظریه بازی‌ها قرار گرفت.
در دهه ۸۰ محققین علوم کامپیوتر به منطق شناختی روی آوردند، فاگین\LTRfootnote{\lr{Fagin}}، هالپرن\LTRfootnote{\lr{Halpern}}، موزِز\LTRfootnote{\lr{Moses}} و وَردی\LTRfootnote{\lr{Vardi}} که از این دسته به حساب می‌آمدند مقالاتی را که در طی حدود ۱۰ سال در مورد منطق شناختی به چاپ رسانده بودند در کتابی با نام «استدلال درباره دانش» \citep{RAKFagin1995} جمع آوری کرده و در سال ۱۹۹۵ منتشر کردند.

منطق شناختی، منطقی وجهی است که به استدلال برمبنای دانش\LTRfootnote{\lr{information}} و فرادانش\LTRfootnote{\lr{higher order information}}\index{فرادانش} می‌پردازد، فرادانش دانشی است که عاملی درباره‌ی دانش خود و یا دانش دیگر عامل‌ها دارد. برای روشن شدن موضوع مثالی را با سه بازیکن 1، 2 و 3، و سه کارت قرمز، سفید وآبی در نظر بگیرید. کارت‌ها در میان بازیکنان به این صورت توزیع شده‌اند که قرمز، سفید و آبی به ترتیب در دستان 1، 2 و 3 قرار دارد. فرض کنید بازیکنان تنها کارت خویش را می‌بینند و همه می‌دانند که کارت‌ها به گونه‌ای میانشان توزبع شده است که هریک فقط یک کارت در دست دارند.  بوسیله‌ی منطق شناختی می‌توان جملات پیچیده‌ای را مانند «بازیکن 1 می‌داند که بازیکن 2 نمی‌داند که چه کارتی در دستان بازیکن 3 است.» صوری کرد. منطق شناخت با چنین فرادانش‌هایی سر و کار دارد. حتی فرادانش‌های پیچیده‌تری مانند همه‌دانی مشترک وجود دارد. برای نمونه در مثال بالا علم به اینکه دقیقاً سه کارت وجود دارد و علم به رنگ کارت‌ها همه‌دانی مشترک است. منطق شناختی همچنین واکاوی مفهومی خوبی از همه‌دانی مشترک در اختیار می‌گذارد.

اگرچه منطق شناختی آنالیز مناسبی از فرادانش در اختیار می‌گذارد ولی بررسی تغییر دانش خارج از گستره‌ی این منطق است. منطق‌های شناختی پویا منطق‌ شناختی را به گونه‌ای توسیع می‌دهند که استدلال درباره‌ی تغییرات دانش نیز امکان‌پذیر باشد. این توسیع از طرفی از نوعی معناشناسی زبان طبیعی الهام گرفته شد که در آن معنای جمله بعنوان طریقی برای تغییر داده‌های کسانی که آن را می‌شنوند در نظر گرفته می‌شود، و از طرف دیگر از مطالعه‌ی بازی‌ها که تغییر داده‌ها و فرادانش‌ها نقش بسزایی در آنها ایفا می‌کند. سیستم‌های منطقی مختلفی بر این اساس در طول سال‌ها شکل گرفته است که برجسته‌ترین آنها عبارتند از \citep{Gerbrandy1997} (الهام گرفته از \citep{Veltman1996})، \citep{Batlag1998} و  \citep{Cate2002}.

در منطق شناختی پویا تغییر وضعیت موجود توسط داده‌های جدید را به‌روزرسانی می‌خوانیم. ساده‌ترین مثال زمانی است که عاملی می‌فهمد که گزاره‌ی $ \varphi $ برقرار است. به‌روزرسانی با یک گزاره در این مثال به این معنی است که گزینه‌هایی که عامل ممکن می‌دانست ولی در آنها گزاره‌ی $ \varphi $ برقرار نیست حذف می‌شوند. در یک سیستم چند عاملی ممکن است عامل‌های مختلف دسترسی مختلفی به داده‌های جدید داشته باشند و همچنین اطلاعات عامل‌ها درباره‌ی دیگر عامل‌ها نقش بازی کند، از این رو می‌توان به‌روزرسانی پیچیده‌تری را در مثال قبل مدل کرد: فرض کنید بازیکن 1 کارت خود را به بازیکن 2 نشان دهد و بازیکن 3 نیز این را ببیند ولی از محتوای کارت خبردار نشود. در نتیجه دانش بازیکنان به این صورت تغییر می‌کند: بازیکن 2 می‌داند که محتوای کارت بازیکن 1 چیست، بازیکن 3 می‌داند که بازیکن 2 می‌داند محتوای کارت بازیکن 1 جیست ولی خودش محتوا را نمی‌داند و بازیکن 1 می‌داند که بازیکن 3 این را می‌داند.

اگرچه نظریه‌ی احتمال منطق نیست لکن حوزه‌ی مطالعاتی مناسبی برای منطق است، زیرا در بسیاری از حوزه‌های کاربردی به منظور استدلال درباره‌ی دانش، اهمیت توانایی استدلال درباره‌ی احتمال رخدادهای معین به همراه دانش عامل‌ها رخ می‌نماید و اغلب احتمال به عنوان نظریه‌ای برای مدلسازی استدلال مطرح می‌شود.
از این رو همه‌ی مقالات منتشر شده در علم اقتصاد که به استدلال درباره‌ی دانش می‌پردازند (که بازگشت می‌کنند به مقاله‌ی اصلی اومان \citep{Aumann1976}) با ساختاری احتمالاتی مدل می‌شوند، هرچند آنها زبانی منطقی که بصورتی روشن استدلال درباره‌ی احتمال را جایز کند در نظر نگرفته‌اند. با این اوصاف تلاش‌هایی در جهت بیرون کشیدن منطق بعنوان بهترین راه استدلال از دل نظریه‌ی احتمال صورت گرفته است. که یکی از مناسب‌تزین آنها منطق احتمالاتی است که در \citep{Fagin1990} معرفی شده است.