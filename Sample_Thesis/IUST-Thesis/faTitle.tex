% !TeX root=main.tex
% در این فایل، عنوان پایان‌نامه، مشخصات خود، متن تقدیمی‌، ستایش، سپاس‌گزاری و چکیده پایان‌نامه را به فارسی، وارد کنید.
% توجه داشته باشید که جدول حاوی مشخصات پروژه/پایان‌نامه/رساله و همچنین، مشخصات داخل آن، به طور خودکار، درج می‌شود.
%%%%%%%%%%%%%%%%%%%%%%%%%%%%%%%%%%%%
% دانشگاه خود را وارد کنید
\university{علم و صنعت ایران}
% دانشکده، آموزشکده و یا پژوهشکده  خود را وارد کنید
\faculty{دانشکده مهندسی کامپیوتر}
% گروه آموزشی خود را وارد کنید
\department{گروه هوش مصنوعی و رباتیک}
% گروه آموزشی خود را وارد کنید
\subject{مهندسی کامپیوتر}
% گرایش خود را وارد کنید
\field{هوش مصنوعی و رباتیک}
% عنوان پایان‌نامه را وارد کنید
\title{نوشتن پروژه، پایان‌نامه و رساله با استفاده از کلاس 
IUST-Thesis}
% نام استاد(ان) راهنما را وارد کنید
\firstsupervisor{استاد راهنمای اول}
\secondsupervisor{استاد راهنمای دوم}
% نام استاد(دان) مشاور را وارد کنید. چنانچه استاد مشاور ندارید، دستور پایین را غیرفعال کنید.
\firstadvisor{استاد مشاور اول}
%\secondadvisor{استاد مشاور دوم}
% نام دانشجو را وارد کنید
\name{امیر}
% نام خانوادگی دانشجو را وارد کنید
\surname{امین‌طوسی}
% شماره دانشجویی دانشجو را وارد کنید
\studentID{87922012}
% تاریخ پایان‌نامه را وارد کنید
\thesisdate{اسفند ۱۳۹۰}
% به صورت پیش‌فرض برای پایان‌نامه‌های کارشناسی تا دکترا به ترتیب از عبارات «پروژه»، «پایان‌نامه» و »رساله» استفاده می‌شود؛ اگر  نمی‌پسندید هر عنوانی را که مایلید در دستور زیر قرار داده و آنرا از حالت توضیح خارج کنید.
%\projectLabel{پایان‌نامه}

% به صورت پیش‌فرض برای عناوین مقاطع تحصیلی کارشناسی تا دکترا به ترتیب از عبارات «کارشناسی»، «کارشناسی ارشد» و »دکترا» استفاده می‌شود؛ اگر  نمی‌پسندید هر عنوانی را که مایلید در دستور زیر قرار داده و آنرا از حالت توضیح خارج کنید.
%\degree{}

\firstPage
\besmPage
\davaranPage

%\vspace{.5cm}
% در این قسمت اسامی اساتید راهنما، مشاور و داور باید به صورت دستی وارد شوند
%\renewcommand{\arraystretch}{1.2}
\begin{center}
\begin{tabular}{| p{8mm} | p{18mm} | p{.17\textwidth} |p{14mm}|p{.2\textwidth}|c|}
\hline
ردیف	& سمت & نام و نام خانوادگی & مرتبه \newline دانشگاهی &	دانشگاه یا مؤسسه &	امضـــــــــــــا\\
\hline
۱  &	استاد راهنما				 & دکتر \newline محمود فتحی & دانشیار & دانشگاه \newline علم و صنعت ایران &  \\
\hline
۲ &     استاد مشاور				 & دکتر \newline ناصر مزینی & استادیار & دانشگاه \newline علم و صنعت ایران & \\
\hline
۳ &      استاد مدعو\newline  خارجی			 & دکتر \newline محمدحسن \newline قاسمیان & استاد & دانشگاه \newline تربیت مدرس & \\
\hline
۴ &	استاد مدعو\newline  خارجی			 & دکتر \newline  نصرالله مقدم & استادیار & دانشگاه \newline  تربیت مدرس& \\
\hline
۵ &	استاد مدعو\newline  داخلی			 & دکتر\newline  رضا برنگی & استادیار & دانشگاه \newline  علم و صنعت ایران & \\
\hline
۶ &	استاد مدعو\newline  داخلی			 & دکتر\newline  محسن سریانی & استادیار & دانشگاه \newline  علم و صنعت ایران & \\
\hline
۷ &	استاد مدعو\newline  داخلی			 &دکتر \newline محمدرضا جاهدمطلق & دانشیار& دانشگاه \newline  علم و صنعت ایران & \\
\hline
\end{tabular}
\end{center}

\esalatPage
\mojavezPage


% چنانچه مایل به چاپ صفحات «تقدیم»، «نیایش» و «سپاس‌گزاری» در خروجی نیستید، خط‌های زیر را با گذاشتن ٪  در ابتدای آنها غیرفعال کنید.
 % پایان‌نامه خود را تقدیم کنید!

 \newpage
\thispagestyle{empty}
{\Large تقدیم به:}\\
\begin{flushleft}
{\huge
همسر و فرزندانم\\
\vspace{7mm}
و\\
\vspace{7mm}
پدر و مادرم
}
\end{flushleft}


% سپاس‌گزاری
\begin{acknowledgementpage}
سپاس خداوندگار حکیم را که با لطف بی‌کران خود، آدمی را زیور عقل آراست.


در آغاز وظیفه‌  خود  می‌دانم از زحمات بی‌دریغ استاد  راهنمای خود،  جناب آقای دکتر ...، صمیمانه تشکر و  قدردانی کنم  که قطعاً بدون راهنمایی‌های ارزنده‌  ایشان، این مجموعه  به انجام  نمی‌رسید.

از جناب  آقای  دکتر ...   که زحمت  مطالعه و مشاوره‌  این رساله را تقبل  فرمودند و در آماده سازی  این رساله، به نحو احسن اینجانب را مورد راهنمایی قرار دادند، کمال امتنان را دارم.

همچنین لازم می‌دانم از پدید آورندگان بسته زی‌پرشین، مخصوصاً جناب آقای  وفا خلیقی، که این پایان‌نامه با استفاده از این بسته، آماده شده است و همه دوستانمان در گروه پارسی‌لاتک کمال قدردانی را داشته باشم.

 در پایان، بوسه می‌زنم بر دستان خداوندگاران مهر و مهربانی، پدر و مادر عزیزم و بعد از خدا، ستایش می‌کنم وجود مقدس‌شان را و تشکر می‌کنم از خانواده عزیزم به پاس عاطفه سرشار و گرمای امیدبخش وجودشان، که بهترین پشتیبان من بودند.
% با استفاده از دستور زیر، امضای شما، به طور خودکار، درج می‌شود.
\signature 
\end{acknowledgementpage}
%%%%%%%%%%%%%%%%%%%%%%%%%%%%%%%%%%%%
% کلمات کلیدی پایان‌نامه را وارد کنید
\keywords{زی‌پرشین، لاتک، قالب پایان‌نامه، الگو}
%چکیده پایان‌نامه را وارد کنید، برای ایجاد پاراگراف جدید از \\ استفاده کنید. اگر خط خالی دشته باشید، خطا خواهید گرفت.
\fa-abstract{
این پایان‌نامه، به بحث در مورد نوشتن پروژه، پایان‌نامه و رساله با استفاده از کلاس 
\lr{IUST-Thesis}
می‌پردازد.
حروف‌چینی پروژه کارشناسی، پایان‌نامه یا رساله یکی از موارد پرکاربرد استفاده از زی‌پرشین است. 
زی‌پرشین بسته‌ای است که به همت آقای وفا خلیقی آماده شده است و امکان حروف‌چینی فارسی در \lr{\LaTeXe}{} را  برای فارسی‌زبانان فراهم کرده است.
از جمله مزایای لاتک آن است که در صورت وجود یک کلاس آماده برای حروف‌چینی یک سند خاص مانند یک پایان‌نامه، کاربر بدون درگیری با جزییات حروف‌چینی و صفحه‌آرایی می‌تواند سند خود را آماده نماید.
\\
شاید با قالب‌های لاتکی که برخی از مجلات برای مقالات خود عرضه می‌کنند مواجه شده باشید. اگر نظیر این کار در دانشگاههای مختلف برای اسناد متنوع آنها مانند پایا‌ن‌نامه‌ها آماده شود، دانشجویان به جای وقت گذاشتن روی صفحه‌آرایی مطالب خود، روی محتوای متن خود تمرکز خواهند نمود. به علاوه با آشنایی با لاتک خواهند توانست از امکانات بسیار این نرم‌افزار جهت نمایش بهتر دست‌آوردهای خود استفاده کنند.
به همین خاطر، یک کلاس با نام 
\lr{IUST-Thesis}
 برای حروف‌چینی پروژه‌ها، پایان‌نامه‌ها و رساله‌های دانشگاه علم و صنعت ایران با استفاده از نرم‌افزار زی‌پرشین،  آماده شده است. این فایل به 
گونه‌ای طراحی شده است که کلیات خواسته‌های مورد نیاز  مدیریت تحصیلات تکمیلی دانشگاه علم و صنعت ایران 
\cite{IUSTThesisGuide}
را برآورده می‌کند و نیز، حروف‌چینی بسیاری از قسمت‌های آن، به طور خودکار انجام می‌شود.
}

%\fa-abstract{
%%این پایان‌نامه، به بحث در مورد نوشتن پروژه، پایان‌نامه و رساله با استفاده از کلاس 
%%\lr{IUST-Thesis}
%%می‌پردازد. 
%%حروف‌چینی پروژه کارشناسی، پایان‌نامه یا رساله یکی از موارد پرکاربرد استفاده از زی‌پرشین است. 
%%زی‌پرشین بسته‌ای است که به همت آقای وفا خلیقی آماده شده است و امکان حروف‌چینی فارسی در  را  برای فارسی‌زبانان فراهم کرده است.
%%از جمله مزایای لاتک آن است که در صورت وجود یک کلاس آماده برای حروف‌چینی یک سند خاص مانند یک پایان‌نامه، کاربر بدون درگیری با جزییات حروف‌چینی و صفحه‌آرایی می‌توان سند خود را آماده نماید.
%
%شاید با قالب‌های لاتکی که برخی از مجلات برای مقالات خود عرضه می‌کنند مواجه شده باشید. اگر نظیر این کار در دانشگاههای مختلف برای اسناد متنوع آنها مانند پایا‌ن‌نامه‌ها آماده شود، دانشجویان به جای وقت گذاشتن روی صفحه‌آرایی مطالب خود، روی محتوای متن خود تمرکز خواهند نمود. به علاوه با آشنایی با لاتک خواهند توانست از امکانات بسیار این نرم‌افزار جهت نمایش بهتر دستآوردهای خود استفاده کنند.
%به همین خاطر، یک کلاس با نام 
%\lr{IUST-Thesis}
% برای حروف‌چینی پروژه‌ها، پایان‌نامه‌ها و رساله‌های دانشگاه علم و صنعت ایران با استفاده از نرم‌افزار زی‌پرشین،  آماده شده است. این فایل به 
%گونه‌ای طراحی شده است که کلیات خواسته‌های مورد نیاز  مدیریت تحصیلات تکمیلی دانشگاه علم و صنعت ایران را برآورده می‌کند و نیز، حروف‌چینی بسیاری از قسمت‌های آن، به طور خودکار انجام می‌شود.
%}

\abstractPage

\newpage\clearpage