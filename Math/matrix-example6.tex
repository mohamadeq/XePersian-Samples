% !TEX TS-program = XeLaTeX
% Commands for running this example:
% 	 xelatex matrix-example6
% End of Commands
\documentclass{article}
\pagestyle{empty}
\usepackage{amsmath,array}
\usepackage{xepersian}
\begin{document}
\setlength\delimitershortfall{-1pt}
\begin{align}
\underline{A} &= \left[\begin{array}{*7c}
    a_{11} & a_{12} & 0 & \ldots & \ldots & \ldots & 0\\
    a_{21} & a_{22} & a_{23} & 0 & \ldots & \ldots & 0\\
    0 & a_{32} & a_{33} & a_{34} & 0 & \ldots & 0\\
    \vdots & \vdots & \vdots & \vdots & \vdots & \vdots & \vdots\\
    \hdotsfor{7}\\
    \vdots & \vdots & \vdots & \vdots & \vdots & \vdots & \vdots\\
    0 & \ldots & 0 & a_{n-2,n-3} & a_{n-2,n-2} & a_{n-2,n-1} & 0\\
    0 & \ldots & \ldots & 0 & q_{n-1,n-2} & a_{n-1,n-1} & a_{n-1,n}\\
    0 & \ldots & \ldots & \ldots & 0 & a_{n,n-1} & a_{nn}
  \end{array}\right]
\end{align}
\end{document}
