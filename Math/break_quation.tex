% !TEX TS-program = XeLaTeX
% Commands for running this example:
% 	 xelatex break_quation
% End of Commands
\documentclass{article}
\pagestyle{empty}
\usepackage{amsmath}
\usepackage{breqn}
\usepackage{xepersian}
\settextfont{XB Zar}
%\setlatintextfont{Linux Libertine}% از نسخه ۱.۰.۴ زی‌پرشین این دستور الزامی نیست و قلم  پیش‌فرض تک مورد استفاده قرار می‌گیرد.
%\setdigitfont{XB Zar}
\title{شکستن خودکار فرمول}
\author{}
\begin{document}
\maketitle
با استفاده از بستهٔ \lr{breqn} می‌توانید فرمول را بصورت خودکار بشکنید و نیازی به محیط \lr{align} ندارید. \lr{breqn} جزو بسته \lr{mh} می‌باشد.

ضمناً  breqn باید بعد از amsmath فراخوانی شود و  گاهی اوقات فرمولها را آن جوری که ما دوست داریم نمی‌شکند!

\begin{dmath}[label={sna74}]
\frac{1}{6} \left(\sigma(k,h,0) +\frac{3(h-1)}{h}\right)
+\frac{1}{6} \left(\sigma(h,k,0) +\frac{3(k-1)}{k}\right)
=\frac{1}{6} \left(\frac{h}{k} +\frac{k}{h} +\frac{1}{hk}\right)
+\frac{1}{2} -\frac{1}{2h} -\frac{1}{2k},
\end{dmath}

\begin{dmath*}
T(n) \hiderel{\leq} T(2^{\lceil\lg n\rceil})
\leq c(3^{\lceil\lg n\rceil}
-2^{\lceil\lg n\rceil})
<3c\cdot3^{\lg n}
=3c\,n^{\lg3}
\end{dmath*}

\begin{dgroup*}
\begin{dmath*}
H_1^3 = x_1 + x_2 + x_3
\end{dmath*},
\begin{dmath*}
H_2^2 = x_1^2 + x_1 x_2 + x_2^2 - q_1 - q_2
\end{dmath*},
\begin{dsuspend}
و
\end{dsuspend}
\begin{dmath*}
H_3^1 = x_1^3 - 2x_1 q_1 - x_2 q_1
\end{dmath*}.
\end{dgroup*}
\end{document}