% !TEX TS-program = XeLaTeX 
% Commands for running this example:
% xelatex ghab-poem-example
% End of Commands
\documentclass{article}
\pagestyle{empty}
\renewcommand{\baselinestretch}{1.5}
\usepackage{fancyhdr,color}
\usepackage{fontspec}
\usepackage[Kashida]{xepersian}
\usepackage{bidipoem}
\newfontfamily\borderfont[SizeFeatures={Size=19},Color=8B0000]{WebOMints GD}
\fancyhf{}
\pagestyle{fancy}
\renewcommand{\headrulewidth}{0pt}
\setlength{\headheight}{19pt}
\setlength{\unitlength}{1bp}
\renewcommand{\poemcolsepskip}{1.5cm}
\fancyhead[HL]{\borderfont%
\begin{picture}(0,0)(62,-18)
\put(0,0){O}
\multiput(18,0)(18,0){24}{N}
\put(450,0){M}
\multiput(0,-18)(0,-18){34}{T}
\put(0,-630){R}
\multiput(18,-630)(18,0){24}{Q}
\multiput(450,-18)(0,-18){34}{S}
\put(450,-630){P}
\end{picture}}
\begin{document}
\begin{traditionalpoem}
جهان چون بزاری برآید همی & بدو نیک روزی سرآید همی \\
چو بستی کمر بر در راه آز & شود کار گیتیت یکسر دراز \\
بیک روی جستن بلندی سزاست & اگر در میان دم اژدهاست \\
و دیگر که گیتی ندارد درنگ & سرای سپنجی چه پهن و چه تنگ \\
پرستنده آز و جویای کین & بگیتی ز کس نشنود آفرین \\
چو سرو سهی گوژ گردد بباغ & بدو بر شود تیره روشن چراغ \\
کند برگ پژمرده و بیخ سست & سرش سوی پستی گراید نخست \\
بروید ز خاک و شود باز خاک & همه جای ترسست و تیمار و باک \\
سر مایهٔ مرد سنگ و خرد & ز گیتی بی‌آزاری اندر خورد \\
در دانش و آنگهی راستی & گرین دو نیابی روان کاستی \\
اگر خود بمانی بگیتی دراز & ز رنج تن آید برفتن نیاز \\
یکی ژرف دریاست بن ناپدید & در گنج رازش ندارد کلید \\
اگر چند یابی فزون بایدت & همان خورده یک روز بگزایدت \\
سه چیزت بباید کزان چاره نیست & وزو بر سرت نیز پیغاره نیست \\
خوری گر بپوشی و گر گستری & سزد گرد بدیگر سخن ننگری \\
چو زین سه گذشتی همه رنج و آز & چه در آز پیچی چه اندر نیاز \\
چو دانی که بر تو نماند جهان & چه پیچی تو زان جای نوشین روان \\
بخور آنچ داری و بیشی مجوی & که از آز کاهد همی آبروی \\
دل شاه ترکان چنان کم شنود & همیشه برنج از پی آز بود \\
ازان پس که برگشت زان رزمگاه & که رستم برو کرد گیتی سیاه \\
بشد تازیان تا بخلخ رسید & بننگ از کیان شد سرش ناپدید \\
بکاخ اندر آمد پرآزار دل & ابا کاردانان هشیاردل \\
چو پیران و گرسیوز رهنمون & قراخان و چون شیده و گرسیون \\
برایشان همه داستان برگشاد & گذشته سخنها همه کرد یاد \\
که تا برنهادم بشاهی کلاه & مرا گشت خورشید و تابنده ماه \\
مرا بود بر مهتران دسترس & عنان مرا برنتابید کس \\
ز هنگام رزم منوچهر باز & نبد دست ایران بتوران دراز \\
شبیخون کند تا در خان من & از ایران بیازند بر جان من \\
دلاور شد آن مردم نادلیر & گوزن اندر آمد ببالین شیر \\
برین کینه گر کار سازیم زود & وگرنه برآرند زین مرز دود \\
سزد گر کنون گرد این کشورم & سراسر فرستادگان گسترم \\
ز ترکان وز چین هزاران هزار & کمربستگان از در کارزار \\
بیاریم بر گرد ایران سپاه & بسازیم هر سو یکی رزمگاه \\
همه موبدان رای هشیار خویش & نهادند با گفت سالار خویش \\
که ما را ز جیحون بباید گذشت & زدن کوس شاهی بران پهن دشت \\
بموی لشکر گهی ساختن & شب و روز نسودن از تاختن \\
که آن جای جنگست و خون ریختن & چه با گیو و با رستم آویختن \\
سرافراز گردان گیرنده شهر & همه تیغ کین آب داده به زهر \\
چو افراسیاب آن سخنها شنود & برافروخت از بخت و شادی نمود \\
ابر پهلوانان و بر موبدان & بکرد آفرینی برسم ردان \\
نویسندهٔ نامه را پیش خواند & سخنهای بایسته چندی براند \\
فرستادگان خواست از انجمن & بنزدیک فغفور و شاه ختن \\
فرستاد نامه به هر کشوری & بهر نامداری و هر مهتری \\
سپه خواست کاندیشهٔ جنگ داشت & ز بیژن بدان گونه دل تنگ داشت \\
دو هفته برآمد ز چین و ختن & ز هر کشوری شد سپاه انجمن \\
چو دریای جوشان زمین بردمید & چنان شد که کس روز روشن ندید \\
گله هرچ بودش ز اسبان یله & بشهر اندر آورد یکسر گله \\
همان گنجها کز گه تور باز & پدر بر پسر بر همی داشت راز \\
سر بدره‌ها را گشادن گرفت & شب و روز دینار دادن گرفت \\
چو لشکر سراسر شد آراسته & بدان بی‌نیازی شد از خواسته \\
ز گردان گزین کرد پنجه هزار & همه رزم‌جویان سازنده کار \\
بشیده که بودش نبرده پسر & ز گردان جنگی برآورده سر \\
بدو گفت کین لشکر سرفراز & سپردم ترا راه خوارزم ساز \\
نگهبان آن مرز خوارزم باش & همیشه کمربستهٔ رزم باش \\
دگر پنجه از نامداران چین & بفرمود تا کرد پیران گزین \\
بدو گفت تا شهر ایران برو & ممان رخت و مه تخت سالار نو \\
در آشتی هیچ گونه مجوی & سخن جز بجنگ و بکینه مگوی \\
کسی کو برد آب و آتش بهم & ابر هر دوان کرده باشد ستم \\
دو پر مایه بیدار و دو پهلوان & یکی پیر و باهوش و دیگر جوان \\
برفتند با پند افراسیاب & برام پیر و جوان بر شتاب \\
ابا ترگ زرین و کوپال و تیغ & خروشان بکردار غرنده میغ \\
پس آگاهی آمد به پیروز شاه & که آمد ز توران بایران سپاه \\
جفاپیشه بدگوهر افراسیاب & ز کینه نیاید شب و روز خواب \\
برآورد خواهد همی سر ز ننگ & ز هر سو فرستاد لشکر بجنگ \\
همی زهر ساید بنوک سنان & که تابد مگر سوی ایران عنان \\
سواران جنگی چو سیصد هزار & بجیحون همی کرد خواهد گذار \\
سپاهی که هنگام ننگ و نبرد & ز جیحون بگردون برآورد گرد \\
دلیران بدرگاه افراسیاب & ز بانگ تبیره نیابند خواب \\
ز آوای شیپور و زخم درای & تو گویی برآید همی دل ز جای \\
گر آید بایران بجنگ آن سپاه & هژبر دلاور نیاید براه \\
سر مرز توران به پیران سپرد & سپاهی فرستاد با او نه خرد \\
سوی مرز خوارزم پنجه هزار & کمربسته رفت از در کارزار \\
سپهدارشان شیدهٔ شیر دل & کز آتش ستاند بشمشیر دل \\
سپاهی بکردار پیلان مست & که با جنگ ایشان شود کوه پست \\
چو بشنید گفتار کاراگهان & پراندیشه بنشست شاه جهان \\
بکاراگهان گفت کای بخردان & من ایدون شنیدستم از موبدان \\
که چون ماه ترکان برآید بلند & ز خورشید ایرانش آید گزند \\
سیه مارکورا سر آید بکوب & ز سوراخ پیچان شود سوی چوب \\
چو خسرو به بیداد کارد درخت & بگردد برو پادشاهی و تخت \\
همه موبدان را بر خویش خواند & شنیده سخن پیش ایشان براند \\
نشستند با شاه ایران براز & بزرگان فرزانه و رزم ساز \\
چو دستان سام و چو گودرز و گیو & چو شیدوش و فرهاد و رهام نیو \\
چو طوس و چو رستم یل پهلوان & فریبرز و شاپور شیر دمان \\
دگر بیژن گیو با گستهم & چو گرگین چون زنگه و گژدهم \\
جزین نامداران لشکر همه & که بودند شاه جهان را رمه \\
ابا پهلوانان چنین گفت شاه & که ترکان همی رزم جویند و گاه \\
چو دشمن سپه کرد و شد تیز چنگ & بباید بسیچید ما را بجنگ \\
بفرمود تا بوق با گاودم & دمیدند و بستند رویینه خم \\
از ایوان به میدان خرامید شاه & بیاراستند از بر پیل گاه \\
بزد مهره در جام بر پشت پیل & زمین را تو گفتی براندود نیل \\
هوا نیلگون شد زمین رنگ رنگ & دلیران لشکر بسان پلنگ \\
بچنگ اندرون گرز و دل پر ز کین & ز گردان چو دریای جوشان زمین \\
خروشی برآمد ز درگاه شاه & که ای پهلوانان ایران سپاه \\
کسی کو بساید عنان و رکیب & نباید که یابد بخانه شکیب \\
بفرمود کز روم وز هندوان & سواران جنگی گزیده گوان \\
دلیران گردنکش از تازیان & بسیچیدهٔ جنگ شیر ژیان \\
کمربسته خواهند سیصد هزار & ز دشت سواران نیزه گزار \\
هر آنکو چهل روزه را نزد شاه & نیاید نبیند بسر بر کلاه \\
پراگنده بر گرد کشور سوار & فرستاده با نامه شهریار \\
دو هفته برآمد بفرمان شاه & بجنبید در پادشاهی سپاه \\
ز لشکر همه کشور آمد بجوش & زگیتی بر آمد سراسر خروش \\
بشبگیر گاه خروش خروس & ز هر سوی برخاست آوای کوس \\
بزرگان هر کشوری با سپاه & نهادند سر سوی درگاه شاه \\
در گنجهای کهن باز کرد & سپه را درم دادن آغاز کرد \\
همه لشکر از گنج و دینار شاه & بسر بر نهادند گوهر کلاه \\
به بر گستوان و بجوشن چو کوه & شدند انجمن لشکری همگروه \\
چو شد کار لشکر همه ساخته & وزیشان دل شاه پرداخته \\
نخستین ازان لشکر نامدار & سواران شمشیر زن سی هزار \\
گزین کرد خسرو برستم سپرد & بدو گفت کای نامبردار گرد \\
ره سیستان گیر و برکش بگاه & بهندوستان اندر آور سپاه \\
ز غزنین برو تا براه برین & چو گردد ترا تاج و تخت و نگین \\
چو آن پادشاهی شود یکسره & ببشخور آید پلنگ و بره \\
فرامرز را ده کلاه و نگین & کسی کو بخواهد ز لشکر گزین \\
بزن کوس رویین و شیپور و نای & بکشمیر و کابل فزون زین مپای \\
که ما را سر از جنگ افراسیاب & نیابد همی خورد و آرام و خواب \\
الانان و غزدژ بلهراسب داد & بدو گفت کای گرد خسرو نژاد \\
برو با سپاهی بکردار کوه & گزین کن ز گردان لشکر گروه \\
سواران شایستهٔ کارزار & ببر تا برآری ز دشمن دمار \\
باشکش بفرمود تا سی هزار & دمنده هژبران نیزه گزار \\
برد سوی خوارزم کوس بزرگ & سپاهی بکردار درنده گرگ \\
زند بر در شهر خوارزم گاه & ابا شیدهٔ رزم زن کینه خواه \\
سپاه چهارم بگودرز داد & چه مایه ورا پند و اندرز داد \\
که رو با بزرگان ایران بهم & چو گرگین و چون زنگه و گستهم \\
زواره فریبرز و فرهاد و گیو & گرازه سپهدار و رهام نیو \\
بفرمود بستن کمرشان بجنگ & سوی رزم توران شدن بی درنگ \\
سپهدار گودرز کشوادگان & همه پهلوانان و آزادگان \\
نشستند بر زین بفرمان شاه & سپهدار گودرز پیش سپاه \\
بگودرز فرمود پس شهریار & چو رفتی کمر بستهٔ کارزار \\
نگر تا نیازی به بیداد دست & نگردانی ایوان آباد پست \\
کسی کو بجنگت نبندد میان & چنان ساز کش از تو ناید زیان \\
که نپسندد از ما بدی دادگر & سپنجست گیتی و ما برگذر \\
چو لشکر سوی مرز توران بری & من تیز دل را بتش سری \\
نگر تا نجوشی بکردار طوس & نبندی بهر کار بر پیل کوس \\
جهاندیده‌ای سوی پیران فرست & هشیوار وز یادگیران فرست \\
بپند فراوانش بگشای گوش & برو چادر مهربانی بپوش \\
بهر کار با هر کسی دادکن & ز یزدان نیکی دهش یاد کن \\
چنین گفت سالار لشکر بشاه & که فرمان تو برتر از شید و ماه \\
بدان سان شوم کم تو فرمان دهی & تو شاه جهانداری و من رهی \\
برآمد خروش از در پهلوان & ز بانگ تبیره زمین شد نوان \\
بلشکر گه آمد دمادم سپاه & جهان شد ز گرد سواران سیاه \\
به پیش سپاه اندرون پیل شست & جهان پست گشته ز پیلان مست \\
وزان ژنده پیلان جنگی چهار & بیاراسته از در شهریار \\
نهادند بر پشتشان تخت زر & نشستنگه شاه با زیب و فر \\
بگودرز فرمود تا بر نشست & بران تخت زر از بر پیل مست \\
برانگیخت پیلان و برخاست گرد & مر آن را بنیک اختری یاد کرد \\
که از جان پیران برآریم دود & بران سان که گرد پی پیل بود \\
بی آزار لشکر بفرمان شاه & همی رفت منزل بمنزل سپاه \\
چو گودرز نزدیک زیبد رسید & سران را ز لشکر همی برگزید \\
هزاران دلیران خنجر گزار & ز گردان لشکر دلاور سوار \\
از ایرانیان نامور ده‌هزار & سخن گوی و اندر خور کارزار \\
سپهدار پس گیو را پیش خواند & همه گفتهٔ شاه با او براند \\
بدو گفت کای پور سالار سر & برافراخته سر ز بسیار سر \\
گزین کردم اندر خورت لشکری & که هستند سالار هر کشوری \\
بدان تا بنزدیک پیران شوی & بگویی و گفتار او بشنوی \\
بگویی به پیران که من با سپاه & بزیبد رسیدم بفرمان شاه \\
شناسی تو گفتار و کردار خویش & بی آزاری و رنج و تیمار خویش \\
همه شهر توران بدی را میان & ببستند با نامدار کیان \\
فریدون فرخ که با داغ و درد & ز گیتی بشد دیده پر آب زرد \\
پر از درد ایران پر از داغ شاه & که با سوک ایرج نتابید ماه \\
ز ترکان تو تنها ازان انجمن & شناسی بمهر و وفا خویشتن \\
دروغست بر تو همین نام مهر & نبینم بدلت اندر آرام مهر \\
همانست کن شاه آزرمجوی & مرا گفت با او همه نرم گوی \\
ازان کو بکارسیاوش رد & بیفگند یک روز بنیاد بد \\
بنزد منش دستگاهست نیز & ز خون پدر بیگناهست نیز \\
گناهی که تا این زمان کرده‌ای & ز شاهان گیتی که آزرده‌ای \\
همی شاه بگذارد از تو همه & بدی نیکی انگارد از تو همه \\
نباید که بر دست ما بر تباه & شوی بر گذشته فراوان گناه \\
دگر کز پی جنگ افراسیاب & زمانه همی بر تو گیرد شتاب \\
بزرگان ایران و فرزند من & بخوانند بر تو همه پند من \\
سخن هرچ دانی بدیشان بگوی & وزیشان همیدون سخن بازجوی \\
اگر راست باشد دلت با زبان & گذشتی ز تیمار و رستی بجان \\
بر و بوم و خویشانت آباد گشت & ز تیغ منت گردن آزاد گشت \\
ور از تو پدیدار آید گناه & نماند بتو مهر و تخت و کلاه \\
نجویم برین کینه آرام و خواب & من و گرز و میدان افراسیاب \\
کزو شاه ما را بکین خواستن & نباید بسی لشکر آراستن \\
مگر پند من سربسر بشنوی & بگفتار هشیار من بگروی \\
نخستین کسی کو پی افگند کین & بخون ریختن برنوشت آستین \\
بخون سیاوش یازید دست & جهانی به بیداد بر کرد پست \\
بسان سگانش ازان انجمن & ببندی فرستی بنزدیک من \\
بدان تا فرستم بنزدیک شاه & چه شان سر ستاند چه بخشد کلاه \\
تو نشنیدی آن داستان بزرگ & که شیر ژیان آورد پیش گرگ \\
که هر کو بخون کیان دست آخت & زمانه بجز خاک جایش نساخت \\
دگر هرچ از گنج نزدیک تست & همه دشمن جان تاریک تست \\
ز اسپان پرمایه و گوهران & ز دیبا و دینار وز افسران \\
ز ترگ و ز شمشیر و برگستوان & ز خفتان، وز خنجر هندوان \\
همه آلت لشکر و سیم و زر & فرستی بنزدیک ما سربسر \\
به بیداد کز مردمان بستدی & فراز آوریدی ز دست بدی \\
بدان باز خری مگر جان خویش & ازین درکنی زود درمان خویش \\
چه اندر خور شهریارست ازان & فرستم بنزدیک شاه جهان \\
ببخشیم دیگر همه بر سپاه & بجای مکافات کرده گناه \\
و دیگر که پور گزین ترا & نگهبان گاه و نگین ترا \\
برادرت هر دو سران سپاه & که همزمان برآرند گردن بماه \\
چو هر سه بدین نامدار انجمن & گروگان فرستی بنزدیک من \\
بدان تا شوم ایمن از کار تو & برآرد درخت وفا بار تو \\
تو نیز آنگهی برگزینی دو راه & یکی راه‌جویی بنزدیک شاه \\
ابا دودمان نزد خسرو شوی & بدان سایهٔ مهر او بغنوی \\
کنم با تو پیمان که خسرو ترا & بخورشید تابان برآرد سرا \\
ز مهر دل او تو آگه تری & کزو هیچ ناید چز از بهتری \\
بشویی دل از مهر افراسیاب & نبینی شب تیره او را بخواب \\
گر از شاه ترکان بترسی ز بد & نخواهی که آیی بایران سزد \\
بپرداز توران و بنشین بچاج & ببر تخت ساج و بر افراز تاج \\
ورت سوی افراسیابست رای & برو سوی او جنگ ما را مپای \\
اگر تو بخواهی بسیچید جنگ & مرا زور شیرست و چنگ پلنگ \\
بترکان نمانم من از تخت بهر & کمان من ابرست و بارانش زهر \\
بسیچیدهٔ جنگ خیز اندرآی & گرت هست با شیر درنده پای \\
چو صف برکشید از دو رویه سپاه & گنهکار پیدا شد از بیگناه \\
گرین گفته‌های مرا نشنوی & بفرجام کارت پشیمان شوی \\
پشیمانی آنگه نداردت سود & که تیغ زمانه سرت را درود \\
بگفت این سخن پهلوان با پسر & که بر خوان بپیران همه دربدر \\
ز پیش پدر گیو شد تا ببلخ & گرفته بیاد آن سخنهای تلخ \\
فرود آمد و کس فرستاد زود & بران سان که گودرز فرموده بود \\
همان شب سپاه اندر آورد گرد & برفت از در بلخ تا ویسه گرد \\
که پیران بدان شهر بد با سپاه & که دیهیم ایران همی جست و گاه \\
فرستاده چون سوی پیران رسید & سپدار ایران سپه را بدید \\
بگفتند کآمد سوی بلخ گیو & ابا ویژگان سپهدار نیو \\
چو بشنید پیران برافراخت کوس & شد از سم اسبان زمین آبنوس \\
ده و دو هزارش ز لشکر سوار & فراز آمد اندر خور کارزار \\
ازیشان دو بهره هم آنجا بماند & برفت و جهاندیدگانرا بخواند \\
بیامد چو نزدیک جیحون رسید & بگرد لب آب لشکر کشید \\
بجیحون پر از نیزه دیوار کرد & چو با گیو گودرز دیدار کرد \\
دو هفته شد اندر سخنشان درنگ & بدان تا نباشد به بیداد جنگ \\
ز هر گونه گفتند و پیران شنید & گنهکاری آمد ز ترکان پدید \\
بزرگان ایران زمان یافتند & بریشان بگفتار بشتافتند \\
برافگند یپران هم اندر شتاب & نوندی بنزدیک افراسیاب \\
که گودرز کشوادگان با سپاه & نهاد از بر تخت گردان کلاه \\
فرستاده آمد بنزدیک من & گزین پور او مهتر انجمن \\
مار گوش و دل سوی فرمان تست & بپیمان روانم گروگان تست \\
سخن چون بسالار ترکان رسید & سپاهی ز جنگ آوران برگزید \\
فرستاد نزدیک پیران سوار & ز گردان شمشیر زن سی هزار \\
بدو گفت بردار شمشیر کین & وزیشان بپرداز روی زمین \\
نه گودرز باید که ماند نه گیو & نه فرهاد و گرگین نه رهام نیو \\
که بر ما سپه آمد از چار سوی & همی گاه توران کنند آرزوی \\
جفا پیشه گشتم ازین پس بجنگ & نجویم بخون ریختن بر درنگ \\
برای هشیوار و مردان مرد & برآرم ز کیخسرو این بار گرد \\
چو پیران بدید آن سپاه بزرگ & بخون تشنه هر یک بکردار گرگ \\
بر آشفت ازان پس که نیرو گرفت & هنرها بشست از دل آهو گرفت \\
جفا پیشه گشت آن دل نیکخوی & پر اندیشه شد رزم کرد آرزوی \\
بگیو آنگهی گفت برخیز و رو & سوی پهلوان سپه باز شو \\
بگویش که از من تو چیزی مجوی & که فرزانگان آن نبینند روی \\
یکی آنکه از نامدارگوان & گروگان همی خواهی این کی توان \\
و دیگر که گفتی سلیح و سپاه & گرانمایه اسبان و تخت و کلاه \\
برادرکه روشن جهان منست & گزیده پسر پهلوان منست \\
همی گویی از خویشتن دور کن & ز بخرد چنین خام باشد سخن \\
مرا مرگ بهتر ازان زندگی & که سالار باشم کنم بندگی \\
یکی داستان زد برین بر پلنگ & چو با شیر جنگ آورش خاست جنگ \\
بنام ار بریزی مرا گفت خون & به از زندگانی بننگ اندرون \\
و دیگر که پیغام شاه آمدست & بفرمان جنگم سپاه آمدست \\
چو پاسخ چنین یافت برگشت گیو & ابا لشکری نامبردار و نیو \\
سپهدار چون گیو برگشت از وی & خروشان سوی جنگ بنهاد روی \\
دمان از پس گیو پیران دلیر & سپه را همی راند برسان شیر \\
بیامد چو پیش کنابد رسید & بران دامن کوه لشکر کشید \\
چو گیو اندر آمد بپیش پدر & همی گفت پاسخ همه دربدر \\
بگودرز گفت اندرآور سپاه & بجایی که سازی همی رزمگاه \\
که او را همی آشتی رای نیست & بدلش اندرون داد را جای نیست \\
ز هر گونه با او سخن راندم & همه هرچ گفتی برو خواندم \\
چو آمد پدیدار ازیشان گناه & هیونی برافگند نزدیک شاه \\
که گودرز و گیو اندر آمد بجنگ & سپه باید ایدر مرا بی درنگ \\
سپاه آمد از نزدافراسیاب & چو ما بازگشتیم بگذاشت آب \\
کنون کینه را کوس بر پیل بست & همی جنگ ما را کند پیشدست \\
چنین گفت با گیو پس پهلوان & که پیران بسیری رسید از روان \\
همین داشتم چشم زان بد نهان & ولیکن بفرمان شاه جهان \\
بایست رفتن که چاره نبود & دلش را کنون شهریار آزمود \\
یکی داستان گفته بودم بشاه & چو فرمود لشکر کشیدن براه \\
که دل را ز مهر کسی برگسل & کجا نیستش با زبان راست دل \\
همه مهر پیران بترکان برست & بشوید همی شاه ازو پاک دست \\
چو پیران سپاه از کنابد براند & بروز اندرون روشنایی نماند \\
سواران جوشن وران صد هزار & ز ترکان کمربستهٔ کارزار \\
برفتند بسته کمرها بجنگ & همه نیزه و تیغ هندی بچنگ \\
چو دانست گودرز کآمد سپاه & بزد کوس و آمد ز زیبد براه \\
ز کوه اندر آمد بهامون گذشت & کشیدند لشکر بران پهن دشت \\
بکردار کوه از دو رویه سپاه & ز آهن بسر بر نهاده کلاه \\
برآمد خروشیدن کرنای & بجنبد همی کوه گفتی ز جای \\
ز زیبد همی تاکنابد سپاه & در و دشت ازیشان کبود و سیاه \\
ز گرد سپه روز روشن نماند & ز نیزه هوا جز بجوشن نماند \\
وز آواز اسبان و گرد سپاه & بشد روشنایی ز خورشید و ماه \\
ستاره سنان بود و خروشید تیغ & از آهن زمین بود وز گرز میغ \\
بتوفید ز آواز گردان زمین & ز ترگ و سنان آسمان آهنین \\
چو گودرز توران سپه را بدید & که برسان دریا زمین بردمید \\
درفش از درفش و گروه از گروه & گسسته نشد شب برآمد ز کوه \\
چو شب تیره شد پیل پیش سپاه & فرازآوریدند و بستند راه \\
برافروختند آتش از هردو روی & از آواز گردان پرخاشجوی \\
جهان سربسر گفتی آهرمنست & بدامن بر از آستین دشمنست \\
ز بانگ تبیره بسنگ اندرون & بدرد دل اندر شب قیر گون \\
سپیده برآمد ز کوه سیاه & سپهدار ایران به پیش سپاه \\
بسوده اسب اندر آورد پای & یلان را بهر سو همی ساخت جای \\
سپه را سوی میمنه کوه بود & ز جنگ دلیران بی‌اندوه بود \\
سوی میسره رود آب روان & چنان در خور آمد چو تن را روان \\
پیاده که اندر خور کارزار & بفرمود تا پیش روی سوار \\
صفی بر کشیدند نیزه‌وران & ابا گرزداران و کنداوران \\
همیدون پیاده بسی نیزه‌دار & چه با ترکش و تیر و جوشن‌گذار \\
کمانها فگنده بباز و درون & همی از جگرشان بجوشید خون \\
پس پشت ایشان سواران جنگ & کز آتش بخنجر ببردند رنگ \\
پس پشت لشکر ز پیلان گروه & زمین از پی پیل گشته ستوه \\
درفش خجسته میان سپاه & ز گوهر درفشان بکردار ماه \\
ز پیلان زمین سربسر پیلگون & ز گرد سواران هوا نیلگون \\
درخشیدن تیغهای بنفش & ازان سایهٔ کاویانی درفش \\
تو گفتی که اندرشب تیره‌چهر & ستاره همی برفشاند سپهر \\
بیاراست لشکر بسان بهشت & بباغ وفا سرو کینه بکشت \\
فریبزر را داد پس میمنه & پس پشت لشکر حصار و بنه \\
گرازه سر تخمهٔ گیوگان & زواره نگهدار تخت کیان \\
بیاری فریبرز برخاستند & بیک روی لشکر بیاراستند \\
برهام فرمود پس پهلوان & که ای تاج و تخت و خرد را روان \\
برو با سواران سوی میسره & نگه‌دار چنگال گرگ از بره \\
بیفروز لشکرگه از فر خویش & سپه را همی دار در بر خویش \\
بدان آبگون خنجر نیو سوز & چو شیر ژیان با یلان رزم توز \\
برفتند یارانش با او بهم & ز گردان لشکر یکی گستهم \\
دگر گژدهم رزم را ناگزیر & فروهل که بگذارد از سنگ تیر \\
بفرمود با گیو تا دو هزار & برفتند بر گستوان‌ور سوار \\
سپرد آن زمان پشت لشکر بدوی & که بد جای گردان پرخاشجوی \\
برفتند با گیو جنگاوران & چو گرگین و چون زنگهٔ شاوران \\
درفشی فرستاد و سیصد سوار & نگهبان لشکر سوی رودبار \\
همیدون فرستاد بر سوی کوه & درفشی و سیصد ز گردان گروه \\
یکی دیده‌بان بر سر کوهسار & نگهبان روز و ستاره شمار \\
شب و روز گردن برافراخته & ازان دیده‌گه دیده‌بان ساخته \\
بجستی همی تا ز توران سپاه & پی مور دیدی نهاده براه \\
ز دیده خروشیدن آراستی & بگفتی بگودرز و برخاستی \\
بدان سان بیاراست آن رزمگاه & که رزم آرزو کرد خورشید و ماه \\
چو سالار شایسته باشد بجنگ & نترسد سپاه از دلاور نهنگ \\
ازان پس بیامد بسالارگاه & که دارد سپه را ز دشمن نگاه \\
درفش دلفروز بر پای کرد & سپه را بقلب اندرون جای کرد \\
سران را همه خواند نزدیک خویش & پس پشت شیدوش و فرهاد پیش \\
بدست چپش رزم‌دیده هجیر & سوی راست کتمارهٔ شیرگیر \\
ببستند ز آهن بگردش سرای & پس پشت پیلان جنگی بپای \\
سپهدار گودرزشان در میان & درفش از برش سایهٔ کاویان \\
همی بستد از ماه و خورشید نور & نگه کرد پیران بلشکر ز دور \\
بدان ساز و آن لشکر آراستن & دل از ننگ و تیمار پیراستن \\
در و دشت و کوه و بیابان سنان & عنان بافته سربسر با عنان \\
سپهدار پیران غمی گشت سخت & برآشفت با تیره خورشید بخت \\
ازان پس نگه کرد جای سپاه & نیامدش بر آرزو رزمگاه \\
نه آوردگه دید و نه جای صف & همی برزد از خشم کف را بکف \\
برین گونه کآمد ببایست ساخت & چو سوی یلان چنگ بایست آخت \\
پس از نامداران افراسیاب & کسی کش سر از کینه گیرد شتاب \\
گزین کرد شمشیرزن سی‌هزار & که بودند شایستهٔ کارزار \\
بهومان سپرد آن زمان قلبگاه & سپاهی هژبر اوژن و رزمخواه \\
بخواند اندریمان و او خواست را & نهاد چپ لشکر و راست را \\
چپ لشکرش را بدیشان سپرد & ابا سی‌هزار از دلیران گرد \\
چو لهاک جنگی و فرشیدورد & ابا سی‌هزار از دلیران مرد \\
گرفتند بر میمنه جایگاه & جهان سربسر گشت ز آهن سیاه \\
چو زنگولهٔ گرد و کلباد را & سپهرم که بد روز فریاد را \\
برفتند با نیزه‌ور ده هزار & بپشت سواران خنجرگزار \\
برون رفت رویین رویینه‌تن & ابا ده هزار از یلان ختن \\
بدان تا دران بیشه اندر چو شیر & کمینگه کند با یلان دلیر \\
طلایه فرستاد بر سوی کوه & سپهدار ایران شود زو ستوه \\
گر از رزمگه پی نهد پیشتر & وگر جنبد از خویشتن بیشتر \\
سپهدار رویین بکردار شیر & پس پشت او اندر آید دلیر \\
همان دیده‌بان بر سر کوه کرد & که جنگ سواران بی‌اندوه کرد \\
ز ایرانیان گر سواری ز دور & عنان تافتی سوی پیکار تور \\
نگهبان دیده گرفتی خروش & همه رزمگاه آمدی زو بجوش \\
دو لشکر بروی اندر آورد روی & همه نامداران پرخاشجوی \\
چنین ایستاده سه روز و سه شب & یکی را بگفتن نجنبید لب \\
همی گفت گودرز گر پشت خویش & سپارم بدیشان نهم پای پیش \\
سپاه اندر آید پس پشت من & نماند جز از باد در مشت من \\
شب و روز بر پای پیش سپاه & همی جست نیک اختر هور و ماه \\
که روزی که آن روز نیک‌اخترست & کدامست و جنبش کرا بهترست \\
کجا بردمد باد روز نبرد & که چشم سواران بپوشد بگرد \\
بریشان بیابم مگر دستگاه & بکردار باد اندر آرم سپاه \\
نهاده سپهدار پیران دو چشم & که گودرز رادل بجوشد ز خشم \\
کند پشت بر دشت و راند سپاه & سپاه اندآرد بپشت سپاه \\
بروز چهارم ز پیش سپاه & بشد بیژن گیو تا قلبگاه \\
بپیش پدر شد همه جامه چاک & همی بسمان بر پراگند خاک \\
بدو گفت کای باب کارآزمای & چه داری چنین خیره ما را بپای \\
بپنجم فرازآمد این روزگار & شب و روز آسایش آموزگار \\
نه خورشید شمشیر گردان بدید & نه گردی بروی هوا بردمید \\
سواران بخفتان و خود اندرون & یکی رابرگ بر نجنبید خون \\
بایران پس از رستم نامدار & نبودی چو گودرز دیگر سوار \\
چینن تا بیامد ز جنگ پشن & ازان کشتن و رزمگاه گشن \\
بلاون که چندان پسر کشته دید & سر بخت ایرانیان گشته دید \\
جگر خسته گشستست و گم کرده‌راه & نخواهد که بیند همی رزمگاه \\
بپیرانش بر چشم باید فگند & نهادست سر سوی کوه بلند \\
سپهدار کو ناشمرده سپاه & ستاره شمارد همی گرد ماه \\
تو بشناس کاندر تنش نیست خون & شد ازجنگ جنگاوران او زبون \\
شگفت از جهاندیده گودرز نیست & که او را روان خود برین مرز نیست \\
شگفت از تو آید مرا ای پدر & که شیر ژیان از تو جوید هنر \\
دو لشکر همی بر تو دارند چشم & یکی تیز کن مغز و بفروز خشم \\
کنون چون جهان گرم و روشن هوا & بگیرد همی رزم لشکر نوا \\
چو این روزگار خوشی بگذرد & چو پولاد روی زمین بفسرد \\
چو بر نیزه‌ها گردد افسرده چنگ & پس پشت تیغ آید و پیش سنگ \\
که آید ز گردان بپیش سپاه & که آورد گیردبدین رزمگاه \\
ور ایدونک ترسد همی از کمین & ز جنگ سواران و مردان کین \\
بمن داد باید سواری هزار & گزین من اندرخور کارزار \\
برآریم گرد از کمینگاهشان & سرافشان کنیم از بر ماهشان \\
ز گفتار بیژن بخندید گیو & بسی آفرین کرد بر پور نیو \\
بدادار گفت از تو دارم سپاس & تو دادی مرا پور نیکی‌شناس \\
همش هوش دادی و هم زور کین & شناسای هر کار و جویای دین \\
بمن بازگشت این دلاور جوان & چنانچون بود بچهٔ پهلوان \\
چنین گفت مر جفت را نره شیر & که فرزند ما گر نباشد دلیر \\
ببریم ازو مهر و پیوند پاک & پدرش آب دریا بود مام خاک \\
ولیکن تو ای پور چیره سخن & زبان بر نیا بر گشاده مکن \\
که او کاردیدست و داناترست & برین لشکر نامور مهترست \\
کسی کو بود سودهٔ کارزار & نباید بهر کارش آموزگار \\
سواران ما گرد ببار اندرند & نه ترکان برنگ و نگار اندرند \\
همه شوربختند و برگشته سر & همه دیده پرخون و خسته جگر \\
همی خواهد این باب کارآزمای & که ترکان بجنگ اندر آرند پای \\
پس پشتشان دور ماند ز کوه & برد لشکر کینه‌ور همگروه \\
ببینی تو گوپال گودرز را & که چون برنوردد همی مرز را \\
و دیگر کجا ز اختر نیک و بد & همی گردش چرخ را بشمرد \\
چو پیش آید آن روزگار بهی & کند روی گیتی ز ترکان تهی \\
چنین گفت بیژن به پیش پدر & که ای پهلوان جهان سربسر \\
خجسته نیا را گر اینست رای & سزد گر نداریم رومی قبای \\
شوم جوشن و خود بیرون کنم & بمی روی پژمرده گلگلون کنم \\
چو آیم جهان پهلوان را بکار & بیایم کمربستهٔ کارزار \\
وزان لشکر ترک هومان دلیر & بپیش برادر بیامد چو شیر \\
که ای پهلوان رد افراسیاب & گرفت اندرین دشت ما را شتاب \\
بهفتم فراز آمد این روزگار & میان بسته در جنگ چندین سوار \\
از آهن میان سوده و دل ز کین & نهاده دو دیده بایران زمین \\
چه داری بروی اندرآورده روی & چه اندیشه داری بدل در بگوی \\
گرت رای جنگست جنگ آزمای & ورت رای برگشتن ایدر مپای \\
که ننگست ازین بر تو ای پهلوان & بدین کار خندند پیر و جوان \\
همان لشکرست این که از ما بجنگ & برفتند و رفته ز روی آب و رنگ \\
کزیشان همه رزمگه کشته بود & زمین سربسر رود خون گشته بود \\
نه زین نامداران سواری کمست & نه آن دوده را پهلوان رستمست \\
گرت آرزو نیست خون ریختن & نخواهی همی لشکر انگیختن \\
ز جنگ‌آوران لشکری برگزین & بمن ده تو بنگر کنون رزم و کین \\
چو بشنید پیران ز هومان سخن & بدو گفت مشتاب و تندی مکن \\
بدان ای برادر که این رزمخواه & که آمد چنین پیش ما با سپاه \\
گزین بزرگان کیخسروست & سر نامداران هر پهلوست \\
یکی آنک کیخسرو از شاه من & بدو سر فرازد بهر انجمن \\
و دیگر که از پهلوانان شاه & ندانم چو گودرز کس را بجاه \\
بگردن‌فرازی و مردانگی & برای هشیوار و فرزانگی \\
سدیگر که پرداغ دارد جگر & پر از خون دل از درد چندان پسر \\
که از تن سرانشان جدامانده‌ایم & زمین را بخون گرد بنشانده‌ایم \\
کنون تا بتنش اندرون جان بود & برین کینه چون مار پیچان بود \\
چهارم که لشکر میان دو کوه & فرود آوریدست و کرده گروه \\
ز هر سو که پویی بدو راه نیست & براندیش کین رنج کوتاه نیست \\
بکوشید باید بدان تا مگر & ازان کوه‌پایه برآرند سر \\
مگر مانده گردند و سستی کنند & بجنگ اندرون پیشدستی کنند \\
چو از کوه بیرون کند لشکرش & یکی تیرباران کنم بر سرش \\
چو دیوار گرد اندر آریمشان & چو شیر ژیان در بر آریمشان \\
بریشان بگردد همه کام ما & برآید بخورشید بر نام ما \\
تو پشت سپاهی و سالار شاه & برآورده از چرخ گردان کلاه \\
کسی کو بنام بلندش نیاز & نباشد چه گردد همی گرد آز \\
و دیگر که از نامداران جنگ & نیاید کسی نزد ما بی‌درنگ \\
ز گردان کسی را که بی‌نام‌تر & ز جنگ سواران بی‌آرام‌تر \\
ز لشکر فرستد بپیشت بکین & اگر برنوردی برو بر زمین \\
ترا نام ازان برنیاید بلند & بایرانیان نیز ناید گزند \\
وگر بر تو بر دست یابد بخون & شوند این دلیران ترکان زبون \\
نگه کرد هومان بگفتار اوی & همی خیره دانست پیکار اوی \\
چنین داد پاسخ کز ایران سوار & نباشد که با من کند کارزار \\
ترا خود همین مهربانیست خوی & مرا کارزار آمدست آرزوی \\
وگر کت بکین جستن آهنگ نیست & بدلت اندرون آتش جنگ نیست \\
کنم آنچ باید بدین رزمگاه & نمایم هنرها بایران سپاه \\
شوم چرمهٔ گامزن زین کنم & سپیده دمان جستن کین کنم \\
نشست از بر زین سپیده‌دمان & چو شیر ژیان با یکی ترجمان \\
بیامد بنزدیک ایران سپاه & پر از جنگ دل سر پر از کین شاه \\
چو پیران بدانست کو شد بجنگ & بروبرجهان گشت ز اندوه تنگ \\
بجوشیدش از درد هومان جگر & یکی داستان یاد کرد از پدر \\
که دانا بهر کار سازد درنگ & سر اندر نیارد بپیکار و ننگ \\
سبکسار تندی نماید نخست & بفرجام کار انده آرد درست \\
زبانی که اندر سرش مغز نیست & اگر در بارد همان نغز نیست \\
چو هومان بدین رزم تندی نمود & ندانم چه آرد بفرجام سود \\
جهانداورش باد فریادرس & جز اویش نبینم همی یار کس \\
چو هومان ویسه بدان رزمگاه & که گودرز کشواد بد با سپاه \\
بیامد که جوید ز گردان نبرد & نگهبان لشکر بدو بازخورد \\
طلایه بیامد بر ترجمان & سواران ایران همه بدگمان \\
بپرسید کین مرد پرخاشجوی & بخیره بدشت اندر آورده روی \\
کجا رفت خواهد همی چون نوند & بچنگ اندرون گرز و بر زین کمند \\
بایرانیان گفت پس ترجمان & که آمد گه گرز و تیر و کمان \\
که این شیردل نامبردار مرد & همی با شما کرد خواهد نبرد \\
سر ویسگانست هومان بنام & که تیغش دل شیر دارد نیام \\
چو دیدند ایرانیان گرز اوی & کمر بستن خسروی برز اوی \\
همه دست نیزه گزاران ز کار & فروماند از فر آن نامدار \\
همه یکسره بازگشتند ازوی & سوی ترجمانش نهادند روی \\
که رو پیش هومان بترکی زبان & همه گفتهٔ ما بروبر بخوان \\
که ما رابجنگ تو آهنگ نیست & ز گودرز دستوری جنگ نیست \\
اگر جنگ جوید گشادست راه & سوی نامور پهلوان سپاه \\
ز سالار گردان و گردنکشان & بهومان بدادند یک یک نشان \\
که گردان کجایند و مهتر کجاست & که دارد چپ لشکر و دست راست \\
وزانپس هیونی تگاور دمان & طلایه برافگند زی پهلوان \\
که هومان ازان رزمگه چون پلنگ & سوی پهلوان آمد ایدر بجنگ \\
چو هومان ز نزد سواران برفت & بیامد بنزدیک رهام تفت \\
وزانجا خروشی برآورد سخت & که ای پور سالار بیدار بخت \\
چپ لشکر و چنگ شیران توی & نگهبان سالار ایران توی \\
بجنبان عنان اندرین رزمگاه & میان دو صف برکشیده سپاه \\
بورد با من ببایدت گشت & سوی رود خواهی وگر سوی دشت \\
وگر تو نیابی مگر گستهم & بیاید دمان با فروهل بهم \\
که جوید نبردم ز جنگاوران & بتیغ و سنان و بگرز گران \\
هرآنکس که پیش من آید بکین & زمانه برو بر نوردد زمین \\
وگر تیغ ما را ببیند بجنگ & بدرد دل شیر و چرم پلنگ \\
چنین داد رهام پاسخ بدوی & که ای نامور گرد پرخاشجوی \\
زترکان ترا بخرد انگاشتم & ازین سان که هستی نپنداشتم \\
که تنها بدین رزمگاه آمدی & دلاور بپیش سپاه آمدی \\
بر آنی که اندر جهان تیغ‌دار & نبندد کمر چون تو دیگر سوار \\
یکی داستان از کیان یاد کن & زفام خرد گردن آزاد کن \\
که هر کو بجنگ اندر آید نخست & ره بازگشتن ببایدش جست \\
ازاینها که تو نام بردی بجنگ & همه جنگ را تیز دارند چنگ \\
ولیکن چو فرمان سالار شاه & نباشد نسازد کسی رزمگاه \\
اگر جنگ گردان بجویی همی & سوی پهلوان چون بپویی همی \\
ز گودرز دستوری جنگ خواه & پس از ما بجنگ اندر آهنگ خواه \\
بدو گفت هومان که خیره مگوی & بدین روی با من بهانه مجوی \\
تو این رزم را جای مردان گزین & نه مرد سوارانی و دشت کین \\
وزانجا بقلب سپه برگذشت & دمان تا بدان روی لشکرگذشت \\
بنزد فریبرز با ترجمان & بیامد بکردار باد دمان \\
یکی برخروشید کای بدنشان & فروبرده گردن ز گردنکشان \\
سواران و پیلان و زرینه کفش & ترا بود با کاویانی درفش \\
بترکان سپردی بروز نبرد & یلانت بایران نخوانند مرد \\
چو سالار باشی شوی زیردست & کمر بندگی را ببایدت بست \\
سیاوش رد را برادر توی & بگوهر ز سالار برتر توی \\
تو باشی سزاوار کین خواستن & بکینه ترا باید آراستن \\
یکی با من اکنون به آوردگاه & ببایدت گشتن بپیش سپاه \\
بخورشید تابان برآیدت نام & که پیش من اندر گذاری تو گام \\
وگر تو نیایی بحنگم رواست & زواره گرازه نگر تاکجاست \\
کسی را ز گردان بپیش من آر & که باشد ز ایرانیان نامدار \\
چنین داد پاسخ فریبرز باز & که با شیر درنده کینه مساز \\
چنینست فرجام روز نبرد & یکی شاد و پیروز و دیگر بدرد \\
بپیروزی اندر بترس از گزند & که یکسان نگردد سپهر بلند \\
درفش ار ز من شاه بستد رواست & بدان داد پیلان و لشکر که خواست \\
بکین سیاوش پس از کیقباد & کسی کو کلاه مهی برنهاد \\
کمر بست تا گیتی آباد کرد & سپهدار گودرز کشواد کرد \\
همیشه بپیش کیان کینه‌خواه & پدر بر پدر نیو و سالار شاه \\
و دیگر که از گرز او بی‌گمان & سرآید بسالارتان بر زمان \\
سپه را به ویست فرمان جنگ & بدو بازگردد همه نام و ننگ \\
اگر با توم جنگ فرمان دهد & دلم پر ز دردست درمان دهد \\
ببینی که من سر چگونه ز ننگ & برآرم چو پای اندر آرم بجنگ \\
چنین پاسخش داد هومان که بس & بگفتار بینم ترا دسترس \\
بدین تیغ کاندر میان بسته‌ای & گیابر که از جنگ خود رسته‌ای \\
بدین گرز جویی همی کارزار & که بر ترگ و جوشن نیاید بکار \\
وزآنجا بدان خیرگی بازگشت & تو گفتی مگر شیر بدساز گشت \\
کمربستهٔ کین آزادگان & بنزدیک گودرز کشوادگان \\
بیامد یکی بانگ برزد بلند & که ای برمنش مهتر دیوبند \\
شنیدم همه هرچ گفتی بشاه & وزان پس کشیدی سپه را براه \\
چنین بود با شاه پیمان تو & بپیران سالار فرمان تو \\
فرستاده کامد بتوران سپاه & گزین پور تو گیو لشکرپناه \\
ازان پس که سوگند خوردی بماه & بخورشید و ماه و بتخت و کلاه \\
که گر چشم من درگه کارزار & بپیران برافتد برارم دمار \\
چو شیر ژیان لشکر آراستی & همی برزو جنگ ما خواستی \\
کنون از پس کوه چون مستمند & نشستی بکردار غرم نژند \\
بکردار نخچیر کز شرزه شیر & گریزان و شیر از پس اندر دلیر \\
گزیند ببیشه درون جای تنگ & نجوید ز تیمار جان نام و ننگ \\
یکی لشکرت را بهامون گذار & چه داری سپاه از پس کوهسار \\
چنین بود پیمانت با شهریار & که بر کینه گه کوه گیری حصار \\
بدو گفت گودرز کاندیشه کن & که باشد سزا با تو گفتن سخن \\
چو پاسخ بیابی کنون ز انجمن & به بیدانشی بر نهی این سخن \\
تو بشناس کز شاه فرمان من & همین بود سوگند و پیمان من \\
کنون آمدم با سپاهی گران & از ایران گزیده دلاور سران \\
شما هم بکردار روباه پیر & ببیشه در از بیم نخچیرگیر \\
همی چاره سازید و دستان و بند & گریزان ز گرز و سنان و کمند \\
دلیری مکن جنگ ما را مخواه & که روباه با شیر ناید براه \\
چو هومان ز گودرز پاسخ شنید & چو شیر اندران رزمگه بردمید \\
بگودرز گفت ار نیایی بجنگ & تو با من نه زانست کایدت ننگ \\
ازان پس که جنگ پشن دیده‌ای & سر از رزم ترکان بپیچیده‌ای \\
به لاون بجنگ آزمودی مرا & به آوردگه بر ستودی مرا \\
ار ایدونک هست اینک گویی همی & وزین کینه کردار جویی همی \\
یکی برگزین از میان سپاه & که با من بگردد به آوردگاه \\
که من از فریبرز و رهام جنگ & بجستم بسان دلاور پلنگ \\
بگشتم سراسر همه انجمن & نیاید ز گردان کسی پیش من \\
بگودرز بد بند پیکارشان & شنیدن نه ارزید گفتارشان \\
تو آنی که گویی بروز نبرد & بخنجر کنم لاله بر کوه زرد \\
یکی با من اکنون بدین رزمگاه & بگرد و بگرز گران کینه‌خواه \\
فراوان پسر داری ای نامور & همه بسته بر جنگ ما بر کمر \\
یکی را فرستی بر من بجنگ & اگر جنگ‌جویی چه جویی درنگ \\
پس اندیشه کرد اندران پهلوان & که پیشش که آید بجنگ از گوان \\
گر از نامداران هژبری دمان & فرستم بنزدیک این بدگمان \\
شود کشته هومان برین رزمگاه & ز ترکان نیاید کسی کینه‌خواه \\
دل پهلوانش بپیچد بدرد & ازان پس بتندی نجوید نبرد \\
سپاهش بکوه کنابد شود & بجنگ اندرون دست ما بد شود \\
ور از نامداران این انجمن & یکی کم شود گم شود نام من \\
شکسته شود دل گوان را بجنگ & نسازند زان پس به جایی درنگ \\
همان به که با او نسازیم کین & بروبر ببندیم راه کمین \\
مگر خیره گردند و جویند جنگ & سپاه اندر آرند زان جای تنگ \\
چنین داد پاسخ بهومان که رو & بگفتار تندی و در کار نو \\
چو در پیش من برگشادی زبان & بدانستم از آشکارت نهان \\
که کس را ز ترکان نباشد خرد & کز اندیشهٔ خویش رامش برد \\
ندانی که شیر ژیان روز جنگ & نیالاید از بن بروباه چنگ \\
و دیگر دو لشکر چنین ساخته & همه بادپایان سر افراخته \\
بکینه دو تن پیش سازند جنگ & همه نامداران بخایند چنگ \\
سپه را همه پیش باید شدن & به انبوه زخمی بباید زدن \\
تو اکنون سوی لشکرت باز شو & برافراز گردن بسالار نو \\
کز ایرانیان چند جستم نبرد & نزد پیش من کس جز از باد سرد \\
بدان رزمگه بر شود نام تو & ز پیران برآید همه کام تو \\
بدو گفت هومان ببانگ بلند & که بی کردن کار گفتار چند \\
یکی داستان زد جهاندار شاه & بیاد آورم اندرین کینه‌گاه \\
که تخت کیان جست خواهی مجوی & چو جویی از آتش مبرتاب روی \\
ترا آرزو جنگ و پیکار نیست & وگر گل چنی راه بی‌خار نیست \\
نداری ز ایران یکی شیرمرد & که با من کند پیش لشکرنبرد \\
بچاره همی بازگردانیم & نگیرم فریبت اگر دانیم \\
همه نامدراان پرخاشجوی & بگودرز گفتند کاینست روی \\
که از ما یکی را به آوردگاه & فرستی بنزدیک او کینه‌خواه \\
چنین داد پاسخ که امروز روی & ندارد شدن جنگ را پیش اوی \\
چو هومان ز گودرز برگشت چیر & برآشفت برسان شیر دلیر \\
بخندید و روی از سپهبد بتافت & سوی روزبانان لشکر شتافت \\
کمان را بزه کرد و زیشان چهار & بیفگند ز اسب اندران مرغزار \\
چو آن روزبانان لشکر ز دور & بدیدند زخم سرافراز تور \\
رهش بازدادند و بگریختند & بورد با او نیاویختند \\
ببالا برآمد بکردار مست & خروشش همی کوه را کرد پست \\
همی نیزه برگاشت بر گرد سر & که هومان ویسه است پیروزگر \\
خروشیدن نای رویین ز دشت & برآمد چو نیزه ز بالا بگشت \\
ز شادی دلیران توران سپاه & همی ترگ سودند بر چرخ ماه \\
چو هومان بیامد بدان چیرگی & بپیچید گودرز زان خیرگی \\
سپهبد پر از شرم گشته دژم & گرفته برو خشم و تندی ستم \\
بننگ از دلیران بپالود خوی & سپهبد یکی اختر افگند پی \\
کزیشان بد این پیشدستی بخون & بدانند و هم بر بدی رهنمون \\
ازان پس بگردنکشان بنگرید & که تا جنگ او را که آید پدید \\
خبر شد به بیژن که هومان چو شیر & بپیش نیای تو آمد دلیر \\
چو بشنید بیژن برآشفت سخت & بخشم آمد آن شیر پنجه ز بخت \\
بفرمود تا برنهادند زین & بران پیل تن دیزهٔ دوربین \\
بپوشید رومی زره جنگ را & یکی تنگ بر بست شبرنگ را \\
بپیش پدر شد پر از کیمیا & سخن گفت با او ز بهر نیا \\
چنین گفت مر گیو را کای پدر & بگفتم ترا من همه دربدر \\
که گودرز را هوش کمتر شدست & بیین نبینی که دیگر شدست \\
دلش پر نهیبست و پر خون جگر & ز تیمار وز درد چندان پسر \\
که از تن سرانشان جدا کرده دید & بدان رزمگه جمله افگنده دید \\
نشان آنک ترکی بیامد دلیر & میان دلیران بکردار شیر \\
بپیش نیا رفت نیزه بدست & همی بر خروشید برسان مست \\
چنان بد کزین لشکر رنامدار & سواری نبود از در کارزار \\
که او را بنیزه برافراختی & چو بر بابزن مرغ بر ساختی \\
تو ای مهربان باب بسیار هوش & دو کتفم بدرع سیاوش بپوش \\
نشاید جز از من که سازم نبرد & بدان تا برآرم ز مردیش گرد \\
بدو گفت گیو ای پسر هوش دار & بگفتار من سربسر گوش دار \\
تا گفته بودم که تندی مکن & ز گودرز بر بد مگردان سخن \\
که او کار دیده‌ست و داناترست & بدین لشکر نامور مهترست \\
سواران جنگی بپیش اندرند & که بر کینه گه پیل را بشکرند \\
نفرمود با او کسی را نبرد & جوانی مگر مر ترا خیره کرد \\
که گردن بدین سان برافراختی & بدین آرزو پیش من تاختی \\
نیم من بدین کار همداستان & مزن نیز پیشم چنین داستان \\
بدو گفت بیژن که گر کام من & نجویی نخواهی مگر نام من \\
شوم پیش سالار بسته کمر & زنم دست بر جنگ هومان ببر \\
وزآنجا بزد اسب و برگاشت روی & بنزدیک گودرز شد پوی پوی \\
ستایش کنان پیش او شد بدرد & هم این داستان سربسر یاد کرد \\
که ای پهلوان جهاندار شاه & شناسای هر کار و زیبای گاه \\
شگفتی همی بینم از تو یکی & وگر چند هستم بهوش اندکی \\
کزین رزمگه بوستان ساختی & دل از کین ترکان بپرداختی \\
شگفتی‌تر آنک از میان سپاه & یکی ترک بدبخت گم کرده راه \\
بیامد که یزدان نیکی‌کنش & همی بد سگالید با بد تنش \\
بیاوردش از پیش توران سپاه & بدان تا بدست تو گردد تباه \\
بدام آمده گرگ برگاشتی & ندانم کزین خود چه پنداشتی \\
تو دانی که گر خون او بی‌درنگ & بریزند پیران نیاید بجنگ \\
مپدار کو کینه بیش آورد & سپه را برین دشت پیش آورد \\
من اینک بخون چنگ را شسته‌ام & همان جنگ او را کمر بسته‌ام \\
چو دستور باشد مرا پهلوان & شوم پیش او چون هژبر دمان \\
بفرماید اکنون سپهبد به گیو & مگر کان سلیح سیاوش نیو \\
دهد مر مرا خود و رومی زره & ز بند زره برگشاید گره \\
چو بشنید گودرز گفتار اوی & بدید آن دل و رای هشیار اوی \\
ز شادی برو آفرین کرد سخت & که از تو مگرداد جاوید بخت \\
تو تا برنشستی بزین پلنگ & نهنگ از دم آسود و شیران ز جنگ \\
بهر کارزار اندر آیی دلیر & بهر جنگ پیروز باشی چو شیر \\
نگه کن که با او به آوردگاه & توانی شدن زان پس آورد خواه \\
که هومان یکی بدکنش ریمنست & بورد جنگ او چو آهرمنست \\
جوانی و ناگشته بر سر سپهر & نداری همی بر تن خویش مهر \\
بمان تا یکی رزم دیده هژبر & فرستم بجنگش بکردار ابر \\
برو تیرباران کند چون تگرگ & بسر بر بدوزدش پولاد ترگ \\
بدو گفت بیژن که ای پهلوان & هنرمند باشد دلیر و جوان \\
مرا گر بدیدی برزم فرود & ز سر باز باید کنون آزمود \\
بجنگ پشن بر نوشتم زمین & نبیند کسی پشت من روز کین \\
مرا زندگانی نه اندر خورست & گر از دیگرانم هنر کمترست \\
وگر بازداری مرا زین سخن & بدان روی کهنگ هومان مکن \\
بنالم من از پهلوان پیش شاه & نخواهم کمر زان سپس نه کلاه \\
بخندید گودرز و زو شاد شد & بسان یکی سرو آزاد شد \\
بدو گفت نیک اختر و بخت گیو & که فرزند بیند همی چون تو نیو \\
تو تا چنگ را باز کردی بجنگ & فروماند از جنگ چنگ پلنگ \\
ترا دادم این رزم هومان کنون & مگر بخت نیکت بود رهنمون \\
گر این اهرمن را بدست تو هوش & براید بفرمان یزدان بکوش \\
بنام جهاندار یزدان ما & بپیروزی شاه و گردان ما \\
بگویم کنون گیو را کان زره & که بیژن همی خواهد او را بده \\
گر ایدنک پیروز باشی بروی & ترا بیشتر نزد من آبروی \\
ز فرهاد و گیوت برآرم بجاه & بگنج و سپاه و بتخت و کلاه \\
بگفت این سخن با نبیره نیا & نبیره پر از بند و پر کیمیا \\
پیاده شد از اسب و روی زمین & ببوسید و بر باب کرد آفرین \\
بخواند آن زمان گیو را پهلوان & سخن گفت با او ز بهر جوان \\
وزان خسروانی زره یاد کرد & کجا خواست بیژن ز بهر نبرد \\
چنین داد پاسخ پدر را پسر & که ای پهلوان جهان سربسر \\
مرا هوش و جان و جهان این یکیست & بچشمم چنین جان او خوار نیست \\
بدو گفت گودرز کای مهربان & جز این برد باید بوی بر گمان \\
که هر چند بیژن جوانست و نو & بهر کار دارد خرد پیشرو \\
و دیگر که این جای کین جستنست & جهان را ز آهرمنان شستنست \\
بکین سیاوش بفرمان شاه & نشاید بپیوند کردن نگاه \\
و گر بارد از ابر پولاد تیغ & نشاید که دارم ما جان دریغ \\
نشاید شکستن دلش را بجنگ & بگوشیدنش جامهٔ نام و ننگ \\
که چون کاهلی پیشه گیرد جوان & بماند منش پست و تیره روان \\
چو پاسخ چنین یافت چاره نبود & یکی با پسر نیز بند آزمود \\
بگودرز گفت ای جهان پهلوان & بجایی که پیکار خیزد بجان \\
مرا خود شب و روز کارست پیش & چرا داد باید مرا جان خویش \\
نه فرزند باید نه گنج و سپاه & نه آزرم سالار و فرمان شاه \\
اگر جنگ جوید سلیحش کجاست & زره دارد از من چه بایدش خواست \\
چنین گفت پیش پدر رزمساز & که ما را بدرع تو ناید نیاز \\
برانی که اندر جهان سربسر & بدرع تو جویند مردان هنر \\
چو درع سیاوش نباشد بجنگ & نجویند گردنکشان نام و ننگ \\
برانگیخت اسب از میان سپاه & که آید ز لشکر به آوردگاه \\
چو از پیش گودرز شد ناپدید & دل گیو ز اندوه او بردمید \\
پشیمان شد از درد دل خون گریست & نگر تا غم و مهر فرزند چیست \\
یکی بسمان برفرازید سر & پر از خون دل از درد خسته جگر \\
بدادار گفت ار جهان‌داوری & یکی سوی این خسته‌دل بنگری \\
نسوزی تو از جان بیژن دلم & که ز آب مژه تا دل اندر گلم \\
بمن بازبخشش تو ای کردگار & بگردان ز جانش بد روزگار \\
بیامد پراندیشه دل پهلوان & پراز خون دل ازبهر رفته جوان \\
بدل گفت خیره بیازردمش & چرا خواسته پیش ناوردمش \\
گر او را ز هومان بد آید بسر & چه باید مرا درع و تیغ و کمر \\
بمانم پر از حسرت و درد و خشم & پر از آرزو دل پر از آب چشم \\
وزانجا دمان هم بکردار گرد & بپیش پسر شد بجای نبرد \\
بدو گفت ما را چه داری بتنگ & همی تیزی آری بجای درنگ \\
سیه مار چندان دمد روز جنگ & که از ژرف دریا برآید نهنگ \\
درفشیدن ماه چندان بود & که خورشید تابنده پنهان بود \\
کنون سوی هومان شتابی همی & ز فرمان من سر بتابی همی \\
چنین برگزینی همی رای خویش & ندانی که چون آیدت کار پیش \\
بدو گفت بیژن که ای نیو باب & دل من ز کین سیاوش متاب \\
که هومان نه از روی وز آهنست & نه پیل ژیان و نه آهرمنست \\
یکی مرد جنگست و من جنگجوی & ازو برنتابم ببخت تو روی \\
نوشته مگر بر سرم دیگرست & زمانه بدست جهانداورست \\
اگر بودنی بود دل را بغم & سزد گر نداری نباشی دژم \\
چو بنشید گفتار پور دلیر & میان بستهٔ جنگ برسان شیر \\
فرودآمد از دیزهٔ راهجوی & سپر داد و درع سیاوش بدوی \\
بدو گفت گر کارزارت هواست & چنین بر خرد کام تو پادشاست \\
برین بارهٔ گامزن برنشین & که زیر تو اندر نوردد زمین \\
سلیحم همیدون بکار آیدت & چو با اهرمن کارزار آیدت \\
چو اسب پدر دید بر پای پیش & چو باد اندر آمد ز بالای خویش \\
بران بارهٔ خسروی برنشست & کمربست و بگرفت گرزش بدست \\
یکی ترجمان را ز لشکر بجست & که گفتار ترکان بداند درست \\
بیامد بسان هژبر ژیان & بکین سیاوش بسته میان \\
چو بیژن بنزدیک هومان رسید & یکی آهنین کوه پوشیده دید \\
ز جوشن همه دشت روشن شده & یکی پیل در زیر جوشن شده \\
ازان پس بفرمود تا ترجمان & یکی بانگ برزد بران بدگمان \\
که گر جنگ جویی یگی بازگرد & که بیژن همی با تو جوید نبرد \\
همی گوید ای رزم دیده سوار & چه پویانی اسب اندرین مرغزار \\
کز افراسیاب اندر آیدت بد & ز توران زمین بر تو نفرین سزد \\
بکینه پی‌افگنده و بدخوی & ز ترکان گنهکارتر کس توی \\
عنان بازکش زین تگاور هیون & کت اکنون ز کینه بجوشید خون \\
یکی برگزین جایگاه نبرد & بدشت و در و کوه با من بگرد \\
وگر در میان دو رویه سپاه & بگردی بلاف از پی نام و جاه \\
کجا دشمن و دوست بیند ترا & دل اکنون کجا برگزیند ترا \\
چو بشنید هومان بدو گفت زه & زره را بکینم تو بستی گره \\
ز یزدان سپاس و بدویم پناه & کت آورد پیشم بدین رزمگاه \\
بلشکر بران سان فرستمت باز & که گیو از تو ماند بگرم و گداز \\
سرت را ز تن دور مانم نه دیر & چنان کز تبارت فراوان دلیر \\
چه سودست کآمد بنزدیک شب & رو اکنون بزنهار تاریک شب \\
من اکنون یکی باز لشگر شوم & بشبگیر نزدیک مهتر شوم \\
وزآنجا دمان گردن افراخته & بیایم نبرد ترا ساخته \\
چنین پاسخ آورد بیژن که شو & پست باد و آهرمنت پیشرو \\
همه دشمنان سربسر کشته باد & گر آواره از جنگ برگشته باد \\
چو فردا بیایی به آوردگاه & نبیند ترا نیز شاه و سپاه \\
سرت را چنان دور مانم ز پای & کزان پس بلشکر نیایدت رای \\
وزآن جایگه روی برگاشتند & بشب دشت پیکار بگذاشتند \\
بلشکر گه خویش بازآمدند & بر پهلوانان فراز آمدند \\
همه شب بخواب اند آسیب شیب & ز پیکارشان دل شده ناشکیب \\
سپیده چو از کوه سربردمید & شد آن دامن تیره شب ناپدید \\
بپوشید هومان سلیح نبرد & سخن پیش پیران همه یاد کرد \\
که من بیژن گیو را خواستم & همه شب همی جنگش آراستم \\
یکی ترجمان را ز لشکر بخواند & بگلگون بادآورش برنشاند \\
که رو پیش بیژن بگویش که زود & بیایی دمان گر من آیم چو دود \\
فرستاده برگشت و با او بگفت & که با جان پاکت خرد باد جفت \\
سپهدار هومان بیامد چو گرد & بدان تا ز بیژن بجوید نبرد \\
چو بشنید بیژن بیامد دمان & بسیچیده جنگ با ترجمان \\
بپشت شباهنگ بر بسته تنگ & چو جنگی پلنگی گرازان بجنگ \\
زره با گره بر بر پهلوی & درفشان سر از مغفر خسروی \\
بهومان چنین گفت کای بادسار & ببردی ز من دوش سر یاددار \\
امیدستم امروز کین تیغ من & سرت را ز بن بگسلاند ز تن \\
که از خاک خیزد ز خون تو گل & یکی داستان اندر آری بدل \\
که با آهوان گفت غرم ژیان & که گر دشت گردد همه پرنیان \\
ز دامی که پای من آزادگشت & نپویم بران سوی آباد دشت \\
چنین داد پاسخ که امروز گیو & بماند جگر خسته بر پور نیو \\
بچنگ منی در بسان تذرو & که بازش برد بر سر شاخ سرو \\
خروشان و خون از دو دیده چکان & کشانش بچنگال و خونش مکان \\
بدو گفت بیژن که تا کی سخن & کجا خواهی آهنگ آورد کن \\
بکوه کنابد کنی کارزار & اگر سوی زیبد برآرای کار \\
که فریادرسمان نباشد ز دور & نه ایران گراید بیاری نه تور \\
برانگیختند اسب و برخاست گرد & بزه بر نهاده کمان نبرد \\
دو خونی برافراخته سر بماه & چنان کینه‌ور گشته از کین شاه \\
ز کوه کنابد برون تاختند & سران سوی هامون برافراختند \\
برفتند چندانک اندر زمی & ندیدند جایی پی آدمی \\
نه بر آسمان کرگسان را گذر & نه خاکش سپرده پی شیر نر \\
نه از لشکران یار و فریادرس & بپیرامن اندر ندیدند کس \\
نهادند پیمان که با ترجمان & نباشند در چیرگی بدگمان \\
بدان تا بد و نیک با شهریار & بگویند ازین گردش روزگار \\
که کردار چون بود و پیکار چون & چه زاری رسید اندرین دشت خون \\
بگفتند و زاسبان فرود آمدند & ببند زره بر کمر برزدند \\
بر اسبان جنگی سواران جنگ & یکی برکشیدند چون سنگ تنگ \\
چو بر بادپایان ببستند زین & پر از خشم گردان و دل پر ز کین \\
کمانها چوبایست برخاستند & بمیدان تنگ اندرون تاختند \\
چپ و راست گردان و پیچان عنان & همان نیزه و آب داده سنان \\
زرهشان درآورد شد لخت لخت & نگر تا کرا روز برگشت و بخت \\
دهنشان همی از تبش مانده باز & بب و بسایش آمد نیاز \\
پس آسوده گشتند و دم برزدند & بران آتش تیز نم برزدند \\
سپر برگرفتند و شمشیر تیز & برآمد خروشیدن رستخیز \\
چو بر درفشان که از تیره میغ & همی آتش افروخت ازهردو تیغ \\
زآهن بدان آهن آبدار & نیامد بزخم اندرون تابدار \\
بکردارآتش پرنداوران & فرو ریخت ازدست کنداوران \\
نبد دسترسشان بخون ریختن & نشد سیر دلشان زآویختن \\
عمود از پس تیغ برداشتند & از اندازه پیکار بگذاشتند \\
ازان پس بران بر نهادند کار & که زور آزمایند در کارزار \\
بدین گونه جستند ننگ و نبرد & که از پشت زین اندر آرند مرد \\
کمربند گیرد کرا زور بیش & رباید ز اسب افگند خوار پیش \\
ز نیروی گردان دوال رکیب & گسست اندر آوردگاه از نهیب \\
همیدون نگشتند ز اسبان جدا & نبودند بر یکدگر پادشا \\
پس از اسب هر دو فرود آمدند & ز پیکار یکبار دم برزدند \\
گرفته بدست اسپشان ترجمان & دو جنگی بکردار شیر دمان \\
بدان ماندگی باز برخاستند & بکشتی گرفتن بیاراستند \\
زشبگیر تا سایه گسترد شید & دو خونی ازین سان به بیم و امید \\
همی رزم جستند یک با دگر & یکی را ز کینه نه برگشت سر \\
دهن خشک و غرقه شده تن در آب & ازان رنج و تابیدن آفتاب \\
وزان پس بدستوری یکدگر & برفتند پویان سوی آبخور \\
بخورد آب و برخاست بیژن بدرد & ز دادار نیکی دهش یاد کرد \\
تن از درد لرزان چو از باد بید & دل از جان شیرین شده ناامید \\
بیزدان چنین گفت کای کردگار & تو دانی نهان من و آشکار \\
اگر داد بینی همی جنگ ما & برین کینه جستن بر آهنگ ما \\
ز من مگسل امروز توش مرا & نگه دار بیدار هوش مرا \\
جگر خسته هومان بیامد چو زاغ & سیه گشت از درد رخ چون چراغ \\
بدان خستگی باز جنگ آمدند & گرازان بسان پلنگ آمدند \\
همی زور کرد این بران آن برین & گه این را بسودی گه آنرا زمین \\
ز بیژن فزون بود هومان بزور & هنر عیب گردد چو برگشت هور \\
ز هر گونه زور آزمودند و بند & فراز آمد آن بند چرخ بلند \\
بزد دست بیژن بسان پلنگ & ز سر تا میانش بیازید چنگ \\
گرفتش بچپ گردن و راست ران & خم آورد پشت هیون گران \\
برآوردش از جای و بنهاد پست & سوی خنجر آورد چون باد دست \\
فرو برد و کردش سر از تن جدا & فگندش بسان یکی اژدها \\
بغلتید هومان بخاک اندرون & همه دشت شد سربسر جوی خون \\
نگه کرد بیژن بدان پیلتن & فگنده چو سرو سهی بر چمن \\
شگفت آمدش سخت و برگشت ازوی & سوی کردگار جهان کرد روی \\
که ای برتر از جایگاه و زمان & ز جان سخن‌گوی و روشن‌روان \\
توی تو که جز تو جهاندار نیست & خرد را بدین کار پیکار نیست \\
مرا زین هنر سربسر بهره نیست & که با پیل کین جستنم زهره نیست \\
بکین سیاوش بریدمش سر & بهفتاد خون برادر پدر \\
روانش روان ورا بنده باد & بچنگال شیران تنش کنده باد \\
سرش را بفتراک شبرنگ بست & تنش را بخاک اندر افگند پست \\
گشاده سلیح و گسسته کمر & تنش جای دیگر دگر جای سر \\
زمانه سراسر فریبست و بس & بسختی نباشدت فریادرس \\
جهان را نمایش چو کردار نیست & سپردن بدو دل سزاوار نیست \\
بترسید ازو یار هومان چو دید & که بر مهتر او چنان بد رسید \\
چو شد کار هومان ویسه تباه & دوان ترجمانان هر دو سپاه \\
ستایش‌کنان پیش بیژن شدند & چو پیش بت چین برهمن شدند \\
بدو گفت بیژن مترس از گزند & که پیمان همانست و بگشاد بند \\
تو اکنون سوی لشکر خویش پوی & ز من هرچ دیدی بدیشان بگوی \\
بشد ترجمان بیژن آمد دمان & بکوه کنابد بزه بر کمان \\
چو بیژن نگه کرد زان رزمگاه & نبودش گذر جز بتوران سپاه \\
بترسید از انبوه مردم کشان & که یابند زان کار یکسر نشان \\
بجنگ اندر آیند برسان کوه & بسنده نباشد مگر با گروه \\
برآهخت درع سیاوش ز سر & بخفتان هومان بپوشید بر \\
بران چرمهٔ پیل‌پیکر نشست & درفش سر نامداران بدست \\
برفت و بران دشت کرد آفرین & بران بخت بیدار و فرخ زمین \\
چو آن دیده‌بانان لشکر ز دور & درفش و نشان سپهدار تور \\
بدیدند زان دیده برخاستند & بشادی خروشیدن آراستند \\
طلایه هیونی برافگند زود & بنزدیک پیران بکردار دود \\
که هومان بپیروزی شهریار & دوان آمد از مرکز کارزار \\
درفش سپهدار ایران نگون & تنش غرقه مانده بخاک اندرون \\
همه لشکرش برگرفته خروش & بهومان نهاده سپهدار گوش \\
چو بیژن میان دو رویه سپاه & رسید اندران سایهٔ تاج و گاه \\
بتوران رسید آن زمان ترجمان & بگفت آنچ دید از بد بدگمان \\
هم آنگه بپیران رسید آگهی & که شد تیره آن فر شاهنشهی \\
سبک بیژن اندر میان سپاه & نگونسار کرد آن درفش سیاه \\
چو آن دیده‌بانان ایران سپاه & نگون یافتند آن درفش سیاه \\
سوی پهلوان روی برگاشتند & وزان دیده گه نعره برداشتند \\
وزآنجا هیونی بسان نوند & طلایه سوی پهلوان برفگند \\
که بیژن بپروزی آمد چو شیر & درفش سیه را سر آورده زیر \\
چو دیوانگان گیو گشته نوان & بهرسو خروشان و هر سو دوان \\
همی آگهی جست زان نیوپور & همی ماتم آورد هنگام سور \\
چو آگاهی آمد ز بیژن بدوی & دمان پیش فرزند بنهاد روی \\
چو چشمش بروی گرامی رسید & ز اسب اندر آمد چنان چون سزید \\
بغلتید و بنهاد بر خاک سر & همی آفرین خواند بر دادگر \\
گرفتش ببر باز فرزند را & دلیر و جوان و خردمند را \\
وزآنجا دمان سوی سالار شاه & ستایش کنان برگرفتند راه \\
چو دیدند مر پهلوان را ز دور & نبیره فرود آمد از اسب تور \\
پر از خون سلیح و پر از خاک سر & سرگرد هومان بفتراک بر \\
بپیش نیا رفت بیژن چو دود & همی یاد کرد آن کجا رفته بود \\
سلیح و سر و اسب هومان گرد & به پیش سپهدار گودرز برد \\
ز بیژن چنان شاد شد پهلوان & که گفتی برافشاند خواهد روان \\
گرفت آفرین پس بدادار بر & بران اختر و بخت بیدار بر \\
بگنجور فرمود پس پهلوان & که تاج آر با جامهٔ خسروان \\
گهربافته پیکر و بوم زر & درفشان چو خورشید تاج و کمر \\
ده اسب آوریدند زرین لگام & پری‌روی زرین کمر ده غلام \\
بدو داد و گفت از گه سام شیر & کسی ناورید اژدهایی بزیر \\
گشادی سپه را بدین جنگ دست & دل شاه ترکان بهم بر شکست \\
همه لشکر شاه ایران چو شیر & دمان و دنان بادپایان بزیر \\
وز اندوه پیران برآورد خشم & دل از درد خسته پر از آب چشم \\
بنستیهن آنگه فرستاد کس & که ای نامور گرد فریادرس \\
سزد گر کنی جنگ را تیز چنگ & بکین برادر نسازی درنگ \\
بایرانیان بر شبیخون کنی & زمین را بخون رود جیحون کنی \\
ببر ده هزار آزموده سوار & کمر بسته بر کینه و کارزار \\
مگر کین هومان تو بازآوری & سر دشمنان را بگاز آوری \\
چو رفتی بنزدیک لشکر فراز & سپه را یکی سوی هومان بساز \\
بدو گفت نستیهن ایدون کنم & که از خون زمین رود جیحون کنم \\
دو بهره چو از تیره شب درگذشت & ز جوش سواران بجوشید دشت \\
گرفتند ترکان همه تاختن & بدان تاختن گردن افراختن \\
چو نستیهن آن لشکر کینه‌خواه & بیاورد نزدیک ایران سپاه \\
سپیده‌دمان تا بدانجا رسید & چو از دیده گه دیده‌بانش بدید \\
چو کارآگهان آگهی یافتند & سبک سوی گودرز بشتافتند \\
که آمد سپاهی چو کوه روان & که گویی ندارند گویا زبان \\
بران سان که رسم شبیخون بود & سپهدار داند که آن چون بود \\
بلشکر بفرمود پس پهلوان & که بیدار باشید و روشن‌روان \\
بخواند آن زمان بیژن گیو را & ابا تیغ‌زن لشکر نیو را \\
بدو گفت نیک اختر و کام تو & شکسته دل دشمن از نام تو \\
ببر هرک باید ز گردان من & ازین نامداران و مردان من \\
پذیره شو این تاختن را چو شیر & سپاه اندر آورد به مردی بزیر \\
گزین کرد بیژن ز لشکر سوار & دلیران و پرخاشجویان هزار \\
رسیدند پس یک بدیگر فراز & دو لشکر پر از کینه و رزمساز \\
همه گرزها بر کشیدند پاک & یکی ابر بست از بر تیره خاک \\
فرود آمد از کوه ابر سیاه & بپوشید دیدار توران سپاه \\
سپهدار چون گرد تیره بدید & کزو لشکر ترک شد ناپدید \\
کمانها بفرمود کردن بزه & برآمد خروش از مهان و ز که \\
چو بیژن به نستیهن اندر رسید & درفش سر ویسگان را بدید \\
هوا سربسر گشته زنگارگون & زمین شد بکردار دریای خون \\
ز ترکان دو بهره فتاده نگون & بزیر پی اسب غرقه بخون \\
یکی تیر بر اسب نستیهنا & رسید از گشاد و بر بیژنا \\
ز درد اندر آمد تگاور بروی & رسید اندرو بیژن جنگجوی \\
عمودی بزد بر سر ترگ‌دار & تهی ماند ازو مغز و برگشت کار \\
چنین گفت بیژن بایرانیان & که هر کو ببندد کمر بر میان \\
بجز گرز و شمشیر گیرد بدست & کمان بر سرش بر کنم پاک پست \\
که ترکان بدیدن پری چهره‌اند & بجنگ از هنر پاک بی‌بهره‌اند \\
دلیری گرفتند کنداوران & کشیدند لشکر پرندآوران \\
چو پیلان همه دشت بر یکدگر & فگنده ز تنها جدا مانده سر \\
ازان رزمگه تا بتوران سپاه & دمان از پس اندر گرفتند راه \\
چو پیران ندید آن زمان با سپاه & برادر بدو گشت گیتی سیاه \\
بکارآگهان گفت زین رزمگاه & هیونی بتازد به آوردگاه \\
که آردنشانی ز نستیهنم & وگرنه دو دیده ز سر برکنم \\
هیونی برون تاختند آن زمان & برفت و بدید و بیامد دمان \\
که نستیهن آنک بدان رزمگاه & ابا نامداران توران سپاه \\
بریده سرافگنده بر سان پیل & تن از گرز خسته بکردار نیل \\
چو بشنید پیران برآمد بجوش & نماند آن زمان با سپهدار هوش \\
همی کند موی و همی ریخت آب & ازو دور شد خورد و آرام و خواب \\
بزد دست و بدرید رومی قبای & برآمد خروشیدن های های \\
همی گفت کای کردگار جهان & همانا که با تو بدستم نهان \\
که بگسست از بازوان زور من & چنین تیره شد اختر و هور من \\
دریغ آن هژبر افن گردگیر & جوان دلاور سوار هژیر \\
گرامی برادر جهانبان من & سر ویسگان گرد هومان من \\
چو نستیهن آن شیر شرزه بجنگ & که روباه بودی بجنگش پلنگ \\
کرا یابم اکنون بدین رزمگاه & بجنگ اندر آورد باید سپاه \\
بزد نای رویین و بربست کوس & هوا نیلگون شد زمین آبنوس \\
ز کوه کنابد برون شد سپاه & بشد روشنایی ز خورشید و ماه \\
سپهدار ایران بزد کرنای & سپاه اندر آورد و بگرفت جای \\
میان سپه کاویانی درفش & بپیش اندرون تیغهای بنفش \\
همه نامدارن پرخاشخر & ابا نیزه و گرزهٔ گاوسر \\
سپیده‌دمان اندر آمد سپاه & به پیکار تا گشت گیتی سیاه \\
برفتند زان پی به بنگاه خویش & بخیمه شد این، آن بخرگاه خویش \\
سپهدار ایران به زیبد رسید & از اندیشه کردن دلش بردمید \\
همی گفت کامروز رزمی گران & بکردیم و کشتیم ازیشان سران \\
گمانی برم زانک پیران کنون & دواند سوی شاه ترکان هیون \\
وزو یار خواهد بجنگ سپاه & رسانم کنون آگهی من بشاه \\
نویسندهٔ نامه را خواند و گفت & برآورد خواهم نهان از نهفت \\
اگر برگشایی تو لب را ز بند & زبان آورد بر سرت برگزند \\
یکی نامه فرمود نزدیک شاه & بگاه کردن ز کار سپاه \\
بخسرو نمود آن کجا رفته بود & سخن هرچ پیران بود گفته بود \\
فرستادن گیو و پیوند و مهر & نمودن بدو کار گردان سپهر \\
ز پاسخ که دادند مر گیو را & بزرگان و فرزانهٔ نیو را \\
وزان لشکری کز پسش چون پلنگ & بیاورد سوی کنابد بجنگ \\
ازان پس کجا رزمگه ساختند & وزان رزم دلرا بپرداختند \\
ز هومان و نستیهن جنگجوی & سراسر همه یاد کرد اندر اوی \\
ز کردار بیژن که روز نبرد & بدان گرزداران توران چه کرد \\
سخن سربسر چون همه گفته بود & ز پیکار و جنگ آن کجا رفته بود \\
بپردخت زان پس بافراسیاب & که با لشکر آمد بنزدیک آب \\
گر او از لب رود جیحون سپاه & بایران گذارد سپه را براه \\
تو دانی که با او نداریم پای & ایا فرخجسته جهان کدخدای \\
مگر خسرو آید بپشت سپاه & بسر بر نهد بندگانرا کلاه \\
ور ایدونک پیران کند دست پیش & بخواهد سپه یاور از شاه خویش \\
بخسرو رسد زان سپس آگهی & ک با او چه سازد ببختت رهی \\
و دیگر که از رستم دیو بند & ز لهراسب وز اشکش هوشمند \\
ز کردار ایشان به کهتر خبر & رساند مگر شاه پیروزگر \\
چو نامه بمهر اندر آورد و بند & بفرمود تا بر ستور نوند \\
تشستنگه خسروی ساختند & فراوان تگاور برون تاختند \\
بفرمود تا رفت پیشش هجیر & جوانی بکردار هشیار و پیر \\
بگفت آن سخن سربسر پهلوان & بپیش هشیوار پور جوان \\
بدو گفت کای پور هشیاردل & یکی تیز گردان بدین کاردل \\
اگر مر تو را نزد من دستگاه & همی جست باید کنونست گاه \\
چو بستانی این نامه هم در زمان & برو هم بکردار باد دمان \\
شب و روز ماسای و سر بر مخار & ببر نامهٔ من بر شهریار \\
بپدرود کردن گرفتش ببر & برون آمد از پیش فرخ پدر \\
ز لشکر دو تن را بر خویش خواند & سبکشان باسب تگاور نشاند \\
برون شد ز پرده‌سرای پدر & بهر منزلی بر هیونی دگر \\
خور و خواب و آرامشان بر ستور & چه تاریکی شب چه تابنده هور \\
بران گونه پویان براه آمدند & بیک هفته نزدیک شاه آمدند \\
چو از راه ایران بیامد سوار & کس آمد بر خسرو نامدار \\
پذیره فرستاد شماخ را & چه مایه دلیران گستاخ را \\
بپرسید چون دید روی هجیر & که ای پهلوان‌زادهٔ شیرگیر \\
درودست باری که بس ناگهان & رسیدی به نزدیک شاه جهان \\
بفرمود تا پرده برداشتند & باسبش ز درگاه بگذاشتند \\
هجیر اندر آمد چو خسرو بدوی & نگه کرد پیشش بمالید روی \\
بپرسید بسیار و بنشاندش & هزاران هجیر آفرین خواندش \\
ز گوهر یکی تاج پیروزه شاه & بسر بر نهادش چو رخشنده ماه \\
ز گودرز وز مهتران سپاه & ز هر یک یکایک بپرسید شاه \\
درود بزرگان بخسرو بداد & همه کار لشکر برو کرد یاد \\
بدو داد پس نامهٔ پهلوان & جوان خردمند روشن‌روان \\
نویسنده را پیش بنشاندند & بفرمود تا نامه برخواندند \\
چو برخواند نامه بخسرو دبیر & ز یاقوت رخشان دهان هجیر \\
بیاگند وزان پس بگنجور گفت & که دینار و دیبا بیار از نهفت \\
بیاورد بدره چو فرمان شنید & همی ریخت تا شد سرش ناپدید \\
بیاورد پس جامه زرنگار & چنانچون بود از در شهریار \\
همیدون ببردند پیش هجیر & ابا زین زرین ده اسب هژیر \\
بیارانش بر خلعت افگند نیز & درم داد و دینار و هرگونه چیز \\
ازان پس جو از جای برخاستند & نشستنگه می بیاراستند \\
هجیر و بزرگان خسروپرست & گرفتند یکسر همه می بدست \\
نشستند یک روز و یک شب بهم & همی رای زد خسرو از بیش و کم \\
بشبگیر خسرو سر و تن بشست & بپیش جهانداور آمد نخست \\
بپوشید نو جامهٔ بندگی & دو دیده چو ابری ببارندگی \\
دوتایی شده پشت و بنهاد سر & همی آفرین خواند بر دادگر \\
ازو خواست پیروزی و فرهی & بدو جست دیهیم و تخت مهی \\
بیزدان بنالید ز افراسیاب & بدرد از دو دیده فرو ریخت آب \\
وزآنجا بیامد چو سرو سهی & نشست از برگاه شاههنشهی \\
دبیر خردمند را پیش خواند & سخنهای بایسته با او براند \\
چو آن نامه را زود پاسخ نوشت & پدید آورید اندرو خوب و زشت \\
نخست آفرین کرد بر کردگار & کزو دید نیک و بد روزگار \\
دگر آفرین کرد بر پهلوان & که جاوید بادی و روشن‌روان \\
خجسته سپهدار بسیار هوش & همه رای و دانش همه جنگ و جوش \\
خداوند گوپال و تیغ بنفش & فروزندهٔ کاویانی درفش \\
سپاس از جهاندار یزدان ما & که پیروز بودند گردان ما \\
از اختر ترا روشنایی نمود & ز دشمن برآورد ناگاه دود \\
نخست آنک گفتی که مر گیو را & بزرگان فرزانه و نیو را \\
بنزدیک پیران فرستاده‌ام & چه مایه ورا پندها داده‌ام \\
نپذرفت ازان پس خود او پند من & نجست اندرین کار پیوند من \\
سپهبد یکی داستان زد برین & چو دستور پیشین برآورد کین \\
که هر مهتری کو روان کاستست & ز نیکی ببخت بد آراستست \\
مرا زان سخن پیش بود آگهی & که پیران دل از کین نخواهد تهی \\
ولیکن ازان خوب کردار او & نجستم همی ژرف پیکار او \\
کنون آشکارا نمود این سپهر & که پیران بتوران گراید بمهر \\
کنون چون نبیند جز افراسیاب & دلش را تو از مهر او برمتاب \\
گر او بر خرد برگزیند هوا & بکوشش نروید ز خاراگیا \\
تو با دشمن ار خوب گویی رواست & از آزادگان خوب گفتن سزاست \\
و دیگر ز پیکار جنگ‌آوران & کجا یاد کردی به گرز گران \\
ز نیک‌اختر و گردش هور و ماه & ز کوشش نمودن بران رزمگاه \\
مرا این درستست کز کار کرد & تو پیروز باشی بروز نبرد \\
نبیره کجا چون تو دارد نیا & بجنگ اندرون باشدش کیمیا \\
ز شیران چه زاید مگر نره شیر & چنانچون بود نامدار و دلیر \\
به بیداد برنیست این کار تو & بسندست یزدان نگهدار تو \\
تو زور و دلیری ز یزدان شناس & ازو دار تا زنده باشی سپاس \\
سدیگر که گفتی که افراسیاب & سپه را همی بگذارند ز آب \\
ز پیران فرستاده شد نزد اوی & سپاهش بایران نهادست روی \\
همانست یکسر که گفتی سخن & کنون باز پاسخ فگندیم بن \\
بدان ای پر اندیشه سالار من & بهر کار شایستهٔ کار من \\
که او بر لب رود جیحون درنگ & نه ازان کرد کید بر ما بجنگ \\
که خاقان برو لشکر آرد ز چین & فراز آمدش از دو رویه کمین \\
و دیگر که از لشکران گران & پراگنده برگرد توران سران \\
بدو دشمن آمد ز هر سو پدید & ازان بر لب رود جیحون کشید \\
بپنجم سخن کگهی خواستی & بمهر گوان دل بیاراستی \\
چو لهراسب و چون اشکش تیزچنگ & چو رستم سپهبد دمنده نهنگ \\
بدان ای سپهدار و آگاه باش & بهر کار با بخت همراه باش \\
کزان سو که شد رستم شیرمرد & ز کشمیر و کابل برآورد گرد \\
وزان سو که شد اشکش تیزهوش & برآمد ز خوارزم یکسر خروش \\
برزم اندرون شیده برگشت ازوی & سوی شهر گرگان نهادست روی \\
وزان سو که لهراسب شد با سپاه & همه مهتران برگشادند راه \\
الانان و غز گشت پرداخته & شد آن پادشاهی همه ساخته \\
گر افراسیاب اندر آید براه & زجیحون بدین سو گذارد سپاه \\
بگیرند گردان پس پشت اوی & نماند بجز باد در مشت اوی \\
تو بشناس کو شهر آباد خویش & بر و بوم و فرخنده بنیاد خویش \\
بگفتار پیران نماند بجای & بدشمن سپارد نهد پیش پای \\
نجنباند او داستان را دو لب & که ناید خبر زو بمن روز و شب \\
بدان روز هرگز مبادا درود & که او بگذراند سپه را ز رود \\
بما برکند پیشدستی بجنگ & نبیند کس این روز تاریک و تنگ \\
بفرمایم اکنون که بر پیل کوس & ببندد دمنده سپهدار طوس \\
دهستان و گرگان و آن بوم و بر & بگیرد برآرد بخورشید سر \\
من اندر پی طوس با پیل و گاه & بیاری بیایم بپشت سپاه \\
تو از جنگ پیران مبر تاب روی & سپه را بیارای و زو کینه‌جوی \\
چو هومان و نستیهن از پشت اوی & جدا ماند شد باد در مشت اوی \\
گر از نامداران ایران نبرد & بخواهد بفرما وزان برمگرد \\
چو پیران نبرد تو جوید دلیر & کمن بددلی پیش او شو چو شیر \\
به پیکار مندیش ز افراسیاب & بجای آرد دل روی ازو برمتاب \\
چو آید بجنگ اندرون جنگجوی & نباید که برتابی از جنگ روی \\
بریشان تو پیروز باشی بجنگ & نگر دل نداری بدین کار تنگ \\
چنین دارم اومید از کردگار & که پیروز باشی تو در کارزار \\
همیدون گمانم که چون من ز راه & بپشت سپاه اندر آرم سپاه \\
بریشان شما رانده باشید کام & به خورشید تابان برآورده نام \\
ز کاوس وز طوس نزد سپاه & درود فراوان فرستاد شاه \\
بران نامه بنهاد خسرو نگین & فرستاده را داد و کرد آفرین \\
چو از پیش خسرو برون شد هجیر & سپهبد همی رای زد با وزیر \\
ز بس مهربانی که بد بر سپاه & سراسر همه رزم بد رای شاه \\
همی گفت اگر لشکر افراسیاب & بجنباند از جای و بگذارد آب \\
سپاه مرا بگسلاند ز جای & مرا رفت باید همینست رای \\
همانگه شه نوذران را بخواند & بفرمود تا تیز لشکر براند \\
بسوی دهستان سپه برکشید & همه دشت خوارزم لشکر کشید \\
نگهبان لشکر بود روز جنگ & بجنگ اندر آید بسان پلنگ \\
تبیره برآمد ز درگاه طوس & خروشیدن نای رویین و کوس \\
سپاه و سپهبد برفتن گرفت & زمین سم اسبان نهفتن گرفت \\
تو گفتی که خورشید تابان بجای & بماند از نهیب سواران بپای \\
دو هفته همی رفت زان سان سپاه & بشد روشنایی ز خورشید و ماه \\
پراگنده بر گرد کشور خبر & ز جنبیدن شاه پیروزگر \\
چو طوس از در شاه ایران برفت & سبک شاه رفتن بسیچید تفت \\
ابا ده هزار از گزیده سران & همه نامداران و کنداوران \\
بنزدیک گودرز بنهاد روی & ابا نامداران پرخاشجوی \\
ابا پیل و با کوس و با فرهی & ابا تخت و با تاج شاهنشهی \\
هجیر آمد از پیش خسرودمان & گرازان و خندان و دل شادمان \\
ابا خلعت و خوبی و خرمی & تو گفتی همی برنوردد زمی \\
چو آمد به نزدیک پرده‌سرای & برآمد خروشیدن کرنای \\
پذیره شدندش سران سربسر & زمین پر ز آهن هوا پر ز زر \\
چو خیزد بچرخ اندرون داوری & ز ماه و ز ناهید وز مشتری \\
بیاراست لشکر چو چشم خروس & ابا زنگ زرین و پیلان و کوس \\
چو آمد بر نامور پهلوان & بگفت آنچ دید از شه خسروان \\
نوازیدن شاه و پیوند اوی & همی گفت از رادی و پند اوی \\
که چون بر سپه گستریدست مهر & چگونه ز پیغام بگشاد چهر \\
پس آن نامهٔ شهریار جهان & بگودرز داد و درود مهان \\
نوازیدن شاه بشنید ازوی & بمالید بر نامه بر چشم و روی \\
چو بگشاد مهرش بخواننده داد & سخنها برو کرد خواننده یاد \\
سپهدار بر شاه کرد آفرین & بفرمان ببوسید روی زمین \\
ببود آن شب و رای زد با پسر & بشبگیر بنشست و بگشاد در \\
همه نامداران لشگر پگاه & برفتند بر سر نهاده کلاه \\
پس آن نامهٔ شاه، فرخ هجیر & بیاورد و بنهاد پیش دبیر \\
دبیر آن زمان پند و فرمان شاه & ز نامه همی خواند پیش سپاه \\
سپهدار رزی دهان را بخواند & بدیوان دینار دادن نشاند \\
ز اسبان گله هرچ بودش به کوه & بلشکر گه آورد یکسر گروه \\
در گنج دینار و تیغ و کمر & همان مایه‌ور جوشن و خود زر \\
بروزی دهان داد یکسر کلید & چو آمد گه نام جستن پدید \\
برافشاند بر لشکر آن خواسته & سوار و پیاده شد آراسته \\
یکی لشکری گشن برسان کوه & زمین از پی بادپایان ستوه \\
دل شیر غران ازیشان به بیم & همه غرقه در آهن و زر و سیم \\
بفرمودشان جنگ را ساختن & دل و گوش دادن بکین آختن \\
برفتند پیش سپهبد گروه & بر انبوه لشکر بکردار کوه \\
بریشان نگه کرد سالار مرد & زمین تیره دید آسمان لاژورد \\
چنین گفت کز گاه رزم پشین & نیاراست کس رزمگاهی چنین \\
باسب و سلیح و بسیم و بزر & بپیلان جنگی و شیران نر \\
اگر یار باشد جهان‌آفرین & نپیچیم از ایدر عنان تا بچین \\
چو بنشست فرزانگان را بخواند & ابا نامداران برامش نشاند \\
همی خورد شادی‌کنان دل بجای & همی با یلان جنگ را کرد رای \\
بپیران رسید آگهی زین سخن & که سالار ایران چه افگند بن \\
ازان آگهی شد دلش پرنهیب & سوی چاره برگشت و بند و فریب \\
ز دستور فرخنده رای آنگهی & بجست اندر آن کینه جستن رهی \\
یکی نامه فرمود پس تا دبیر & نویسد سوی پهلوان دلپذیر \\
سر نامه کرد آفرین بزرگ & بیزدان پناهش ز دیو سترگ \\
دگر گفت کز کردگار جهان & بخواهم همی آشکار و نهان \\
مگر کز میان تو رویه سپاه & جهاندار بردارد این کینه‌گاه \\
اگر تو که گودرزی آن خواستی & که گیتی بکینه بیاراستی \\
برآمد ازین کینه گه کام تو & چه گویی چه باشد سرانجام تو \\
نگه کن که چندان دلیران من & ز خویشان نزدیک و شیران من \\
تن بی سرانشان فگندی بخاک & ز یزدان نداری همی شرم و باک \\
ز مهر و خرد روی برتافتی & کنون آنچ جستی همه یافتی \\
گه آمد که گردی ازین کینه سیر & بخون ریختن چند باشی دلیر \\
نگه کن کز ایران و توران سوار & چه مایه تبه شد بدین کارزار \\
بکین جستن مرده‌ای ناپدید & سر زندگان چند باید برید \\
گه آمد که بخشایش آید ترا & ز کین جستن آسایش آید ترا \\
اگر بازیابی شده روزگار & بگیتی درون تخم کینه مکار \\
روانت مرنجان و مگذار تن & ز خون ریختن بازکش خویشتن \\
پس از مرگ نفرین بود بر کسی & کزو نام زشتی بماند بسی \\
نباید که زشتی بماندت نام & وگر تو بدان سر شوی شادکام \\
هر آنگه که موی سیه شد سپید & ببودن نماند فراوان امید \\
بترسم که گر بار دیگر سپاه & بجنگ اندر آید بدین رزمگاه \\
نبینی ز هر دو سپه کس بپای & برفته روان تن بمانده بجای \\
ازان پس که داند که پیروز کیست & نگون‌بخت گر گیتی افروز کیست \\
ور ایدونک پیکار و خون ریختن & بدین رزمگه با من آویختن \\
کزین سان همی جنگ شیران کنی & همی از پی شهر ایران کنی \\
بگو تا من اکنون هم اندر شتاب & نوندی فرستم بافراسیاب \\
بدان تا بفرمایدم تا زمین & ببخشم و پس در نوردیم کین \\
چنانچون بگاه منوچهر شاه & ببخشش همی داشت گیتی نگاه \\
هران شهر کز مرز ایران نهی & بگو تا کنیم آن ز ترکان تهی \\
وز آباد و ویران و هر بوم و بر & که فرمود کیخسرو دادگر \\
از ایران بکوه اندر آید نخست & در غرچگان از بر بوم بست \\
دگر طالقان شهر تا فاریاب & همیدون در بلخ تا اندر آب \\
دگر پنجهیر و در بامیان & سر مرز ایران و جای کیان \\
دگر گوزگانان فرخنده جای & نهادست نامش جهان کدخدای \\
دگر مولیان تا در بدخشان & همینست ازین پادشاهی نشان \\
فروتر دگر دشت آموی و زم & که با شهر ختلان براید برم \\
چه شگنان وز ترمذ ویسه گرد & بخارا و شهری که هستش بگرد \\
همیدون برو تا در سغد نیز & نجوید کس آن پادشاهی بنیز \\
وزان سو که شد رستم گرد سوز & سپارم بدو کشور نیمروز \\
ز کوه و ز هامون بخوانم سپاه & سوی باختر برگشاییم راه \\
بپردازم این تا در هندوان & نداریم تاریک ازین پس روان \\
ز کشمیر وز کابل و قندهار & شما را بود آن همه زین شمار \\
وزان سو که لهراسب شد جنگجوی & الانان و غر در سپارم بدوی \\
ازین مرز پیوسته تا کوه قاف & بخسرو سپاریم بی‌جنگ و لاف \\
وزان سو که اشکش بشد همچنین & بپردازم اکنون سراسر زمین \\
وزان پس که این کرده باشم همه & ز هر سو بر خویش خوانم رمه \\
بسوگند پیمان کنم پیش تو & کزین پس نباشم بداندیش تو \\
بدانی که ما راستی خواستیم & بمهر و وفا دل بیاراستیم \\
سوی شاه ترکان فرستم خبر & که ما را ز کینه بپیچید سر \\
همیدون تو نزدیک خسرو بمهر & یکی نامه بنویس و بنمای چهر \\
چنین از ره مهر و پیکار من & ز خون ریختن با تو گفتار من \\
چو پیمان همه کرده باشیم راست & ز من خواسته هرچ خسرو بخواست \\
فرستم همه سربسر نزد شاه & در کین ببندد مگر بر سپاه \\
ازان پس که این کرده باشیم نیز & گروگان فرستاده و داده چیز \\
بپیوندم این هر و آیین و دین & بدوزم بدست وفا چشم کین \\
که بشکست هنگام شاه بزرگ & ز بد گوهر تور و سلم سترگ \\
فریدون که از درد سرگشته شد & کجا ایرج نامور کشته شد \\
ز من هرچ باید بنیکی بخواه & ازان پس برین نامه کن نزد شاه \\
نباید کزین خوب گفتار من & بسستی گمانی برند انجمن \\
که من جز بمهر این نگویم همی & سرانجام نیکی بجویم همی \\
مرا گنج و مردان از آن تو بیش & بمردانگی نام از آن تو پیش \\
ولیکن بدین کینه انگیختن & به بیداد هر جای خون ریختن \\
بسوزد همی بر سپه بر دلم & بکوشم که کین از میان بگسلم \\
سه دیگر که از کردگار جهان & بترسم همی آشکار و نهان \\
که نپسندد از ما بدی دادگر & گزافه نبردارد این شور و شر \\
اگر سر بپیچی ز گفتار من & نجویی همه ژرف کردار من \\
گنهکار دانی مرا بی‌گناه & نخواهی بگفتار کردن نگاه \\
کجا داد و بیداد نزدت یکیست & جز از کینه گستردنت رای نیست \\
گزین کن ز گردان ایران سران & کسی کو گراید برگرز گران \\
همیدون من از لشکر خویش مرد & گزینم چو باید ز بهر نبرد \\
همه یک بدیگر فرازآوریم & سران را ز سر سوی گاز آوریم \\
همیدون من و تو به آوردگاه & بگردیم یک با دگر کینه‌خواه \\
مگر بیگناهان ز خون ریختن & بسایش آیند ز آویختن \\
کسی کش گنهکار داری همی & وزو بر دل آزار داری همی \\
بپیش تو آرم بروز نبرد & ببایدت پیمان یکی نیز کرد \\
که بر ما تو گر دست یابی بخون & شود بخت گردان ترکان نگون \\
نیازاری از بن سپاه مرا & نسوزی بر و بوم و گاه مرا \\
گذرشان دهی تا بتوران شوند & کمین را نسازی بریشان کمند \\
وگر من شوم بر تو پیروزگر & دهد مر مرا اختر نیک بر \\
نسازم بایرانیان بر کمین & نگیریم خشم و نجوییم کین \\
سوی شهر ایران دهم راهشان & گذارم یکایک سوی شاهشان \\
ازیشان نگردد یکی کاسته & شوند ایمن از جان وز خواسته \\
ور ایدونک زینسان نجویی نبرد & دگرگونه خواهی همی کار کرد \\
بانبوه جویی همی کارزار & سپه را سراسر بجنگ اند آر \\
هران خون که آید بکین ریخته & تو باشی بدان گیتی آویخته \\
ببست از بر نامه بر بند را & بخواند آن گرانمایه فرزند را \\
پسر بد مر او را سر انجمن & یکی نام رویین و رویینه تن \\
بدو گفت نزدیک گودرز شو & سخن گوی هشیار و پاسخ شنو \\
چو رویین برفت از در نامور & فرستاده با ده سوار دگر \\
بیامد خردمند روشن‌روان & دمان تا سراپردهٔ پهلوان \\
چو رویین پیران بدرگه رسید & سوی پهلوان سپه کس دوید \\
فرستاده را خواند پس پهلوان & دمان از پس پرده آمد جوان \\
بیامد چو گودرز را دید دست & بکش کرد و سر پیش بنهاد پست \\
سپهدار بر جست و او را چو دود & بغوش تنگ اندر آورد زود \\
ز پیران بپرسید وز لشکرش & ز گردان وز شاه وز کشورش \\
خردمند رویین پس آن نامه پیش & بیاورد و بگزارد پیغام خویش \\
دبیر آمد و نامه برخواند زود & بگودرز گفت آنچ در نامه بود \\
چو نامه بگودرز برخواندند & همه نامداران فرو ماندند \\
ز بس چرب گفتار و ز پند خوب & نمودن بدو راه و پیوند خوب \\
خردمند پیران که در نامه یاد & چه آورد وز پند نیکو چه داد \\
برویین چنین گفت پس پهلوان & که‌ای پور سالار و فرخ جوان \\
تومهمان ما بود باید نخست & پس این پاسخ نامه بایدت جست \\
سراپردهٔ نو بپرداختند & نشستنگه خسروی ساختند \\
بدیبای رومی بیاراستند & خورشها و رامشگران خواستند \\
پراندیشه گشته دل پهلوان & نبشته ابا رای‌زن موبدان \\
همی پاسخ نامه آراستند & سخن هرچ نیکوتر آن خواستند \\
بیک هفته گودرز با رود و می & همی نامه را پاسخ افگند پی \\
ز بالا چو خورشید گیتی فروز & بگشتی سپهبد گه نیم‌روز \\
می و رود و مجلس بیاراستی & فرستاده را پیش خود خواستی \\
چو یک هفته بگذشت هشتم پگاه & نویسنده را خواند سالار شاه \\
بفرمود تا نامه پاسخ نوشت & درختی بنوی بکینه بگشت \\
سرنامه کرد آفرین از نخست & دگر پاسخ آورد یکسر درست \\
که بر خواندم نامه را سربسر & شنیدیم گفتار تو در بدر \\
رسانید رویین بر ما پیام & یکایک همه هرچ بردی تو نام \\
ولیکن شگفت آمدم کار تو & همی زین چنین چرب گفتار تو \\
دلت با زبان هیچ همسایه نیست & روان ترا از خرد مایه نیست \\
بهرجای چربی بکار آوری & چنین تو سخن پرنگار آوری \\
کسی را که از بن نباشد خرد & گمان بر تو بر مهربانی برد \\
چو شوره زمینی که از دور آب & نماید چو تابد برو آفتاب \\
ولیکن نه گاه فریبست و بند & که هنگام گرزست و تیغ و کمند \\
مرا با تو جز کین و پیکار نیست & گه پاسخ و روز گفتار نیست \\
نگر تا چه سان گردد اکنون سپهر & نه جای فریبست و پیوند و مهر \\
کرا داد خواهد جهاندار زور & کرا بردهد بخت پیروز هور \\
ولیکن بدین گفته پاسخ شنو & خرد یاد کن بخت را پیشرو \\
نخست آنک گفتی که از مهر نیز & ز یزدان وز گردش رستخیز \\
نخواهم که آید مرا پیش جنگ & دلم گشت ازین کار بیداد تنگ \\
دلت با زبان آشنایی نداشت & بدان گه که این گفته بر دل گماشت \\
اگر داد بودی بدلت اندرون & ترا پیشدستی نبودی بخون \\
که ز آغاز کار اندر آمد نخست & نبودی بخون ریختن هیچ سست \\
نخستین که آمد بپیش تو گیو & از ایران هشیوار مردان نیو \\
بسازیده مر جنگ را لشکری & ز کشور دمان تا دگر کشوری \\
تو کردی همه جنگ را دست پیش & سپه را تو برکندی از جای خویش \\
خرد، ار پس آمد تو پیش آمدی & بفرجام آرام بیش آمدی \\
ولیکن سرشت بد و خوی بد & ترانگذراند براه خرد \\
بدی خود بدان تخمه در گوهرست & ببد کردن آن تخمه اندر خورست \\
شنیدی که بر ایرج نیک‌بخت & چه آمد ز تور از پی تاج و تخت \\
چو از تور و سلم اندر آمد زمین & سراسر بگسترد بیداد و کین \\
فریدون که از درد دل روز و شب & گشادی بنفرین ایشان دو لب \\
بافراسیاب آمد آن مهر بد & ازان نامداران اندک خرد \\
ز سر با منوچهر نو کین نهاد & همیدون ابا نوذر و کیقباد \\
بکاوس کی کرد خود آنچ کرد & برآورد از ایران آباد گرد \\
ازان پس بکین سیاوش باز & فگند این چنین کینهٔ نو دارز \\
نیامد بدانگه ترا داد یاد & که او بی‌گنه جان شیرین بداد \\
جه مایه بزرگان که از تخت و گاه & از ایران شدند اندرین کین تباه \\
و دیگر که گفتی که با پیر سر & بخون ریختن کس نبندد کمر \\
بدان ای جهاندیدهٔ پرفریب & بهر کار دیده فراز و نشیب \\
که یزدان مرا زندگانی دراز & بدان داد با بخت گردن‌فراز \\
که از شهر توران بروز نبرد & ز کینه برآرم بخورشید گرد \\
بترسم همی زانک یزدان من & ز تن بگسلاند مگر جان من \\
من این کینه را ناوریده بجای & بر و بومتان ناسپرده بپای \\
سدیگر که گفتی ز یزدان پاک & نبینم بدلت اندرون بیم و باک \\
ندانی کزین خیره خون ریختن & گرفتار کردی بفرجام تن \\
من اکنون بدین خوب گفتار تو & اگر باز گردم ز پیکار تو \\
بهنگام پرسش ز من کردگار & بپرسد ازین گردش روزگار \\
که سالاری و گنج و مردانگی & ترا دادم و زور و فرزانگی \\
بکین سیاوش کمر بر میان & نبستی چرا پیش ایرانیان \\
بهفتاد خون گرامی پسر & بپرسد ز من داور دادگر \\
ز پاسخ بپیش جهان‌آفرین & چه گویم چرا بازگشتم ز کین \\
ز کار سیاوش چهارم سخن & که افگندی ای پیر سالار بن \\
که گفتی ز بهر تنی گشته خاک & نشاید ستد زنده را جان پاک \\
تو بشناس کین زشت کردارها & بدل پر ز هر گونه آزارها \\
که با شهر ایران شما کرده‌اید & چه مایه کیان را بیازرده‌اید \\
چه پیمان شکستن چه کین ساختن & همیشه بسوی بدی تاختن \\
چو یاد آورم چون کنم آشتی & که نیکی سراسر بدی کاشتی \\
بپنجم که گفتی که پیمان کنم & ز توران سران را گروگان کنم \\
بنزدیک خسرو فرستیم گنج & ببندیم بر خویشتن راه رنج \\
بدان ای نگهبان توران سپاه & که فرمان جز اینست ما را ز شاه \\
مرا جنگ فرمود و آویختن & بکین سیاوش خون ریختن \\
چو فرمان خسرو نیارم بجای & روان شرم دارد بدیگر سرای \\
ور اومید داری که خسرو بمهر & گشاید برین گفتها بر تو چهر \\
گروگان و آن خواسته هرچ هست & چو لهاک و رویین خسروپرست \\
گسی کن بزودی بنزدیک شاه & سوی شهر ایران گشادست راه \\
ششم شهر ایران که کردی تو یاد & برو و بوم آباد فرخ‌نژاد \\
سپاریم گفتی بخسرو همه & ز هر سو بر خویش خوانم رمه \\
تراکرد یزدان ازان بی‌نیاز & گر آگه نه‌ای تا گشاییم راز \\
سوی باختر تا بمرز خزر & همه گشت لهراسب را سربسر \\
سوی نیمروز اندرون تا بسند & جهان شد بکردار روی پرند \\
تهم رستم نیو با تیغ تیز & برآورد ازیشان دم رستخیز \\
سر هندوان با درفش سیاه & فرستاد رستم بنزدیک شاه \\
دهستان و خوارزم و آن بوم و بر & که ترکان برآورده بودند سر \\
بیابان ازیشان بپرداختند & سوی باختر تاختن ساختند \\
ببارید بر شیده اشکش تگرگ & فراز آوریدش بنزدیک مرگ \\
اسیران وز خواسته چند چیز & فرستاد نزدیک خسرو بنیز \\
وزین سو من و تو به جنگ اندریم & بدین مرکز نام و ننگ اندریم \\
بیک جنگ دیدی همه دستبرد & ازین نامداران و مردان گرد \\
ور ایدونک روی اندر آری بروی & رهانم ترا زین همه گفت و گوی \\
بنیروی یزدان و فرمان شاه & بخون غرقه گردانم این رزمگاه \\
تو ای نامور پهلوان سپاه & نگه کن بدین گردش هور و ماه \\
که بند سپهری فراز آمدست & سربخت ترکان بگاز آمدست \\
نگر تا ز کردار بدگوهرت & چه آرد جهان‌آفرین بر سرت \\
زمانه ز بد دامن اندر کشید & مکافات بد را بد آید پدید \\
تو بندیش هشیار و بگشای گوش & سخن از خردمند مردم نیوش \\
بدان کین چنین لشکر نامدار & سواران شمشیرزن صدهزار \\
همه نامجوی و همه کینه‌خواه & بافسون نگردند ازین رزمگاه \\
زمانه برآمد به هفتم سخن & فگندی وفا را بسوگند بن \\
بپیمان مرا با تو گفتار نیست & خرد را روانت خریدار نیست \\
ازیراک باهرک پیمان کنی & وفا را بفرجام هم بشکنی \\
بسوگند تو شد سیاوش بباد & بگفتار بر تو کس ایمن مباد \\
نبودیش فریادرس روز درد & چه مایه بسختی ترا یاد کرد \\
به هشتم که گفتی مرا تاج و تخت & از آن تو بیشست مردی و بخت \\
همیدون فزونم بمردان و گنج & ولیکن دلم را ز مهرست رنج \\
من ایدون گمانم که تا این زمان & بجنگ آزمودی مرا بی‌گمان \\
گرم بی‌هنر یافتی روز کین & تو دانی کنون بازم از پس ببین \\
بفرجام گفتی ز مردان مرد & تنی چند بگزین ز بهر نبرد \\
من از لشکر ترک هم زین نشان & بیارم سواران مردم‌کشان \\
که از مهربانی که بر لشکرم & نخواهم که بیداد کین گسترم \\
تو با مهربانی نهی پای پیش & که دانی نهان دل و رای خویش \\
بیازارد از من جهاندار شاه & گر از یکدگر بگسلانم سپاه \\
نهم آنک گفتی مبارز گزین & که با من بگردد برین دشت کین \\
یکی لشکری پرگنه پیش من & پرآزار ازیشان دل انجمن \\
نباشد ز من شاه همداستان & کزیشان بگردم بدین داستان \\
نخستین بانبوه زخمی چو کوه & بباید زدن سر بر همگروه \\
میان دو لشکر دو صف برکشید & گر ایدونک پیروزی آید پدید \\
وگرنه همین نامداران مرد & بیاریم و سازیم جای نبرد \\
ازین گفته گر بگسلی باز دل & من از گفتهٔ خود نیم دلگسل \\
ور ایدونک با من به آوردگاه & بسنده نخواهی بدن با سپاه \\
سپه خواه و یاور ز سالار خویش & بژرفی نگه‌دار پیکار خویش \\
پراگنده از لشکرت خستگان & ز خویشان نزدیک و پیوستگان \\
بمان تا کندشان پزشکان درست & زمان جستن اکنون بدین کار تست \\
اگر خواهی از من زمان درنگ & وگر جنگ جویی بیارای جنگ \\
بدان گفتم این تا بروز نبرد & بما بر بهانه نبایدت کرد \\
که ناگاه با ما بجنگ آمدی & کمین کردی و بی‌درنگ آمدی \\
من این کین اگر تا بصد سالیان & بخواهم همانست و اکنون همان \\
ازین کینه برگشتن امید نیست & شب و روز بی‌دیدگان را یکیست \\
چو آن پاسخ نامه گشت اسپری & فرستاده آمد بسان پری \\
کمر بر میان با ستور نوند & ز مردان به گرد اندرش نیز چند \\
فرود آمد از باره رویین گرد & گوان را همه پیش گودرز برد \\
سپهبد بفرمود تا موبدان & زلشکر همه نامور بخردان \\
بزودی سوی پهلوان آمدند & خردمند و روشن‌روان آمدند \\
پس آن پاسخ نامه پیش گوان & بفرمود خواندن همی پهلوان \\
بزرگان که آن نامهٔ دلپذیر & شنیدند گفتار فرخ دبیر \\
هش و رای پیران تنک داشتند & همه پند او را سبک داشتند \\
بگودرز بر آفرین خواند & ورا پهلوان گزین خواندند \\
پس آن نامه را مهر کرد و بداد & برویین پیران ویسه‌نژاد \\
چو از پیش گودرز برخاستند & بفرمود تا خلعت آراستند \\
از اسبان تازی بزرین ستام & چه افسر چه شمشیر زرین نیام \\
ببخشید یارانش را سیم و زر & کرا در خور آمد کلاه و کمر \\
برفت از در پهلوان با سپاه & سوی لشکر خویش بگرفت راه \\
چو رویین بنزدیک پیران رسید & بپیش پدر شد چنانچون سزید \\
بنزدیک تختش فرو برد سر & جهاندیده پیران گرفتش ببر \\
چو بگزارد پیغام سالار شاه & بگفت آنچ دید اندران رزمگاه \\
پس آن نامه برخواند پیشش دبیر & رخ پهلوان سپه شد چو قیر \\
دلش گشت پردرد و جان پرنهیب & بدانست کآمد بتنگی نشیب \\
شکیبایی و خامشی برگزید & بکرد آن سخن بر سپه ناپدید \\
ازان پس چنین گفت پیش سپاه & که گودرز را دل نیامد براه \\
ازان خون هفتاد پور گزین & نیارامدش یک زمان دل ز کین \\
گر ایدونک او بر گذشته سخن & بنوی همی کینه سازد ز بن \\
چرا من بکین برادر کمر & نبندم نخارم ازین کینه سر \\
هم از خون نهصد سر نامدار & که از تن جدا شد گه کارزار \\
که اندر بر و بوم ترکان دگر & سواری چو هومان نبندد کمر \\
چو نستیهن آن سرو سایه فگن & که شد ناپدید از همه انجمن \\
بباید کنون بست ما را کمر & نمانم بایرانیان بوم و بر \\
بنیروی یزدان و شمشیر تیز & برآرم ازان انجمن رستخیز \\
از اسبان گله هرچ شایسته بود & ز هر سو بلشکر گه آورد زود \\
پیاده همه کرد یکسر سوار & دو اسبه سوار از پس کارزار \\
سرگنجهای کهن برگشاد & بدینار دادن دل اندر نهاد \\
چو این کرده شد نزد افراسیاب & نوندی برافگند هنگام خواب \\
فرستاده‌ای با هش و رای پیر & سخن‌گوی و گرد و سوار و دبیر \\
که رو شاه توران سپه را بگوی & که ای دادگر خسرو نامجوی \\
کز آنگه که چرخ سپهر بلند & بگشت از بر تیره خاک نژند \\
چو تو شاه بر گاه ننشست نیز & به کس نام شاهی نپیوست نیز \\
نه زیبا بود جز تو مر تخت را & کلاه و کمر بستن و بخت را \\
ازان کس برآرد جهاندار گرد & که پیش تو آید بروز نبرد \\
یکی بنده‌ام من گنهکار تو & کشیده سر از جان بیدار تو \\
ز کیخسرو از من بیازرد شاه & جزین خویشتن را ندانم گناه \\
که این ایزدی بود بود آنچ بود & ندارد ز گفتار بسیار سود \\
اگر نیز بیند مرا زین گناه & کند گردن آزاد و آید براه \\
رسانم من اکنون بشاه آگهی & که گردون چه آورد پیش رهی \\
کشیدم بکوه کنابد سپاه & بایرانیان بر ببستیم راه \\
وزان سو بیامد سپاهی گران & سپهدار گودرز و با او سران \\
کز ایران ز گاه منوچهر شاه & فزون زان نیامد بتوران سپاه \\
به زیبد یکی جایگه ساختند & سپه را دران کوه بنشاختند \\
سپه را سه روز و سه شب چون پلنگ & بروی اندر آورده بد روی تنگ \\
نجستیم رزم اندران کینه‌گاه & که آید مگر سوی هامون سپاه \\
نیامد سپاهش ازان که برون & سر پهلوانان ما شد نگون \\
سپهدار ایران نیامد ستوه & بهامون نیاورد لشکر ز کوه \\
برادر جهاندار هومان من & بکینه بجوشید ازین انجمن \\
بایران سپه شد که جوید نبرد & ندانم چه آمد بران شیرمرد \\
بیامد بکین جستنش پور گیو & بگردید با گرد هومان نیو \\
ابر دست چون بیژنی کشته شد & سر من ز تیمار او گشته شد \\
که دانست هرگز که سرو بلند & بباغ از گیا یافت خواهد گزند \\
دل نامداران همه بر شکست & همه شادمانی شد از درد پست \\
و دیگر چو نستیهن نامدار & ابا ده هزار آزموده سوار \\
برفت از بر من سپیده دمان & همان بیژنش کند سر در زمان \\
من از درد دل برکشیدم سپاه & غریوان برفتم به آوردگاه \\
یکی رزم تا شب برآمد ز کوه & بکردیم یک با دگر همگروه \\
چو نهصد تن از نامداران شاه & سر از تن جدا شد برین رزمگاه \\
دو بهره ز گردان این انجمن & دل از درد خسته بشمشیر تن \\
بما بر شده چیره ایرانیان & بکینه همه پاک بسته میان \\
بترسم همی زانک گردان سپهر & بخواهد بریدن ز ما پاک مهر \\
وزان پس شنیدم یکی بدخبر & کزان نیز برگشتم آسیمه سر \\
که کیخسرو آید همی با سپاه & بپشت سپهبد بدین رزمگاه \\
گرایدونک گردد درست این خبر & که خسرو کند سوی ما برگذر \\
جهاندار داند که من با سپاه & نیارم شدن پیش او کینه‌خواه \\
مگر شاه با لشکر کینه‌جوی & نهد سوی ایران بدین کینه‌روی \\
بگرداند این بد ز تورانیان & ببندد بکینه کمر بر میان \\
که گر جان ما را ز ایران سپاه & بد آید نباشد کسی کینه‌خواه \\
فرستاده گفت پیران شنید & بکردار باد دمان بردمید \\
مشست از بر بادپای سمند & بکردار آتش هیونی بلند \\
بشد تا بنزدیک افراسیاب & نه دم زد بره بر نه آرام و خواب \\
بنزدیک شاه اندر آمد چو باد & ببوسید تخت و پیامش بداد \\
چو بشنید گفتار پیران بدرد & دلش گشت پرخون و رخساره زرد \\
شد از کار آن کشتگان خسته‌دل & بدان درد بنهاد پیوسته دل \\
وزان نیز کز دشمنان لشکرش & گریزان و ویران شده کشورش \\
ز هر سو پلنگ اندر آورده چنگ & بروبر جهان گشته تاریک و تنگ \\
چو گفتار پیران ازان سان شنید & سپه را همه پای برجای دید \\
به شبگیر چون تاج بر سر نهاد & همانگه فرستاده را در گشاد \\
بفرمود تا بازگردد بجای & سوی نامور بندهٔ کدخدای \\
چنین پاسخ آورد کو را بگوی & که ای مهربان نیکدل راستگوی \\
تو تا زادی از مادر پاکتن & سرافراز بودی بهر انجمن \\
ترا بیشتر نزد من دستگاه & توی برتر از پهلوانان بجاه \\
همیشه یکی جوشنی پیش من & سپر کرده جان و فدی کرده تن \\
همیدون بهر کار با گنج خویش & گزیده ز بهر منی رنج خویش \\
تو بردی ز چین تا بایران سپاه & تو کردی دل و بخت دشمن سیاه \\
نبیند سپه چون تو سالار نیز & نبندد کمر چون تو هشیار نیز \\
ز تور و پشنگ ار دراید بمهر & چو تو پهلوان نیز نارد سپهر \\
نخست آنک گفتی من از انجمن & گنهکار دارم همی خویشتن \\
که کیخسرو آمد ز توران زمین & به ایران و با ما بگسترد کین \\
بدین من که شاهم نیازرده‌ام & بدل هرگز این یاد ناورده‌ام \\
نباید که باشی بدین تنگدل & ز تیمار یابد ترا زنگ دل \\
که آن بودنی بود از کردگار & نیامد بدین بد کس آموزگار \\
که کیخسرو از من نگیرد فروغ & نبیره مخوانش که باشد دروغ \\
نباشم همیدون من او را نیا & نجویم همی زین سخن کیمیا \\
بدن کار او کس گنهکار نیست & مرا با جهاندار پیکار نیست \\
چنین بود و این بودنی کار بود & مرا از تو در دل چه آزار بود \\
و دیگر که گفتی ز کار سپاه & ز گردیدن تیره خورشید و ماه \\
همیشه چنینست کار نبرد & ز هر سو همی گردد این تیره گرد \\
گهی برکشد تا بخورشید سر & گهی اندر آرد ز خورشید بر \\
بیکسان نگردد سپهر بلند & گهی شاد دارد گهی مستمند \\
گهی با می و رود و رامشگران & گهی با غم و گرم و با اندهان \\
تو دل را بدین درد خسته مدار & روان را بدین کار بسته مدار \\
سخن گفتن کشتگان گشت خواب & ز کین برادر تو سر برمتاب \\
دلی کو ز درد برادر شخود & علاج پزشکان نداردش سود \\
سه دیگر که گفتی که خسرو پگاه & بجنگ اندر آید همی با سپاه \\
مبیناد چشم کس آن روزگار & که او پیشدستی نماید بکار \\
که من خود برانم کز ایدر سپاه & ازان سوی جیحون گذارم براه \\
نه گودرز مانم نه خسرو نه طوس & نه گاه و نه تاج و نه بوق و نه کوس \\
بایران ازان گونه رانم سپاه & کزان پس نبیند کسی تاج و گاه \\
بکیخسرو این پس نمانم جهان & بسر بر فرود آیمش ناگهان \\
بخنجر ازان سان ببرم سرش & که گرید بدو لشکر و کشورش \\
مگر کاسمانی دگرگونه کار & فرازآید از گردش روزگار \\
ترا ای جهاندیدهٔ سرافراز & نکردست یزدان بچیزی نیاز \\
ز مردان وز گنج و نیروی دست & همه ایزدی هرچ بایدت هست \\
یکی نامور لشکری ده هزار & دلیر و خردمند و گرد و سوار \\
فرستادم اینک بنزدیک تو & که روشن کند جان تاریک تو \\
از ایرانیان ده وزینها یکی & بچشم یکی ده سوار اندکی \\
چو لشکر بنزد تو آید مپای & سر و تاج گودرز بگسل ز جای \\
همان کوه کو کرده دارد حصار & باسیان جنگی ز پا اندرآر \\
مکش دست ازیشان بخون ریختن & تو پیروز باشی بویختن \\
ممان زنده زیشان بگیتی کسی & که نزد تو آید ازیشان بسی \\
فرستاده بنشیند پیغام شاه & بیامد بر پهلوان سپاه \\
بپیش اندر آمد بسان شمن & خمیده چو از بار شاخ سمن \\
بپیران رسانید پیغام شاه & وزان نامداران جنگی سپاه \\
چو بشنید پیران سپه را بخواند & فرستاده چون این سخن باز راند \\
سپه را سراسر همه داد دل & که از غم بباشید آزاد دل \\
نهانی روانش پر از درد بود & پر از خون دل و بخت برگرد بود \\
که از هر سوی لشکر شهریار & همی کاسته دید در کارزار \\
هم از شاه خسرو دلش بود تنگ & بترسید کاید یکایک بجنگ \\
بیزدان چنین گفت کای کردگار & چه مایه شگفت اندرین روزگار \\
کرا برکشیدی تو افگنده نیست & جز از تو جهاندار دارنده نیست \\
بخسرو نگر تا جز از کردگار & که دانست کید یکی شهریار \\
نگه کن بدین کار گردنده دهر & مر آن را که از خویشتن کرد بهر \\
برآرد گل تازه از خار خشک & شود خاک بابخت بیدار مشک \\
شگفتی‌تر آنک از پی آز مرد & همیشه دل خویش دارد بدرد \\
میان نیا و نبیره دو شاه & ندانم چرا باید این کینه‌گاه \\
دو شاه و دو کشور چنین جنگجوی & دو لشکر بروی اندر آورده روی \\
چه گویی سرانجام این کارزار & کرا برکشد گردش روزگار \\
پس آنگه بیزدان بنالید زار & که ای روشن دادگر کردگار \\
گر افراسیاب اندرین کینه‌گاه & ابا نامداران توران سپاه \\
بدین رزمگه کشته خواهد شدن & سربخت ما گشته خواهد شدن \\
چو کیخسرو آید ز ایران بکین & بدو بازگردد سراسر زمین \\
روا باشد ار خسته در جوشنم & برآرد روان کردگار از تنم \\
مبیناد هرگز جهانبین من & گرفته کسی راه و آیین من \\
کرا گردش روز با کام نیست & ورا زندگانی و مرگش یکیست \\
وزان پس ز ایران سپه کرنای & برآمد دم بوق و هندی درای \\
دو رویه ز لشکر برآمد خروش & زمین آمد از نعل اسبان بجوش \\
سپاه اندر آمد ز هر سو گروه & بپوشید جوشن همه دشت و کوه \\
دو سالار هر دو بسان پلنگ & فراز آوریدند لشکر بجنگ \\
بکردار باران ز ابر سیاه & ببارید تیر اندران رزمگاه \\
جهان چون شب تیره از تیره میغ & چو ابری که باران او تیر و تیغ \\
زمین آهنین کرده اسبان بنعل & برو دست گردان بخون گشته لعل \\
ز بس خسته ترک اندران رزمگاه & بریده سرانشان فگنده براهچ \\
برآورد گه جای گشتن نماند & پی اسب را برگذشتن نماند \\
زمین لاله‌گون شد هوا نیلگون & برآمد همی موج دریای خون \\
دو سالار گفتند اگر همچنین & بداریم گردان برین دشت کین \\
شب تیره را کس نماند بجای & جز از چرخ گردان و گیهان خدای \\
چو پیران چنان دید جای نبرد & بلهاک فرمود و فرشیدورد \\
که چندان کجا با شما لشکرست & کسی کاندرین رزمگه درخورست \\
سران را ببخشید تا بر سه روی & بوند اندرین رزمگه کینه‌جوی \\
وزیشان گروهی که بیدارتر & سپه را ز دشمن نگهدارتر \\
بدیشان سپارید پشت سپاه & شما بر دو رویه بگیرید راه \\
بلهاک فرمود تا سوی کوه & برد لشکر خویش را همگروه \\
همیدون سوی رود فرشیدورد & شود تا برارد بخورشید گرد \\
چو آن نامداران توران سپاه & گسستند زان لشکر کینه‌خواه \\
نوندی برافگند بر دیده‌بان & ازان دیده گه تا در پهلوان \\
نگهبان گودرز خود با سپاه & همی داشت هر سو ز دشمن نگاه \\
دو رویه چو لهاک و فرشیدورد & ز راه کیمن برگشادند گرد \\
سواران ایران برآویختند & همی خاک با خون برآمیختند \\
نوندی برافگند هر سو دوان & بگاه کردن بر پهلوان \\
نگه کرد گودرز تا پشت اوی & که دارد ز گردان پرخاشجوی \\
گرامی پسر شیر شرزه هجیر & بپشت پدر بود با تیغ و تیر \\
بفرمود تا شد بپشت سپاه & بر گیو گودرز لشکرپناه \\
بگوید که لشکر سوی رود و کوه & بیاری فرستد گروها گروه \\
ودیگر بفرمود گفتن بگیو & که پشت سپه را یکی مرد نیو \\
گزیند سپارد بدو جای خویش & نهد او از آن جایگه پای پیش \\
هجیر خردمند بسته کمر & چو بشنید گفتار فرخ پدر \\
بیامد بسوی برادر دوان & بگفت آن کجا گفته بد پهلوان \\
چز بشنید گیو این سخن بردمید & ز لشکر یکی نامور برگزید \\
کجا نام او بود فرهاد گرد & بخواند و سپه یکسر او را سپرد \\
دو صد کار دیده دلاور سران & بفرمود تا زنگه شاوران \\
برد تاختن سوی فرشیدورد & برانگیزد از رود وز آب گرد \\
ز گردان دو صد با درفشی چو باد & بفرخنده گرگین میلاد داد \\
بدو گفت ز ایدر بگردان عنان & اباگرز و با آبداده سنان \\
کنون رفت باید بران رزمگاه & جهان کرد باید بریشان سیاه \\
که پشت سپهشان بهم بر شکست & دل پهلوانان شد از درد پست \\
ببیژن چنین گفت کای شیرمرد & توی شیر درنده روز نبرد \\
کنون شیرمردی بکار آیدت & که با دشمنان کارزار آیدت \\
از ایدر برو تا بقلب سپاه & ز پیران بدان جایگه کینه‌خواه \\
ازیشان نپرهیز و تن پیش‌دار & که آمد گه کینه در کارزار \\
که پشت همه شهر توران بدوست & چو روی تو بیند بدردش پوست \\
اگر دست‌یابی برو کار بود & جهاندار و نیک اخترت یار بود \\
بیاساید از رنج و سختی سپاه & شود شادمانه جهاندار و شاه \\
شکسته شود پشت افراسیاب & پر از خون کند دل دو دیده پر آب \\
بگفت این سخن پهلوان با پسر & پسر جنگ را تنگ بسته کمر \\
سواران که بودند بر میسره & بفرمود خواندن همه یکسره \\
گرازه برون آمد و گستهم & هجیر سپهدار و بیژن بهم \\
وزآنجا سوی قلب توران سپاه & گرانمایگان برگرفتند راه \\
بکردار گرگان بروز شکار & بران بادپایان اخته زهار \\
میان سپاه اندرون تاختند & ز کینه همی دل بپرداختند \\
همه دشت بر گستوانور سوار & پراگنده گشته گه کارزار \\
چه مایه فتاده بپای ستور & کفن جوشن و سینهٔ شیر گور \\
چو رویین پیران ز پشت سپاه & بدید آن تکاپوی و گرد سیاه \\
بیامد بپشت سپاه بزرگ & ابا نامداران بکردار گرگ \\
برآویخت برسان شرزه پلنگ & بکوشید و هم بر نیامد بجنگ \\
بیفگند شمشیر هندی ز مشت & بنومیدی از جنگ بنمود پشت \\
سپهدار پیران و مردان خویش & بجنگ اندرون پای بنهاد پیش \\
چو گیو آن زمان روی پیران بدید & عنان سوی او جنگ را برگشید \\
ازان مهتران پیش پیران چهار & بنیزه ز اسب اندر افگند خوار \\
بزه کرد پیران ویسه کمان & همی تیر بارید بر بدگمان \\
سپر بر سر آورد گیو سترگ & بنیزه درآمد بکردار گرگ \\
چو آهنگ پیران سالار کرد & که جوید بورد با او نبرد \\
فروماند اسبش همیدون بجای & از آنجا که بد پیش ننهاد پای \\
یکی تازیانه بران تیز رو & بزد خشم را نامبردار گو \\
بجوشید بگشاد لب را ز بند & بنفرین دژخیم دیو نژند \\
بیفگند نیزه کمان برگرفت & یکی درقهٔ کرگ بر سر گرفت \\
کمان را بزه کرد و بگشاد بر & که با دست پیران بدوزد سپر \\
بزد بر سرش چارچوبه خدنگ & نبد کارگر تیر بر کوه سنگ \\
همیدون سه چوبه بر اسب سوار & بزد گیو پیکان آهن گذار \\
نشد اسب خسته نه پیران نیو & بدانجا رسیدند یاران گیو \\
چو پیران چنان دید برگشت زود & برفت از پسش گیو تازان چو دود \\
بنزدیک گیو آمد آنگه پسر & که ای نامبردار فرخ پدر \\
من ایدون شنیدستم از شهریار & که پیران فراوان کند کارزار \\
ز چنگ بسی تیزچنگ اژدها & مر او را بود روز سختی رها \\
سرانجام بر دست گودرز هوش & برآید تو ای باب چندین مکوش \\
پس اندر رسیدند یاران گیو & پر از خشم و کینه سواران نیو \\
چو پیران چنان دید برگشت زری & سوی لشکر خویش بنهاد روی \\
خروشان پر از درد و رخساره زرد & بنزدیک لهاک و فرشیدورد \\
بیامد که ای نامداران من & دلیران و خنجرگزاران من \\
شما را ز بهر چنین روزگار & همی پرورانیدم اندر کنار \\
کنون چون بجنگ اندر آمد سپاه & جهان شد بما بر ز دشمن سیاه \\
نبینم کسی کز پی نام و ننگ & بپیش سپاه اندر آید بجنگ \\
چو آواز پیران بدیشان رسید & دل نامداران ز کین بردمید \\
برفتند و گفتند گر جان پاک & نباشد بتن نیستمان بیم و باک \\
ببندیم دامن یک اندر دگر & نشاید گشادن برین کین کمر \\
سوی گیو لهاک و فرشیدورد & برفتند و جستند با او نبرد \\
برآمد بر گیو لهاک نیو & یکی نیزه زد بر کمرگاه گیو \\
همی خواست کو را رباید ز زین & نگونسار از اسب افگند بر زمین \\
بنیزه زره بردرید از نهیب & نیامد برون پای گیو از رکیب \\
بزد نیزه پس گیو بر اسب اوی & ز درد اندر آمد تگاور بروی \\
پیاده شد از باره لهاک مرد & فراز آمد از دور فرشید ورد \\
ابر نیزهٔ گیو تیغی چو باد & بزد نیزه ببرید و برگشت شاد \\
چو گیو اندران زخم او بنگرید & عمود گران از میان برکشید \\
بزد چون یکی تیزدم اژدها & که از دست او خنجر آمد رها \\
سبک دیگری زد بگردنش بر & که آتش ببارید بر تنش بر \\
بجوشید خون بر دهانش از جگر & تنش سست برگشت و آسیمه سر \\
چو گیو اندرین بود لهاک زود & نشست از بر بادپای چو دود \\
ابا گرز و با نیزه برسان شیر & بر گیو رفتند هر دو دلیر \\
چه مایه ز چنگ دلاور سران & برو بر ببارید گرز گران \\
بزین خدنگ اندورن بد سوار & ستوهی نیامدش از کارزار \\
چو دیدند لهاک و فرشیدورد & چنان پایداری ازان شیرمرد \\
ز بس خشم گفتند یک با دگر & که ما را چه آمد ز اختر بسر \\
برین زین همانا که کوهست و روست & برو بر ندرد جز از شیر پوست \\
ز یارانش گیو آنگهی نیزه خواست & همی گشت هر سو چپ و دست راست \\
بدیشان نهاد از دو رویه نهیب & نیامد یکی را سر اندر نشیب \\
بدل گفت کاری نو آمد بروی & مرا زین دلیران پرخاشجوی \\
نه از شهر ترکان سران آمدند & که دیوان مازندران آمدند \\
سوی راست گیو اندر آمد چو گرد & گرازه بپرخاش فرشیدورد \\
ز پولاد در چنگ سیمین ستون & بزیر اندرون باره‌ای چون هیون \\
گرازه چو بگشاد از باد دست & بزین بر شد آن ترگ پولاد بست \\
بزد نیزه‌ای بر کمربند اوی & زره بود نگسست پیوند اوی \\
یکی تیغ در چنگ بیژن چو شیر & بپشت گرازه درآمد دلیر \\
بزد بر سر و ترگ فرشیدورد & زمین را بدرید ترک از نبرد \\
همی کرد بر بارگی دست راست & باسب اندر آمد نبود آنچ خواست \\
پس بیژن اندر دمان گستهم & ابا نامداران ایران بهم \\
بنزدیک توران سپاه آمدند & خلیده‌دل و کینه‌خواه آمدند \\
ز توران سپاه اندریمان چو گرد & بیامد دمان تا بجای نبرد \\
عمودی فروهشت بر گستهم & که تا بگسلاند میانش ز هم \\
بتیغش برآمد بدو نیم گشت & دل گستهم زو پر از بیم گشت \\
بپشت یلان اندر آمد هجیر & ابر اندریمان ببارید تیر \\
خدنگش بدرید برگستوان & بماند آن زمان بارگی بی روان \\
پیاده شد ازباره مرد سوار & سپر بر سر آورد و بر ساخت کار \\
ز ترکان بر آمد سراسر غریو & سواران برفتند برسان دیو \\
مر او را بچاره ز آوردگاه & کشیدند از پیش روی سپاه \\
سپهدار پیران ز سالارگاه & بیامد بیاراست قلب سپاه \\
ز شبگیر تا شب برآمد زکوه & سواران ایران و توران گروه \\
همی گرد کینه برانگیختند & همی خاک با خون برآمیختند \\
از اسبان و مردان همه رفته هوش & دهن خشک و رفته ز تن زور و توش \\
چو روی زمین شد برنگ آبنوس & برآمد ز هر دو سپه بوق و کوس \\
ابر پشت پیلان تبیره زنان & ازان رزمگه بازگشت آن زمان \\
بران بر نهادند هر دو سپاه & که شب بازگردند ز آوردگاه \\
گزینند شبگیر مردان مرد & که از ژرف دریا برآرند گرد \\
همه نامداران پرخاشجوی & یکایک بروی اندر آرند روی \\
ز پیکار یابد رهایی سپاه & نریزند خون سر بیگناه \\
بکردند پیمان و گشتند باز & گرفتند کوتاه رزم دراز \\
دو سالار هر دو زکینه بدرد & همی روی بر گاشتند از نبرد \\
یکی سوی کوه کنابد برفت & یکی سوی زیبد خرامید تفت \\
همانگه طلایه ز لشکر براه & فرستاد گودرز سالار شاه \\
ز جوشنوران هرک فرسوده بود & زخون دست و تیغش بیالوده بود \\
همه جوشن و خود و ترگ و زره & گشادند مربندها را گره \\
چو از بار آهن برآسوده شد & خورش جست و می چند پیموده شد \\
بتدبیر کردن سوی پهلوان & برفتند بیدار پیر و جوان \\
بگودرز پس گفت گیو ای پدر & چه آمد مرا از شگفتی بسر \\
چو من حمله بردم بتوران سپاه & دریدم صف و برگشادند راه \\
بپیران رسیدم نوندم بجای & فروماند و ننهاد از پیش پای \\
چنانم شتاب آمد از کار خویش & که گفتم نباشم دگر یار خویش \\
پس آن گفته شاه بیژن بیاد & همی داشت وان دم مرا یادداد \\
که پیران بدست تو گردد تباه & از اختر همین بود گفتار شاه \\
بدو گفت گودرز کو را زمان & بدست منست ای پسر بی‌گمان \\
که زو کین هفتاد پور گزین & بخواهم بزور جهان‌آفرین \\
ازان پس بروی سپه بنگرید & سران را همه گونه پژمرده دید \\
ز رنج نبرد و ز خون ریختن & بهرجای با دشمن آویختن \\
دل پهلوان گشت زان پر ز درد & که رخسار آزادگان دید زرد \\
بفرمودشان بازگشتن بجای & سپهدار نیک‌اختر و رهنمای \\
بدان تا تن رنج بردارشان & برآساید از جنگ و پیکارشان \\
برفتند و شبگیر بازآمدند & پر از کینه و زرمساز آمدند \\
بسالار برخواندند آفرین & که ای نامور پهلوان زمین \\
شبت خواب چون بود و چون خاستی & ز پیکار ترکان چه آراستی \\
بدیشان چنین گفت پس پهلوان & که ای نیک‌مردان و فرخ گوان \\
سزد گر شما بر جهان‌آفرین & بخوانید روز و شبان آفرین \\
که تا این زمان هرچ رفت از نبرد & به کام دل ما همی گشت گرد \\
فراوان شگفتی رسیدم بسر & جهان را ندیدم مگر بر گذر \\
ز بیداد و داد آنچ آمد بشاه & بد و نیک راهم بدویست راه \\
چو ما چرخ گردان فراوان سرشت & درود آن کجا برزو خود بکشت \\
نخستین که ضحاک بیدادگر & ز گیتی بشاهی برآورد سر \\
جهان را چه مایه بسختی بداشت & جهان آفرین زو همه درگذاشت \\
بداد آنک آورد پیدا ستم & ز باد آمد آن پادشاهی بدم \\
چو بیداد او دادگر برنداشت & یکی دادگر را برو برگماشت \\
برآمد بران کار او چند سال & بد انداخت یزدان بران بدسگال \\
فریدون فرخ شه دادگر & ببست اندر آن پادشاهی کمر \\
همه بند آهرمنی برگشاد & بیاراست گیتی سراسر بداد \\
چو ضحاک بدگوهر بدمنش & که کردند شاهان بدو سرزنش \\
ز افراسیاب آمد آن بد خوی & همان غارت و کشتن و بدگوی \\
که در شهر ایران بگسترد کین & بگشت از ره داد و آیین و دین \\
سیاوش را هم به فرجام کار & بکشت و برآورد از ایران دمار \\
وزانپس کجا گیو ز ایران براند & چه مایه بسختی بتوران بماند \\
نهالیش بد خاک و بالینش سنگ & خورش گوشت نخچیر و پوشش پلنگ \\
همی رفت گم بوده چون بیهشان & که یابد ز کیخسرو آنجا نشان \\
یکایک چو نزدیک خسرو رسید & برو آفرین کرد کو را بدید \\
وزانپس به ایران نهادند روی & خبر شد بپیران پرخاشجوی \\
سبک با سپاه اندر آمد براه & که هر دو کندشان بره برتباه \\
بکرد آنچ بودش ز بد دسترس & جهاندارشان بد نگهدار و بس \\
ازان پس بکین سیاوش سپاه & سوی کاسه رود اندر آمد براه \\
بلاون که آمد سپاه گشتن & شبیخون پیران و جنگ پشن \\
که چندان پسر پیش من کشته شد & دل نامداران همه گشته شد \\
کنون با سپاهی چنین کینه‌جوی & بیامد بروی اندر آورد روی \\
چو با ما بسنده نخواهد بدن & همی داستانها بخواهد زدن \\
همی چاره سازد بدان تا سپاه & ز توران بیاید بدین رزمگاه \\
سران را همی خواهد اکنون بجنگ & یکایک بباید شدن تیز چنگ \\
که گر ما بدین کار سستی کنیم & وگر نه بدین پیشدستی کنیم \\
بهانه کند بازگردد ز جنگ & بپیچد سر از کینه و نام و ننگ \\
ار ایدونک باشید با من یکی & ازیشان فراوان و ما اندکی \\
ازان نامداران برآریم گرد & بدانگه که سازد همی او نبرد \\
ور ایدونک پیران ازین رای خویش & نگردد نهد رزم را پای پیش \\
پذیرفتم اندر شما سربسر & که من پیش بندم بدین کین کمر \\
ابا پیر سر من بدین رزمگاه & بکشتن دهم تن بپیش سپاه \\
من و گرد پیران و رویین و گیو & یکایک بسازیم مردان نیو \\
که کس در جهان جاودانه نماند & بگیتی بما جز فسانه نماند \\
هم آن نام باید که ماند بلند & چو مرگ افگند سوی ما برکمند \\
زمانه بمرگ و بکشتن یکیست & وفا با سپهر روان اندکیست \\
شما نیز باید که هم زین نشان & ابا نیزه و تیغ مردم کشان \\
بکینه ببندید یکسر کمر & هرانکس که هست از شما نامور \\
که دولت گرفتست از ایشان نشیب & کنون کرد باید بکین بر نهیب \\
بتوران چو هومان سواری نبود & که با بیژن گیو رزم آزمود \\
چو برگشته بخت او شد نگون & بریدش سر از تن بسان هیون \\
نباید شکوهید زیشان بجنگ & نشاید کشیدن ز پیکار چنگ \\
ور ایدونک پیران بخواهد نبرد & باندوه لشکر بیارد چو گرد \\
همیدون بانبوه ما همچو کوه & بباید شدن پیش او همگروه \\
که چندان دلیران همه خسته‌دل & ز تیمار و اندوه پیوسته دل \\
برانم که ما را بود دستگاه & ازیشان برآریم گرد سیاه \\
بگفت این سخن سربسر پهلوان & بپیش جهاندیده فرخ گوان \\
چو سالارشان مهربانی نمود & همه پاک بر پای جستند زود \\
برو سربسر خواندند آفرین & که چون تو کسی نیست پر داد و دین \\
پرستنده چون تو فریدون نداشت & که گیتی سراسر بشاهی گذاشت \\
ستون سپاهی و سالار شاه & فرازندهٔ تاج و گاه و کلاه \\
فدی کردهٔ جان و فرزند و چیز & ز سالار شاهان چه جویند نیز \\
همه هرچ شاه از فریبرز جست & ز طوس آن کنون از تو بیند درست \\
همه سربسر مر ترا بنده‌ایم & بفرمان و رایت سرافگنده‌ایم \\
گر ایدونک پیران ز توران سپاه & سران آورد پیش ما کینه‌خواه \\
ز ما ده مبارز و زیشان هزار & نگر تا که پیچد سر از کارزار \\
ور ایدونک لشکر همه همگروه & بجنگ اندر آید بکردار کوه \\
ز کینه همه پاک دلخسته‌ایم & کمر بر میان جنگ را بسته‌ایم \\
فدای تو بادا تن و جان ما & سراسر برینست پیمان ما \\
چو گودرز پاسخ برین سان شنود & بدلش اندرون شادمانی فزود \\
بران نامداران گرفت آفرین & که این نره شیران ایران زمین \\
سپه را بفرمود تا برنشست & همیدون میان را بکینه ببست \\
چپ لشکرش جای رهام گرد & بفرهاد خورشید پیکر سپرد \\
سوی راست جای فریبرز بود & بکتمارهٔ قارنان داد زود \\
بشیدوش فرمود کای پور من & بهر کار شایسته دستور من \\
تو با کاویانی درفش و سپاه & برو پشت لشکر تو باش و پناه \\
بفرمود پس گستهم را که شو & سپه را تو باش این زمان پیشرو \\
ترا بود باید بسالارگاه & نگه‌دار بیدار پشت سپاه \\
سپه را بفرمود کز جای خویش & نگر ناورید اندکی پای پیش \\
همه گستهم را کنید آفرین & شب و روز باشید بر پشت زین \\
برآمد خروش از میان سپاه & گرفتند زاری بران رزمگاه \\
همه سربسر سوی او تاختند & همی خاک بر سر برانداختند \\
که با پیر سر پهلوان سپاه & کمر بست و شد سوی آوردگاه \\
سپهدار پس گستهم را بخواند & بسی پند و اندرز با او براند \\
بدو گفت زنهار بیدار باش & سپه را ز دشمن نگهدار باش \\
شب و روز در جوشن کینه‌جوی & نگر تا گشاده ندارید روی \\
چو آغازی از جنگ پرداختن & بود خواب را بر تو برتاختن \\
همان چون سرآری بسوی نشیب & ز ناخفتگان بر تو آید نهیب \\
یکی دیده‌بان بر سر کوه دار & سپه را ز دشمن بی‌اندوه‌دار \\
ور ایدونک آید ز توران زمین & شبی ناگهان تاختن گر کمین \\
تو باید که پیکار مردان کنی & بجنگ اندر آهنگ گردان کنی \\
ور ایدونک از ما بدین رزمگاه & بدآگاهی آید ز توران سپاه \\
که ما را به آوردگه برکشند & تن بی‌سران مان بتوران کشند \\
نگر تا سپه را نیاری بجنگ & سه روز اندرین کرد باید درنگ \\
چهارم خود آید بپشت سپاه & شه نامبردار با پیل و گاه \\
چو گفتار گودرز زان سان شنید & سرشکش ز مژگان برخ برچکید \\
پذیرفت سر تا بسر پند اوی & همی جست ازان کار پیوند اوی \\
بسالار گفت آنچ فرمان دهی & میان بسته دارم بسان رهی \\
پس از جنگ پیشین که آمد شکست & که توران بران درد بودند پست \\
خروشان پدر بر پسر روی زد & برادر ز خون برادر بدرد \\
همه سر بسر سوگوار و نژند & دژم گشته از گشت چرخ بلند \\
چو پیران چنان دید لشکر همه & چو از گرگ درنده خسته رمه \\
سران را ز لشکر سراسر بخواند & فراوان سخن پیش ایشان براند \\
چنین گفت کای کار دیده گوان & همه سودهٔ رزم پیر و جوان \\
شما را بنزدیک افراسیاب & چه مایه بزرگی و جاهست و آب \\
بپیروزی و فرهی کامتان & بگیتی پراگنده شد نامتان \\
بیک رزم کآمد شما را شکست & کشیدید یکسر ز پیکار دست \\
بدانید یکسر کزین رزمگاه & اگر بازگردد بسستی سپاه \\
پس اندر ز ایران دلاور سران & بیایند با گرزهای گران \\
یکی را ز ما زنده اندر جهان & نبیند کس از مهتران و کهان \\
برون کرد باید ز دلها نهیب & گزیدن مرین غمگنان را شکیب \\
چنین داستان زد شه موبدان & که پیروز یزدان بود جاودان \\
جهان سربسر با فراز و نشیب & چنینست تا رفتن اندر نهیب \\
کنون از بر و بوم و فرزند خویش & که اندیشد از جان و پیوند خویش \\
همان لشکر است این که از جنگ ما & بپیچید و بس کرد آهنگ ما \\
بدین رزمگه بست باید میان & بکینه شدن پیش ایرانیان \\
چنین کرد گودرز پیمان که من & سران برگزینم ازین انجمن \\
یکایک بروی اندر آریم روی & دو لشکر برآساید از گفت و گوی \\
گر ایدونک پیمان بجای آورید & سران را ز لشکر بپای آورید \\
وگر همگروه اندر آید بجنگ & نباید کشیدن ز پیکار چنگ \\
اگر سر همه سوی خنجر بریم & بروزی بزادیم و روزی مریم \\
وگرنه سرانشان برآرم بدار & دو رویه بود گردش روزگار \\
اگر سر بپیچد کس از گفت من & بفرمایمش سر بریدن ز تن \\
گرفتند گردان بپاسخ شتاب & که ای پهلوان رد افراسیاب \\
تو از دیرگه باز با گنج خویش & گزیدستی از بهر ما رنج خویش \\
میان بسته بر پیش ما چون رهی & پسر با برادر بکشتن دهی \\
چرا سر بپیچیم ما خود کیییم & چنین بندهٔ شه ز بهر چییم \\
بگفتند وز پیش برخاستند & بپیکار یکسر بیاراستند \\
همه شب همی ساختند این سخن & که افگند سالار بیدار بن \\
بشبگیر آوای شیپور و نای & ز پرده برآمد بهر دو سرای \\
نشستند بر زین سپیده دمان & همه نامداران بباز و کمان \\
که از نعل اسبان تو گفتی زمین & بپوشد همی چادر آهنین \\
سپهبد بلهاک و فرشیدورد & چنین گفت کای نامداران مرد \\
شما را نگهبان توران سپاه & همی بود باید بدین رزمگاه \\
یکی دیده‌بان بر سر کوهسار & نگهبان روز و ستاره‌شمار \\
گر ایدونک ما را ز گردان سپهر & بد آید ببرد ز ما پاک مهر \\
شما جنگ را کس متازید زود & بتوران شتابید برسان دود \\
کزین تخمهٔ ویسگان کس نماند & همه کشته شد جز شما بس نماند \\
گرفتند مر یکدگر را کنار & بدرد جگر برگسستند زار \\
برفتند و بس روی برگاشتند & غریویدن و بانگ برداشتند \\
پر از کینه سالار توران سپاه & خروشان بیامد به آوردگاه \\
چو گودرز کشوادگان را بدید & سخن گفت بسیار و پاسخ شنید \\
بدو گفت کای پر خرد پهلوان & برنج اندرون چند پیچی روان \\
روان سیاوش را زان چه سود & که از شهر توران برآری تو دود \\
بدان گیتی او جای نیکان گزید & نگیری تو آرام کو آرمید \\
دو لشکر چنین پاک با یکدگر & فگنده چو پیلان ز تن دور سر \\
سپاه دو کشور همه شد تباه & گه آمد که برداری این کینه‌گاه \\
جهان سربسر پاک بی‌مرد گشت & برین کینه پیکار ما سرد گشت \\
ور ایدونک هستی چنین کینه‌دار & ازان کوهپایه سپاه اندرآر \\
تو از لشکر خویش بیرون خرام & مگر خود برآیدت زین کینه کام \\
بتنها من و تو برین دشت کین & بگردیم و کین‌آوران همچنین \\
ز ما هرک او هست پیروزبخت & رسد خود بکام و نشیند بتخت \\
اگر من بدست تو گردم تباه & نجویند کینه ز توران سپاه \\
بپیش تو آیند و فرمان کنند & بپیمان روان را گروگان کنند \\
وگر تو شوی کشته بر دست من & کسی را نیازارم از انجمن \\
مرا با سپاه تو پیکار نیست & بریشان ز من نیز تیمار نیست \\
چو گودرز گفتار پیران شنید & از اختر همی بخت وارونه دید \\
نخست آفرین کرد بر کردگار & دگر یاد کرد از شه نامدار \\
بپیران چنین گفت کای نامور & شنیدیم گفتار تو سربسر \\
ز خون سیاوش بافراسیاب & چه سودست از داد سر برمتاب \\
که چون گوسفندش ببرید سر & پر از خون دل از درد خسته جگر \\
ازان پس برآورد ز ایران خروش & زبس کشتن و غارت و جنگ و جوش \\
سیاوش بسوگند تو سربداد & تو دادی بخیره مر او را بباد \\
ازان پس که نزد تو فرزند من & بیامد کشیدی سر از پند من \\
شتابیدی و جنگ را ساختی & بکردار آتش همی تاختی \\
مرا حاجت از کردگار جهان & برین گونه بود آشکار و نهان \\
که روزی تو پیش من آیی بجنگ & کنون آمدی نیست جای درنگ \\
به پیران سر اکنون به آوردگاه & بگردیم یک با دگر بی‌سپاه \\
سپهدار ترکان برآراست کار & ز لشکر گزید آن زمان ده سوار \\
ابا اسب و ساز و سلیح تمام & همه شیرمرد و همه نیک‌نام \\
همانگه ز ایران سپه پهلوان & بخواند آن زمان ده سوار جوان \\
برون تاختند از میان سپاه & برفتند یکسر به آوردگاه \\
که دیدار دیده بریشان نبود & دو سالار زین گونه زرم آزمود \\
ابا هر سواری ز ایران سپاه & ز توران یکی شد ورا رزم خواه \\
نهادند پس گیو را با گروی & که همزور بودند و پرخاشجوی \\
گروی زره کز میان سپاه & سراسر برو بود نفرین شاه \\
که بگرفت ریش سیاوش بدست & سرش را برید از تن پاک پست \\
دگر با فریبرز کاوس تفت & چو کلباد ویسه بورد رفت \\
چو رهام گودرز با بارمان & برفتند یک با دگر بدگمان \\
گرازه بشد با سیامک بجنگ & چو شیر ژیان با دمنده نهنگ \\
چو گرگین کارآزموده سوار & که با اندریمان کند کارزار \\
ابا بیژن گیو رویین گرد & بجنگ از جهان روشنایی ببرد \\
چو او خواست با زنگه شاوران & دگر برته با کهرم از یاوران \\
چو دیگر فروهل بد و زنگله & برون تاختند از میان گله \\
هجیر و سپهرم بکردار شیر & بدان رزمگاه اندر آمد دلیر \\
چو گودرز کشواد و پیران بهم & همه ساخته دل بدرد و ستم \\
میان بسته هر دو سپهبد بکین & چه از پادشاهی چه از بهر دین \\
بخوردند سوگند یک بادگر & که کس برنگرداند از کینه سر \\
بدان تا کرا گردد امروز کار & که پیروز برگردد از کارزار \\
دو بالا بداندر دو روی سپاه & که شایست کردن بهرسو نگاه \\
یکی سوی ایران دگر سوی تور & که دیدار بودی بلشکر ز دور \\
بپیش اندرون بود هامون و دشت & که تا زنده شایست بر وی گذشت \\
سپهدار گودرز کرد آن نشان & که هر کو ز گردان گردنکشان \\
بزیر آورد دشمنی را چو دود & درفشی ز بالا برآرند زود \\
سپهدار پیران نشانی نهاد & ببالای دیگر همین کرد یاد \\
ازآن پس بهامون نهادند سر & بخون ریختن بسته گردان کمر \\
بتیغ و بگرز و بتیر و کمر & همی آزمودند هرگونه بند \\
دلیران توران و کنداوران & ابا گرز و تیغ و پرنداوران \\
که گر کوه پیش آمدی روز جنگ & نبودی بر آن رزم کردن درنگ \\
همه دستهاشان فروماند پست & در زور یزدان بریشان ببست \\
بدان بلا اندر آویختند & که بسیار بیداد خون ریختند \\
فرومانده اسبان جنگی بجای & تو گفتی که با دست بستست پای \\
بریشان همه راستی شد نگون & که برگشت روز و بجوشید خون \\
چنان خواست یزدان جان‌آفرین & که گفتی گرفت آن گوان را زمین \\
ز مردی که بودند با بخت خویش & برآویختند از پی تخت خویش \\
سران از پی پادشاهی بجنگ & بدادند جان از پی نام و ننگ \\
دمان آمدند اندر آوردگاه & ابا یکدگر ساخته کینه خواه \\
نخستین فریبرز نیو دلیر & ز لشکر برون تاخت برسان شیر \\
بنزدیک کلباد ویسه دمان & بیامد بزه برنهاده کمان \\
همی گشت و تیرش نیامد چو خواست & کشید آن پرنداور از دست راست \\
برآورد و زد تیر بر گردنش & بدو نیم شد تا کمرگه تنش \\
فرود آمد از اسب و بگشاد بند & ز فتراک خویش آن کیانی کمند \\
ببست از بر باره کلباد را & گشاد از برش بند پولاد را \\
ببالا برآمد به پیروز نام & خروشی برآورد و بگذارد گام \\
که سالار ما باد پیروزگر & همه دشمن شاه خسته‌جگر \\
و دیگر گروی زره دیو نیو & برون رفت با پور گودرز گیو \\
بنیزه فراوان برآویختند & همی زهر با خون برآمیختند \\
سناندار نیزه ز چنگ سوار & فرو ریخت از هول آن کارزار \\
کمان برگرفتند و تیر خدنگ & یک اندر دگر تاخته چون پلنگ \\
همی زنده بایست مر گیو را & کز اسب اندر آرد گو نیو را \\
چنان بسته در پیش خسرو برد & ز ترکان یکی هدیهٔ نو برد \\
چو گیو اندر آمد گروی از نهیب & کمان شد ز دستش بسوی نشیب \\
سوی تیغ برد آن زمان دست خویش & دمان گیو نیو اندر آمد بپیش \\
عمودی بزد بر سر و ترگ اوی & که خون اندر آمد ز تارک بروی \\
همیدون ز زین دست بگذاردش & گرفتش ببر سخت و بفشاردش \\
که بر پشت زین مرد بی‌توش گشت & ز اسب اندر افتاد و بیهوش گشت \\
فرود آمد از باره جنگی پلنگ & دو دست از پس پشت بستش چو سنگ \\
نشست از بر زین و او را بپیش & دوانید و شد تا بر یار خویش \\
ببالا برآمد درفشی بدست & بنعره همی کوه را کرد پست \\
به پیروزی شاه ایران زمین & همی خواند بر پهلوان آفرین \\
سه دیگر سیامک ز توران سپاه & بشد با گرازه به آوردگاه \\
برفتند و نیزه گرفته بدست & خروشان بکردار پیلان مست \\
پر از جنگ و پر خشم کینه‌وران & گرفتند زان پس عمود گران \\
چو شیران جنگی برآشوفتند & همی بر سر یکدگر کوفتند \\
زبانشان شد از تشنگی لخت لخت & بتنگی فراز آمد آن کار سخت \\
پیاده شدند و برآویختند & همی گرد کینه برانگیختند \\
گرازه بزد دست برسان شیر & مر او را چو باد اندر آورد زیر \\
چنان سخت زد بر زمین کاستخوانش & شکست و برآمد ز تن نیز جانش \\
گرازه هم آنگه ببستش باسب & نشست از بر زین چو آذرگشسب \\
گرفت آنگه اسب سیامک بدست & ببالا برآمد بکردار مست \\
درفش خجسته بدست اندرون & گرازان و شادان و دشمن نگون \\
خروشان و جوشان و نعره زنان & ابر پهلوان آفرین برکنان \\
چهارم فروهل بد و زنگله & دو جنگی بکردار شیر یله \\
بایران نبرده بتیر و کمان & نبد چون فروهل دگر بدگمان \\
چو از دور ترک دژم را بدید & کمان را بزه کرد و اندر کشید \\
برآورد زان تیرهای خدنگ & گرفته کمان رفت پیشش بجنگ \\
ابر زنگله تیرباران گرفت & ز هر سو کمین سواران گرفت \\
خدنگی برانش برآمد چو باد & که بگذشت بر مرد و بر اسب شاد \\
بروی اندر آمد تگاور ز درد & جدا شد ازو زنگله روی زرد \\
نگون شد سر زنگله جان بداد & تو گفتی همانا ز مادر نزاد \\
فروهل فروجست و ببرید سر & برون کرد خفتان رومی ز بر \\
سرش را بفتراک زین برببست & بیامد گرفت اسب او را بدست \\
ببالا برآمد بسان پلنگ & بخون غرقه گشته بر و تیغ و چنگ \\
درفش خجسته برآورد راست & شده شادمان یافته هرچ خواست \\
خروشید زان پس که پیروز باد & سر خسروان شاه فرخ نژاد \\
به پنجم چو رهام گودرز بود & که با بارمان او نبرد آزمود \\
کمان برگرفتند و تیر خدنگ & برآمد خروش سواران جنگ \\
کمانها همه پاک بر هم شکست & سوی نیزه بردند چون باد دست \\
دو جنگی و هر دو دلیر و سوار & هشیوار و دیده بسی کارزار \\
بگشتند بسیار یک بادگر & بپیچید رهام پرخاشخر \\
یکی نیزه انداخت بر ران اوی & کز اسب اندر آمد بفرمان اوی \\
جدا شد ز باره هم آنگاه ترک & ز اسب اندر افتاد ترک سترگ \\
بپشت اندرش نیزه‌ای زد دگر & سنان اندر آمد میان جگر \\
فرود آمد از باره کرد آفرین & ز دادار بر بخت شاه زمین \\
بکین سیاوش کشیدش نگون & ز کینه بمالید بر روی خون \\
بزین اندر آهخت و بستش چو سنگ & سر آویخته پایها زیر تنگ \\
نشست از بر زین و اسبش کشان & بیامد دوان تا بجای نشان \\
ببالا برآمد شده شاد دل & ز درد و غمان گشته آزاددل \\
به پیروزی شاه و تخت بلند & بکام آمده زیر بخت بلند \\
همی آفرین خواند سالار شاه & ابر شاه کیخسرو و تاج و گاه \\
که پیروزگر شاه پیروز باد & همه روزگارانش نوروز باد \\
ششم بیژن گیو و رویین دمان & بزه برنهادند هر دو کمان \\
چپ و راست گشتند یک با دگر & نبد تیرشان از کمان کارگر \\
برومی عمود آنگهی پور گیو & همی گشت با گرد رویین نیو \\
بر آوردگه بر برو دست یافت & زمین را بدرید و اندر شتافت \\
زد از باد بر سرش رومی ستون & فروریخت از ترگ او مغز و خون \\
به زین پلنگ اندرون جان بداد & ز پیران ویسه بسی کرد یاد \\
پس از پشت باره درآمد نگون & همه تن پر آهن دهن پر ز خون \\
ز اسب اندر آمد سبک بیژنا & مر او را بکردار آهرمنا \\
کمند اندر افگند و بر زین کشید & نبد کس که تیمار رویین کشید \\
برفت از پی سود مایه بباد & هنوز از جوانیش نابوده شاد \\
بر اسبش بکردار پیلی ببست & گرفت آنگهی پالهنگش بدست \\
عنان هیون تگاور بتافت & وز آن جایگه سوی بالا شتافت \\
بچنگ اندرون شیر پیکر درفش & میان دیبه و رنگ خورده بنفش \\
چنینست کار جهان فریب & پس هر فرازی نهاده نشیب \\
وز آن جایگه شد بجای نشان & بنزدیک آن نامور سرکشان \\
همی گفت پیروزگر باد شاه & همیشه سر پهلوان با کلاه \\
جهان پیش شاه جهان بنده باد & همیشه دل پهلوان باد شاد \\
برون تاخت هفتم ز گردان هجیر & یکی نامداری سواری هژیر \\
سپهرم ز خویشان افراسیاب & یکی نامور بود با جاه و آب \\
ابا پور گودرز رزم آزمود & که چون او بلشکر سواری نبود \\
برفتند هر دو بجای نبرد & برآمد ز آوردگه تیره گرد \\
بشمشیر هر دو برآویختند & همی زآهن آتش فروریختند \\
هجیر دلاور بکردار شیر & بروی سپهرم درآمد دلیر \\
بنام جهان‌آفرین کردگار & ببخت جهاندار با شهریار \\
یکی تیغ زد بر سر و ترگ اوی & که آمد هم اندر زمان مرگ اوی \\
درافتاد ز اسبش هم آنگه نگون & بزاری و خواری دهن پر ز خون \\
فرود آمد از باره فرخ هجیر & مر او را ببست از بر زین چو شیر \\
نشست از بر اسب و آن اسب اوی & گرفته عنان و درآورده روی \\
برآمد ببالا و کرد آفرین & بران اختر نیک و فرخ زمین \\
همی زور و بخت از جهاندار دید & وز آن گردش بخت بیدار دید \\
بهشتم ز گردان ناماوران & بشد ساخته زنگهٔ شاوران \\
که همرزمش از تخم او خواست بود & که از جنگ هرگز نه برکاست بود \\
گرفتند هر دو عمود گران & چو او خواست با زنگهٔ شاوران \\
بگشتند ز اندازه بیرون بجنگ & ز بس کوفتن گشت پیکار تنگ \\
فروماند اسبان جنگی ز تگ & که گفتی بتنشان نجنبید رگ \\
چو خورشید تابان ز گنبد بگشت & بکردار آهن بتفسید دشت \\
چنان تشنه گشتند کز جای خویش & نجنبید و ننهاد کس پای پیش \\
زبان برگشادند یک‌بادگر & که اکنون ز گرمی بسوزد جگر \\
بباید برآسود و دم برزدن & پس آنگه سوی جنگ بازآمدن \\
برفتند و اسبان جنگی بجای & فراز آوریدند و بستند پای \\
بسودگی باز برخاستند & بپیکار کینه بیاراستند \\
بکردار آتش ز نیزه سوار & همی گشت بر مرکز کارزار \\
بدآنگه که زنگه برو دست یافت & سنان سوی او کرد و اندر شتافت \\
یکی نیزه زد بر کمرگاه اوی & کز اسبش نگون کرد و برزد بروی \\
چو رعد خروشان یکی ویله کرد & که گفتی بدرید دشت نبرد \\
فرود آمد از باره شد نزد اوی & بران خاک تفته کشیدش بروی \\
مر او را بچاره ز روی زمین & نگون اندر افگند بر پشت زین \\
نشست از بر اسب و بالا گرفت & بترکان چه آمد ز بخت ای شگفت \\
بران کوه فرخ برآمد ز پست & یکی گرگ پیکر درفشی بدست \\
بشد پیش یاران و کرد آفرین & ابر شاه و بر پهلوان زمین \\
برون رفت گرگین نهم کینه‌خواه & ابا اندریمان ز توران سپاه \\
جهاندیده و کارکرده دو مرد & برفتند و جستند جای نبرد \\
بنیزه بگشتند و بشکست پست & کمان برگرفتند هر دو بدست \\
ببارید تیر از کمان سران & بروی اندر آورده کرگ اسپران \\
همی تیر بارید همچون تگرگ & بران اسپر کرگ و بر ترک و ترگ \\
یکی تیر گرگین بزد بر سرش & که بردوخت با ترگ رومی برش \\
بلرزید بر زین ز سختی سوار & یکی تیر دیگر بزد نامدار \\
هم آنگاه ترک اندر آمد نگون & ز چشمش برون آمد از درد خون \\
فرود آمد از باره گرگین چو گرد & سر اندریمان ز تن دور کرد \\
بفتراک بربست و خود برنشست & نوند سوار نبرده بدست \\
بران تند بالا برآمد دمان & همیدون ببازو بزه بر کمان \\
بنیروی یزدان که او بد پناه & بپیروز بخت جهاندار شاه \\
چو پیروز برگشت مرد از نبرد & درفش دلفروز بر پای کرد \\
دهم برته با کهرم تیغ‌زن & دو خونی و هر دو سر انجمن \\
همی آزمودند هرگونه جنگ & گرفتند پس تیغ هندی بچنگ \\
درفش همایون بدست اندرون & تو گفتی بجنبد که بیستون \\
یکایک بپیچید ازو برته روی & یکی تیغ زد بر سر و ترگ اوی \\
که تا سینه کهرم بد و نیک گشت & ز دشمن دل برته بی‌بیم گشت \\
فرود آمد از اسب و او را ببست & بران زین توزی و خود برنشست \\
برآمد ببالا چو شرزه پلنگ & خروشان یکی تیغ هندی بچنگ \\
درفش همایون بدست اندرون & فگنده بران باره کهرم نگون \\
همی گفت شاهست پیروزگر & همیشه کلاهش بخورشید بر \\
چو از روز نه ساعت اندر گذشت & ز ترکان نبد کس بران پهن‌دشت \\
کسی را کجا پروراند بناز & برآید برو روزگار دراز \\
شبیخون کند گاه شادی بروی & همی خواری و سختی آرد بروی \\
ز باد اندر آرد دهدمان بدم & همی داد خوانیم و پیدا ستم \\
بتورانیان بر بد آن جنگ شوم & به آوردگه کردن آهنگ شوم \\
چنان شد که پیران ز توران سپاه & سواری ندید اندر آوردگاه \\
روان‌ها گسسته ز تنشان بتیغ & جهان را تو گفتی نیامد دریغ \\
سپهدار ایران و توران دژم & فراز آمدند اندران کین بهم \\
همی برنوشتند هر دو زمین & همه دل پر از درد و سر پر ز کین \\
به آوردگاه سواران ز گرد & فروماند خورشید روز نبرد \\
بتیغ و بخنجر بگرز و کمند & ز هر گونهٔ برنهادند بند \\
فراز آمد آن گردش ایزدی & از ایران بتوران رسید آن بدی \\
ابا خواست یزدانش چاره نماند & کرا کوشش و زور و یاره نماند \\
نگه کرد پیران که هنگام چیست & بدانست کان گردش ایزدیست \\
ولیکن بمردی همی کرد کار & بکوشید با گردش روزگار \\
ازان پس کمان برگرفتند و تیر & دو سالار لشکر دو هشیار پیر \\
یکی تیرباران گرفتند سخت & چو باد خزان بر جهد بر درخت \\
نگه کرد گودرز تیر خدنگ & که آهن ندارد مر او را نه سنگ \\
ببر گستوان برزد و بردرید & تگاور بلرزید و دم درکشید \\
بیفتاد و پیران درآمد بزیر & بغلتید زیرش سوار دلیر \\
بدانست کآمد زمانه فراز & وزان روز تیره نیابد جواز \\
ز نیرو بدو نیم شد دست راست & هم آنگه بغلتید و بر پای خاست \\
ز گودرز بگریخت و شد سوی کوه & غمی شد ز درد دویدن ستوه \\
همی شد بران کوهسر بر دوان & کزو بازگردد مگر پهلوان \\
نگه کرد گودرز و بگریست زار & بترسید از گردش روزگار \\
بدانست کش نیست با کس وفا & میان بسته دارد ز بهر جفا \\
فغان کرد کای نامور پهلوان & چه بودت که ایدون پیاده دوان \\
بکردار نخچیر در پیش من & کجات آن سپاه ای سر انجمن \\
نیامد ز لشکر ترا یار کس & وزیشان نبینمت فریادرس \\
کجات آنهمه زور و مردانگی & سلیح و دل و گنج و فرزانگی \\
ستون گوان پشت افراسیاب & کنون شاه را تیره گشت آفتاب \\
زمانه ز تو زود برگاشت روی & بهنگام کینه تو چاره مجوی \\
چو کارت چنین گشت زنهار خواه & بدان تات زنده برم نزد شاه \\
ببخشاید از دل همی بر تو بر & که هستس جهان پهلوان سربسر \\
بدو گفت پیران که این خود مباد & بفرجام بر من چنین بد مباد \\
ازین پس مرا زندگانی بود & بزنهار رفتن گمانی بود \\
خود اندر جهان مرگ را زاده‌ایم & بدین کار گردن ترا داده‌ایم \\
شنیدستم این داستان از مهان & که هرچند باشی بخرم جهان \\
سرانجام مرگست زو چاره نیست & بمن بر بدین جای پیغاره نیست \\
همی گشت گودرز بر گرد کوه & نبودش بدو راه و آمد ستوه \\
پیاده ببود و سپر برگرفت & چو نخچیربانان که اندر گرفت \\
گرفته سپر پیش و ژوپین بدست & ببالا نهاده سر از جای پست \\
همی دید پیران مر او را ز دور & بست از بر سنگ سالار تور \\
بینداخت خنجر بکردار تیر & بیامد ببازوی سالار پیر \\
چو گودرز شد خسته بر دست اوی & ز کینه بخشم اندر آورد روی \\
بینداخت ژوپین بپیران رسید & زره بر تنش سربسر بردرید \\
ز پشت اندر آمد براه جگر & بغرید و آسیمه برگشت سر \\
برآمدش خون جگر بر دهان & روانش برآمد هم اندر زمان \\
چو شیر ژیان اندر آمد بسر & بنالید با داور دادگر \\
بران کوه خارا زمانی طپید & پس از کین و آوردگاه آرمید \\
زمانه بزهراب دادست چنگ & بدرد دل شیر و چرم پلنگ \\
چنینست خود گردش روزگار & نگیرد همی پند آموزگار \\
چو گودرز بر شد بران کوهسار & بدیدش بر آن‌گونه افگنده خوار \\
دریده دل و دست و بر خاک سر & شکسته سلیح و گسسته کمر \\
چنین گفت گودرز کای نره شیر & سر پهلوانان و گرد دلیر \\
جهان چون من و چون تو بسیار دید & نخواهد همی با کسی آرمید \\
چو گودرز دیدش چنان مرده‌خوار & بخاک و بخون بر طپیده بزار \\
فروبرد چنگال و خون برگرفت & بخورد و بیالود روی ای شگفت \\
ز خون سیاوش خروشید زار & نیایش همی کرد بر کردگار \\
ز هفتاد خون گرامی پسر & بنالید با داور دادگر \\
سرش را همی خواست از تن برید & چنان بدکنش خویشتن را ندید \\
درفی ببالینش بر پای کرد & سرش را بدان سایه برجای کرد \\
سوی لشکر خویش بنهاد روی & چکان خون ز بازوش چون آب جوی \\
همه کینه‌جویان پرخاشجوی & ز بالا بلشکر نهادند روی \\
ابا کشتگان بسته بر پشت زین & بریشان سرآورده پرخاش و کین \\
چو با کینه‌جویان نبد پهلوان & خروشی برآمد ز پیر و جوان \\
که گودرز بر دست پیران مگر & ز پیری بخون اندر آورد سر \\
همی زار بگریست لشکر همه & ز نادیدن پهلوان رمه \\
درفشی پدید آمد از تیره گرد & گرازان و تازان بدشت نبرد \\
برآمد ز لشکرگه آوای کوس & همی گرد بر آسمان داد بوس \\
بزرگان بر پهلوان آمدند & پر از خنده و شادمان آمدند \\
چنین گفت لشکر مگر پهلوان & ازو بازگردید تیره روان \\
که پیران یکی شیردل مرد بود & همه ساله جویای آورد بود \\
چنین یاد کرد آن زمان پهلوان & سپرده بدو گوش پیر و جوان \\
بانگشت بنمود جای نبرد & بگفت آنک با او زمانه چه کرد \\
برهام فرمود تا برنشست & بوردن او میان را ببست \\
بدو گفت او را بزین برببند & بیاور چنان تازیان بر نوند \\
درفش و سلیحش چنان هم که هست & بدرع و میانش مبر هیچ دست \\
بران گونه چون پهلوان کرد یاد & برون تاخت رهام چون تندباد \\
کشید از بر اسب روشن تنش & بخون اندرون غرقه بد جوشنش \\
چنان هم ببستش بخم کمند & فرود آوریدش ز کوه بلند \\
درفشش چو از جایگاه نشان & ندیدند گردان گردنکشان \\
همه خواندند آفرین سربسر & ابر پهلوان زمین دربدر \\
که ای نامور پشت ایران سپاه & پرستندهٔ تخت تو باد ماه \\
فدای سپه کرده‌ای جان و تن & بپیری زمان روزگار کهن \\
چنین گفت گودرز با مهتران & که چون رزم ما گشت زین سان گران \\
مرا در دل آید که افراسیاب & سپه بگذراند بدین روی آب \\
سپاه وی آسوده از رنج و تاب & بمانده سپاهم چنین در شتاب \\
ولیکن چنین دارم امید من & که آید جهاندار خورشید من \\
بیفروزد این رزمگه را بفر & بیارد سپاهی بنو کینه‌ور \\
یکی هوشمندی فرستاده‌ام & بس شاه را پندها داده‌ام \\
که گر شاه ترکان بیارد سپاه & نداریم پای اندرین کینه‌گاه \\
گمانم چنانست کو با سپاه & بیاری بیاید بدین رزمگاه \\
مر این کشتگان را برین دشت کین & چنین هم بدارید بر پشت زین \\
کزین کشتگان جان ما بیغمست & روان سیاوش زین خرمست \\
اگر هم چنین نزد شاه آوریم & شود شاد و زین پایگاه آوریم \\
که آشوب ترکان و ایرانیان & ازین بد کجا کم شد اندر میان \\
همه یکسره خواندند آفرین & که بی تو مبادا زمان و زمین \\
همه سودمندی ز گفتار تست & خور و ماه روشن بدیدار تست \\
برفتند با کشتگان همچنان & گروی زره را پیاده دوان \\
چو نزدیک بنگاه و لشکر شدند & پذیرهٔ سپهبد سپاه آمدند \\
بپیش سپه بود گستهم شیر & بیامد بر پهلوان دلیر \\
زمین را ببوسید و کرد آفرین & سپاهت بی‌آزار گفتا ببین \\
چنانچون سپردی سپردم بهم & درین بود گودرز با گستهم \\
که اندر زمان از لب دیده‌بان & بگوش آمد از کوه زیبد فغان \\
که از گرد شد دشت چون تیره شب & شگفتی برآمد ز هر سو جلب \\
خروشیدن کوس با کرنای & بجنباند آن دشت گویی ز جای \\
یکی تخت پیروزه بر پشت پیل & درفشان بکردار دریای نیل \\
هوا شد بسان پرند بنفش & ز تابیدن کاویانی درفش \\
درفشی ببالای سرو سهی & پدید آمد از دور با فرهی \\
بگردش سواران جوشنوران & زمین شد بنفش از کران تا کران \\
پس هر درفشی درفشی بپای & چه از اژدها و چه پیکر همای \\
ارگ همچنین تیزرانی کنند & بیک روز دیگر بدینجا رسند \\
ز کوه کنابد همان دیده‌بان & بدید آن شگفتی و آمد دوان \\
چنین گفت گر چشم من تیره نیست & وز اندوه دیدار من خیره نیست \\
ز ترکان برآورد ایزد دمار & همه رنجشان سربسر گشت خوار \\
سپاه اندر آمد ز بالا بپست & خروشان و هر یک درفشی بدست \\
درفش سپهدار توران نگون & همی بینم از پیش غرقه بخون \\
همان ده دلاور کز ایدر برفت & ابا گرد پیران بورد تفت \\
همی بینم از دورشان سرنگون & فگنده بر اسبان و تن پر ز خون \\
دلیران ایران گرازان بهم & رسیدند یکسر بر گستهم \\
وزان سوی زیبد یکی تیره‌گرد & پدید آمد و دشت شد لاژورد \\
میان سپه کاویانی درفش & بپیش اندرون تیغهای بنفش \\
درفش شهنشاه با بوق و کوس & پدید آمد و شد زمین آبنوس \\
برفتند لهاک و فرشیدورد & بدانجا که بد جایگاه نبرد \\
بدیدند کشته بدیدار خویش & سپهبد برادر جهاندار خویش \\
ابا ده سوار آن گزیده سران & ز ترکان دلیران جنگاوران \\
بران دیده برزار و جوشان شدند & ز خون برادر خروشان شدند \\
همی زار گفتند کای نره شیر & سپهدار پیران سوار دلیر \\
چه بایست آن رادی و راستی & چو رفتن ز گیتی چنین خواستی \\
کنون کام دشمن برآمد همه & ببد بر تو گیتی سرآمد همه \\
که جوید کنون در جهان کین تو & که گیرد کنون راه و آیین تو \\
ازین شهر ترکان و افراسیاب & بد آمد سرانجامت ای نیک‌یاب \\
بباید بریدن سر خویش پست & بخون غرقه کردن بر و یال و دست \\
چو اندرز پیران نهادند پیش & نرفتند بر خیره گفتار خویش \\
ز گودرز چون خواست پیران نبرد & چنین گفت با گرد فرشیدورد \\
که گر من شوم کشته بر کینه‌گاه & شما کس مباشید پیش سپاه \\
اگر کشته گردم برین دشت کین & شود تنگ بر نامداران زمین \\
نه از تخمهٔ ویسه ماند کسی & که اندر سرش مغز باشد بسی \\
که بر کینه‌گه چونک ما را کشند & چو سرهای ما سوی ایران کشند \\
ز گودرز خواهد سپه زینهار & شما خویشتن را مدارید خوار \\
همه راه سوی بیابان برید & مگر کز بد دشمنان جان برید \\
بلشکر گه خویش رفتند باز & همه دیده پر خون و دل پر گداز \\
بدانست لشکر سراسر همه & که شد بی‌شبان آن گرازان رمه \\
همه سربسر زار و گریان شدند & چو بر آتش تیز بریان شدند \\
بنزدیک لهاک و فرشیدورد & برفتند با دل پر از باد سرد \\
که اکنون چه سازیم زین رزمگاه & چو شد پهلوان پشت توران سپاه \\
چنین گفت هر کس که پیران گرد & جز از نام نیکو ز گیهان نبرد \\
کرا دل دهد نیز بستن کمر & ز آهن کله برنهادن بسر \\
چنین گفت لهاک فرشیدورد & که از خواست یزدان کرانه که کرد \\
چنین راند بر سر ورا روزگار & که بر کینه کشته شود زار و خوار \\
بشمشیر کرده جدا سر ز تن & نیابد همی کشته گور و کفن \\
بهرجای کشته کشان دشمنش & پر از خون سر و درع و خسته تنش \\
کنون بودنی بود و پیران گذشت & همه کار و کردار او باد گشت \\
ستون سپه بود تا زنده بود & بمهر سپه جانش آگنده بود \\
سپه را ز دشمن نگهدار بود & پسر با برادر برش خوار بود \\
بدان گیتی افتاد نیک و بدش & همانا که نیک است با ایزدش \\
بس از لشکر خویش تیمار خورد & ز گودرز پیمان ستد در نبرد \\
که گر من شوم کشته در کینه‌گاه & نجویی تو کین زان سپس با سپاه \\
گذرشان دهی تا بتوران شوند & کمین را نسازی بریشان کمند \\
ز پیمان نگردند ایرانیان & ازین در کنون نیست بیم زیان \\
سه کارست پیش‌آمده ناگزیر & همه گوش دارید برنا و پیر \\
اگرتان بزنهار باید شدن & کنونتان همی رای باید زدن \\
وگر بازگشتن بخرگاه خویش & سپردن بنیک و ببد راه خویش \\
وگر جنگ را گرد کرده عنان & یکایک بخوناب داده سنان \\
گر ایدون کتان دل گراید بجنگ & بدین رزمگه کرد باید درنگ \\
که پیران ز مهتر سپه خواستست & سپهبد یکی لشکر آراستست \\
زمان تا زمان لشکر آید پدید & همی کینه زینشان بباید کشید \\
ز هرگونه رانیم یکسر سخن & جز از خواست یزدان نباشد ز بن \\
ور ایدون کتان رای شهرست و گاه & همانا که بر ما نگیرند راه \\
وگرتان بزنهار شاهست رای & بباید بسیچید و رفتن ز جای \\
وگرتان سوی شهر ایران هواست & دل هر کسی بر تنش پادشاست \\
ز ما دو برادر مدارید چشم & که هرگز نشوییم دل را ز خشم \\
کزین تخمهٔ ویسگان کس نبود & که بند کمر بر میانش نسود \\
بر اندرز سالار پیران شویم & ز راه بیابان بتوران شویم \\
ار ایدونک بر ما بگیرند راه & بکوشیم تا هستمان دستگاه \\
چو ترکان شنیدند زیشان سخن & یکی نیک پاسخ فگندند بن \\
که سالار با ده یل نامدار & کشیدند کشته بران گونه خوار \\
وزان روی کیخسرو آمد پدید & که یارد بدین رزمگاه آرمید \\
نه اسب و سلیح و نه پای و نه پر & نه گنج و نه سالار و نه نامور \\
نه نیروی جنگ و نه راه گریز & چه با خویشتن کرد باید ستیز \\
اگر بازگردیم گودرز و شاه & پس ما برانند پیل و سپاه \\
رهایی نیابیم یک تن بجان & نه خرگاه بینیم و نه دودمان \\
بزنهار بر ما کنون عار نیست & سپاهست بسیار و سالار نیست \\
ازان پس خود از شاه ترکان چه باک & چه افراسیاب و چه یک مشت خاک \\
چو لشکر چنین پاسخ آراستند & دو پرمایه از جای برخاستند \\
بدانست لهاک و فرشیدورد & کشان نیست هنگام ننگ و نبرد \\
همی راست گویند لشکر همه & تبه گردد از بی‌شبانی رمه \\
بپدرود کردند گرفتند ساز & بیابان گرفتند و راه دراز \\
درفشی گرفته بدست اندرون & پر از درد دل دیدگان پر ز خون \\
برفتند با نامور ده سوار & دلیران و شایستهٔ کارزار \\
بره بر ز ایران سواران بدند & نگهبان آن نامداران بدند \\
برانگیختند اسب ترکان ز جای & طلایه بیفشارد با جای پای \\
یکی ناسگالیده‌شان جنگ خاست & که از خون زمین گشت با کوه راست \\
بکشتند ایرانیان هشت مرد & دلیران و شیران روز نبرد \\
وزانجا برفتند هر دو دلیر & براه بیابان بکردار شیر \\
ز ترکان جزین دو سرافراز گرد & ز دست طلایه دگر جان نبرد \\
پس از دیده گه دیده‌بان کرد غو & که ای سرفرازان و گردان نو \\
ازین لشکر ترک دو نامدار & برون رفت با نامور ده سوار \\
چنان با طلایه برآویختند & که با خاک خون را برآمیختند \\
تنی هشت کشتند ایرانیان & دو تن تیز رفتند بسته میان \\
چو بشنید گودرز گفت آن دو مرد & بود گرد لهاک و فرشیدورد \\
برفتند با گردان افراختن & شکسته نشدشان دل از تاختن \\
گر ایشان از اینجا به توران شوند & بر این لشکر آید همانا گزند \\
هم اندر زمان گفت با سرکشان & که ای نامداران دشمن‌کشان \\
که جوید کنون نام نزدیک شاه & بپوشد سرش را برومی کلاه \\
همه مانده بودند ایرانیان & شده سست و سوده ز آهن میان \\
ندادند پاسخ جز از گستهم & که بود اندر آورد شیر دژم \\
بسالار گفت ای سرافراز شاه & چو رفتی بورد توران سپاه \\
سپردی مرا کوس و پرده‌سرای & بپیش سپه برببودن بپای \\
دلیران همه نام جستند و ننگ & مرا بهره نمد بهنگام جنگ \\
کنون من بدین کار نام آورم & شومشان یکایک بدام آورم \\
بخندید گودرز و زو شاد شد & رخش تازه شد وز غم آزاد شد \\
بدو گفت نیک‌اختری تو ز هور & که شیری و بدخواه تو همچو گور \\
برو کفریننده یار تو باد & چو لهاک سیصد شکار تو باد \\
بپوشید گستهم درع نبرد & ز گردان کرا دید پدرود کرد \\
برون رفت وز لشکر خویش تفت & بجنگ دو ترک سرافراز رفت \\
همی گفت لشکر همه سربسر & که گستهم را زین بد آید بسر \\
یکی لشکر از نزد افراسیاب & همی رفت برسان کشتی برآب \\
بیاری همه جنگجو آمدند & چو نزدیک دشت دغو آمدند \\
خبر شد بدیشان که پیران گذشت & نبرد دلیران دگرگونه گشت \\
همه بازگشتند یکسر ز راه & خروشان برفتند نزدیک شاه \\
چو بشنید بیژن که گستهم رفت & ز لشکر بورد لهاک تفت \\
گمانی چنان برد بیژن که او & چو تنگ اندر آید بدشت دغو \\
نباید که لهاک و فرشیدورد & برآرند ازو خاک روز نبرد \\
نشست از بر دیزهٔ راه‌جوی & بنزدیک گودرز بنهاد روی \\
چو چشمش بروی نیا برفتاد & خروشید و چندی سخن کرد یاد \\
نه خوب آید ای پهلوان از خرد & که هر نامداری که فرمان برد \\
مر او را بخیره بکشتن دهی & بهانه بچرخ فلک برنهی \\
دو تن نامداران توران سپاه & برفتند زین سان دلاور براه \\
ز هومان و پیران دلاورترند & بگوهر بزرگان آن کشورند \\
کنون گستهم شد بجنگ دو تن & نباید که آید برو برشکن \\
همه کام ما بازگردد بدرد & چو کم گردد از لشکر آن رادمرد \\
چو بشنید گودرز گفتار اوی & کشیدن بدان کار تیمار اوی \\
پس اندیشه کرد اندران یک زمان & هم از بد که می‌برد بیژن گمان \\
بگردان چنین گفت سالار شاه & که هر کس که جوید همی نام و گاه \\
پس گستهم رفت باید دمان & مر او را بدن یار با بدگمان \\
ندادند پاسخ کس از انجمن & نه غمخواره بد کس نه آسوده‌تن \\
بگودرز پس گفت بیژن که کس & جز من نباشدش فریادرس \\
که آید ز گردان بدین کار پیش & بسیری نیامد کس از جان خویش \\
مرا رفت باید که از کار اوی & دلم پر ز درد است و پر آب روی \\
بدو گفت گودرز کای شیرمرد & نه گرم آزموده ز گیتی نه سرد \\
نبینی که ماییم پیروزگر & بدین کار مشتاب تند ای پسر \\
بریشان بود گستهم چیره‌بخت & وزیشان ستاند سرو تاج و تخت \\
بمان تا کنون از پس گستهم & سواری فرستم چو شیر دژم \\
که با او بود یارگاه نبرد & سر دشمنان اندر آرد بگرد \\
بدو گفت بیژن که ای پهلوان & خردمند و بیدار و روشن‌روان \\
کنون یار باید که زندست مرد & نه آنگه کجا زو برآرند گرد \\
چو گستهم شد کشته در کارزار & سرآمد برو روز و برگشت کار \\
کجا سود دارد مر او را سپاه & کنون دار گر داشت خواهی نگاه \\
بفرمای تا من ز تیمار اوی & ببندم کمر تنگ بر کار اوی \\
ور ایدونک گویی مرو من سرم & ببرم بدین آبگون خنجرم \\
که من زندگانی پس از مرگ اوی & نخواهم که باشد بهانه مجوی \\
بدو گفت گودرز بشتاب پیش & اگر نیست مهر تو بر جان خویش \\
نیابی همی سیری از کارزار & کمر بند و ببسیچ و سر بر مخار \\
نسوزد همانا دلت بر پدر & که هزمان مر او را بسوزی جگر \\
چو بشنید بیژن فرو برد سر & زمین را ببوسید و آمد بدر \\
برآرم همی گفت از کوه خاک & بدین جنگ جستن مرا زو چه باک \\
کمر بست و برساخت مر جنگ را & بزین اندر آورد شبرنگ را \\
بگیو آگهی شد که بیژن چو گرد & کمر بست بر جنگ فرشیدورد \\
پس گستهم تازیان شد براه & بجنگ سواران توران سپاه \\
هم اندر زمان گیو برجست زود & نشست از بر تازی اسبی چو دود \\
بیامد بره بر چو او را بدید & به تندی عنانش بیکسو کشید \\
بدو گفت چندین زدم داستان & نخواهی همی بود همداستان \\
که باشم بتو شادمان یک زمان & کجا رفت خواهی بدین سان دمان \\
بهر کار درد دلم را مجوی & بپیران سر از من چه باید بگوی \\
جز از تو بگیتیم فرزند نیست & روانم بدرد تو خرسند نیست \\
بدی ده شبان روز بر پشت زین & کشیده ببدخواه بر تیغ کین \\
بسودی بخفتان و خود اندرون & نخواهی همی سیر گشتن ز خون \\
چو نیکی دهش بخت پیروز داد & بباید نشستن برام و شاد \\
بپیش زمانه چه تازی سرت & بس ایمن شدستی بدین خنجرت \\
کسی کو بجوید سرانجام خویش & نجوید ز گیتی چنین کام خویش \\
تو چندین بگرد زمانه مپوی & که او خود سوی ما نهادست روی \\
ز بهر مرا زین سخن بازگرد & نشاید که دارای دل من بدرد \\
بدو گفت بیژن که ای پر خرد & جزین بر تو مردم گمانی برد \\
که کار گذشته بیاری بیاد & نپیچی بخیره همی سر زداد \\
بدان ای پدر کین سخن داد نیست & مگر جنگ لاون ترا یاد نیست \\
که با من چه کرد اندران گستهم & غم و شادمانیش با من بهم \\
ورایدون کجا گردش ایزدی & فرازآورد روزگار بدی \\
نبشته نگردد بپرهیز باز & نباید کشید این سخن را دراز \\
ز پیکار سر بر مگردان که من & فدی کرده دارم بدین کار تن \\
بدو گفت گیو ار بگردی تو باز & همان خوبتر کین نشیب و فراز \\
تو بی‌من مپویی بروز نبرد & منت یار باشم بهر کارکرد \\
بدو گفت بیژن که این خود مباد & که از نامداران خسرونژاد \\
سه گرد از پی بیم خورده دو تور & بتازند پویان بدین راه دور \\
بجان و سر شاه روشن‌روان & بجان نیا نامور پهلوان \\
بکین سیاوش کزین رزمگاه & تو برگردی و من بپویم براه \\
نخواهم برین کار فرمانت کرد & که گویی مرا بازگرد از نبرد \\
چو بشنید گیو این سخن بازگشت & برو آفرین کرد و اندر گذشت \\
که پیروز بادی و شاد آمدی & مبیناد چشم تو هرگز بدی \\
همی تاخت بیژن پس گستهم & که ناید بروبر ز توران ستم \\
چو از دور لهاک و فرشیدورد & گذشتند پویان ز دشت نبرد \\
بیک ساعت از هفت فرسنگ راه & برفتند ایمن ز ایران سپاه \\
یکی بیشه دیدند و آب روان & بدو اندرون سایهٔ کاروان \\
ببیشه درون مرغ و نخچیر و شیر & درخت از بر و سبزه و آب زیر \\
بنخچیر کردن فرود آمدند & وزان تشنگی سوی رود آمدند \\
چو آب اندر آمد ببایست نان & باندوه و شادی نبندد دهان \\
بگشتند بر گرد آن مرغزار & فگندند بسیار مایه شکار \\
برافروختند آتش و زان کباب & بخوردند و کردند سر سوی خواب \\
چو بد روزگار دلیران دژم & کجا خواب سازد بریشان ستم \\
فرو خفت لهاک و فرشیدورد & بسر بر همی پاسبانیش کرد \\
برآمد چو شب تیره شد ماهتاب & دو غمگین سر اندر نهاده بخواب \\
رسید اندران جایگه گستهم & که بودند یاران توران بهم \\
نوند اسب او بوی اسبان شنید & خروشی برآورد و اندر دمید \\
سبک اسب لهاک هم زین نشان & خروشی برآورد چون بیهشان \\
دمان سوی لهاک فرشید ورد & ز خواب خوش آمدش بیدار کرد \\
بدو گفت برخیز زین خواب خوش & بمردی سر بخت خود را بکش \\
که دانا زد این داستان بزرگ & که شیری که بگریزد از چنگ گرگ \\
نباید که گرگ از پسش در کشد & که او را همان بخت خود برکشد \\
چه مایه بپیوند و چندی شتافت & کس از روز بد هم رهایی نیافت \\
هلا زود بشتاب کآمد سپاه & از ایران و بر ما گرفتند راه \\
نشستند بر باره هر دو سوار & کشیدند پویان ازان مرغزار \\
ز بیشه ببالا نهادند روی & دو خونی دلاور دو پرخاشجوی \\
بهامون کشیدند هر دو سوار & پراندیشه تا چون بسیچند کار \\
پدید آمد از دور پس گستهم & ندیدند با او سواری بهم \\
دلیران چو سر را برافراختند & مر او را چو دیدند بشناختند \\
گرفتند یک بادگر گفت و گوی & که یک تن سوی ما نهادست روی \\
نیابد رهایی ز ما گستهم & مگر بخت بد کرد خواهد ستم \\
جز از گستهم نیست کامد بجنگ & درفش دلیران گرفته بچنگ \\
گریزان بباید شد از پیش اوی & مگر کاندر آرد بدین دشت روی \\
وز آنجا بهامون نهادند روی & پس اندر دمان گستهم کینه‌جوی \\
بیامد چو نزدیک ایشان رسید & چو شیر ژیان نعره‌ای برکشید \\
بریشان ببارید تیر خدنگ & چو فرشیدورد اندر آمد بجنگ \\
یکی تیر زد بر سرش گستهم & که با خون برآمیخت مغزش بهم \\
نگون گشت و هم در زمان جان بداد & شد آن نامور گرد ویسه نژاد \\
چو لهاک روی برادر بدید & بدانست کز کارزار آرمید \\
بلرزید وز درد او خیره شد & جهان پیش چشم‌اندرش تیره شد \\
ز روشن‌روانش بسیری رسید & کمان را بزه کرد و اندر کشید \\
شدند آن زمان خسته هر دو سوار & بشمشیر برساختند کارزار \\
یکایک برو گستهم دست یافت & ز کینه چنان خسته اندر شتافت \\
بگردنش بر زد یکی تیغ تیز & برآورد ناگاه زو رستخیز \\
سرش زیر پای اندر آمد چو گوی & که آید همی زخم چوگان بروی \\
چنینست کردار گردان سپهر & ببرد ز پروردهٔ خویش مهر \\
چو سر جوییش پای یابی نخست & وگر پای جویی سرش پیش تست \\
بزین بر چنان خسته بد گستهم & که بگسست خواهد تو گفتی ز هم \\
بیامد خمیده بزین اندرون & همی راند اسب و همی ریخت خون \\
و زآنجا سوی چشمه‌ساری رسید & هم آب روان دید و هم سایه دید \\
فرود آمد و اسب را بر درخت & ببست و به آب اندر آمد ز بخت \\
بخورد آب بسیار و کرد آفرین & ببستش تو گفتی سراسر زمین \\
بپیچید و غلتید بر تیره خاک & سراسر همه تن بشمشیر چاک \\
همی گفت کای روشن کردگار & پدید آر زان لشکر نامدار \\
بدلسوزگی بیژن گیو را & وگرنه دلاور یکی نیو را \\
که گر مرده گر زنده‌زین جایگاه & برد مر مرا سوی ایران سپاه \\
سر نامداران توران سپاه & ببرد برد پیش بیدار شاه \\
بدان تا بداند که من جز بنام & نمردم بگیتی همینست کام \\
همه شب بنالید تا روز پاک & پر از درد چون مار پیچان بخاک \\
چو گیتی ز خورشید شد روشنا & بیامد بدانجایگه بیژنا \\
همی گشت بر گرد آن مرغزار & که یابد نشانی ز گم بوده یار \\
پدید آمد از دور اسب سمند & بدان مرغزار اندرون چون نوند \\
چمان و چران چون پلنگان بکام & نگون گشته زین و گسسته لگام \\
همه آلت زین برو بر نگون & رکیب و کمند و جنا پر ز خون \\
چو بیژن بدید آن ازو رفت هوش & برآورد چو شیر شرزه خروش \\
همی گفت که ای مهربان نیک‌یار & کجایی فگنده در این مرغزار \\
که پشتم شکستی و خستی دلم & کنون جان شیرین ز تن بگسلم \\
بشد بر پی اسب بر چشمه‌سار & مر او را بدید اندران مزغزار \\
همه جوشن ترگ پر خاک و خون & فتاده بدان خستگی سرنگون \\
فروجست بیژن ز شبرنگ زود & گرفتش بغوش در تنگ زود \\
برون کرد رومی قبا از برش & برهنه شد از ترگ خسته سرش \\
ز بس خون دویدن تنش بود زرد & دلش پر ز تیمار و جان پر ز درد \\
بران خستگیهاش بنهاد روی & همی بود زاری کنان پیش اوی \\
همی گفت کای نیک دل یار من & تو رفتی و این بود پیکار من \\
شتابم کنون بیش بایست کرد & رسیدن بر تو بجای نبرد \\
مگر بودمی گاه سختیت یار & چو با اهرمن ساختی کارزار \\
کنون کام دشمن همه راست کرد & برآنرد سر هرچ می‌خواست کرد \\
بگفت این سخن بیژن و گستهم & بجنبید و برزد یکی تیز دم \\
ببیژن چنین گفت کای نیک خواه & مکن خویشتن پیش من در تباه \\
مرا درد تو بتر از مرگ خویش & بنه بر سر خسته بر ترگ خویش \\
یکی چاره کن تا ازین جایگاه & توانی رسانیدنم نزد شاه \\
مرا باد چندان همی روزگار & که بینم یکی چهرهٔ شهریار \\
ازان پس چو مرگ آیدم باک نیست & مرا خود نهالی بجز خاک نیست \\
نمردست هرکس که با کام خویش & بمیرد بیابد سرانجام خویش \\
و دیگر دو بد خواه با ترس و باک & که بر دست من کرد یزدان هلاک \\
مگرشان بزین بر توانی کشید & وگرنه سرانشان ز تنها برید \\
سلیح و سر نامبردارشان & ببر تا بدانند پیکارشان \\
کنی نزد شاه جهاندار یاد & که من سر بخیره ندادم بباد \\
بسودم بهر جای بابخت جنگ & گهٔ نام جستن نمردم بننگ \\
ببیژن نمود آنگهی هر دو تور & که بودند کشته فگنده بدور \\
بگفت این و سستی گرفتش روان & همی بود بیژن بسر بر نوان \\
وز آن جایگه اسب او بیدرنگ & بیاورد و بگشاد از باره تنگ \\
نمد زین بزیر تن خفته مرد & بیفگند و نالید چندی بدرد \\
همه دامن قرطه را کرد چاک & ابر خستگیهاش بر بست پاک \\
وز آن جایگه سوی بالا دوان & بیامد ز غم تیره کرده روان \\
سواران ترکان پراگنده دید & که آمد ز راه بیابان پدید \\
ز بالا چو برق اندر آمد بشیب & دل از مردن گستهم با نهیب \\
ازان بیم دیده سواران دو تن & بشمشیرکم کرد زان انجمن \\
ز فتراک بگشاد زان پس کمند & ز ترکان یکی را بگردن فگند \\
ز اسب اندر آورد و زنهار داد & بدان کار با خویشتن یار داد \\
وز آنجا بیامد بکردار گرد & دمان سوی لهاک و فرشیدورد \\
بدید آن سران سپه را نگون & فگنده بران خاک غرقه بخون \\
بسرشان بر اسبان جنگی بپای & چراگاه سازید و جای چرای \\
چو بیژن چنان دید کرد آفرین & ابر گستهم کو سرآورد کین \\
بفرمود تا ترک زنهار خواه & بزین برکشید آن سران را ز راه \\
ببستندشان دست و پای و میان & کشیدند بر پشت زین کیان \\
وزآنجا سوی گستهم تازیان & بیامد بسان پلنگ ژیان \\
فرود آمد از اسب و او را چو باد & بی آزار نرم از بر زین نهاد \\
بدان ترک فرمود تا برنشست & بغوش او اندر آورد دست \\
سمند نوندش همی راند نرم & بروبر همی آفرین خواند گرم \\
مرگ زنده او را بر شهریار & تواند رسانیدن از کارزار \\
همی راند بیژن پر از درد و غم & روانش پر از انده گستهم \\
چو از روزنه ساعت اندر گذشت & خور از گنبد چرخ گردان بگشت \\
جهاندار خسرو بنزد سپاه & بیامد بدان دشت آوردگاه \\
پذیره شدندش سراسر سران & همه نامداران و جنگاوران \\
برو خواندند آفرین بخردان & که ای شهریار و سر موبدان \\
چنان هم همی بود بر اسب شاه & بدان تا ببینند رویش سپاه \\
بریشان همی خواند شاه آفرین & که آباد بادا بگردان زمین \\
بیین پس پشت لشکر چو کوه & همی رفت گودرز با آن گروه \\
سر کشتگانرا فگنده نگون & سلیح و تن و جامه هاشان بخون \\
همان ده مبارز کز آوردگاه & بیاورده بودند گردان شاه \\
پس لشکر اندر همی راندند & ابر شهریار آفرین خواندند \\
چو گودرز نزدیک خسرو رسید & پیاده شد از دور کو را بدید \\
ستایش کنان پهلوان سپاه & بیامد بغلتید در پیش شاه \\
همه کشتگانرا بخسرو نمود & بگفتش که همرزم هر کس که بود \\
گروی زره را بیاودر گیو & دمان با سپهدار پیران نیو \\
ز اسب اندر آمد سبک شهریار & نیایش همی کرد برکردگار \\
ز یزدان سپاس و بدویم پناه & که او داد پیروزی و دستگاه \\
ز دادار بر پهلوان آفرین & همی خواند و بر لشکرش همچنین \\
که ای نامداران فرخنده پی & شما آتش و دشمنان خشک نی \\
سپهدار گودرز با دودمان & ز بهر دل من چو آتش دمان \\
همه جان و تنها فدا کرده‌اند & دم از شهر توران برآورده‌اند \\
کنون گنج و شاهی مرا با شماست & ندارم دریغ از شما دست راست \\
ازان پس بدان کشتگان بنگرید & چو روی سپهدار پیران بدید \\
فروریخت آب از دو دیده بدرد & که کردار نیکی همی یاد کرد \\
بپیرانش بر دل ازان سان بسوخت & تو گفتی بدلش آتشی برفروخت \\
یکی داستان زد پس از مرگ اوی & بخون دو دیده بیالود روی \\
که بخت بدست اژدهای دژم & بدام آورد شیر شرزه بدم \\
بمردی نیابد کسی زو رها & چنین آمد این تیزچنگ اژدها \\
کشیدی همه ساله تیمار من & میان بسته بودی بپیکار من \\
ز خون سیاوش پر از درد بود & بدانگه کسی را نیازرد بود \\
چنان مهربان بود دژخیم شد & وزو شهر ایران پر از بیم شد \\
مر او را ببرد اهرمن دل ز جای & دگرگونه پیش اندر آورد پای \\
فراوان همی خیره دادمش پند & نیامدش گفتار من سودمند \\
از افراسیابش نه برگشت سر & کنون شهریارش چنین داد بر \\
مکافات او ما جز این خواستیم & همی گاه و دیهیمش آراستیم \\
از اندیشهٔ ما سخن درگذشت & فلک بر سرش بر دگرگونه گشت \\
بدل بر جفاکرد بر جای مهر & بدین سر دگرگونه بنمود چهر \\
کنون پند گودرز و فرمان من & بیفگند گفتار و پیمان من \\
تبه کرد مهر دل پاک را & بزهر اندر آمیخت تریاک را \\
که آمد بجنگ شما با سپاه & که چندان شد از شهر ایران تباه \\
ز توران بسیچید و آمد دمان & که ژوپین گودرز بودش زمان \\
پسر با برادر کلاه و کمر & سلیح و سپاه و همه بوم و بر \\
بداد از پی مهر افراسیاب & زمانه برو کرد چندین شتاب \\
بفرمود تا مشک و کافور ناب & بعنبر برآمیخته با گلاب \\
تنش را بیالود زان سربسر & بکافور و مشکش بیاگند سر \\
بدیبار رومی تن پاک اوی & بپوشید آن جان ناپاک اوی \\
یکی دخمه فرمود خسرو بمهر & بر آورده سر تا بگردان سپهر \\
نهاد اندرو تختهای گران & چنانچون بود در خور مهتران \\
نهادند مر پهلوان را بگاه & کمر بر میان و بسر برکلاه \\
چنینست کردار این پر فریب & چه مایه فرازست و چندی نشیب \\
خردمند را دل ز کردار اوی & بماند همی خیره از کار اوی \\
ازان پس گروی زره را بدید & یکی باد سرد از جگر برکشید \\
نگه کرد خسرو بدان زشت روی & چو دیوی بسر بر فروهشته موی \\
همی گفت کای کردگار جهان & تو دانی همی آشکار و نهان \\
همانا که کاوس بد کرده بود & بپاداش ازو زهر و کین آزمود \\
که دیوی چنین بر سیاوش گماشت & ندانم جزین کینه بر دل چه داشت \\
ولیکن بپیروزی یک خدای & جهاندار نیکی ده و رهنمای \\
که خون سیاوش ز افراسیاب & بخواهم بدین کینه گیرم شتاب \\
گروی زره را گره تا گره & بفرمود تا برکشیدند زه \\
چو بندش جداشد سرش را ز بند & بریدند همچون سر گوسفند \\
بفرمود او را فگندن به آب & بگفتا چنین بینم افراسیاب \\
ببد شاه چندی بران رزمگاه & بدان تا کند سازکار سپاه \\
دهد پادشاهی کرا در خورست & کسی کز در خلعت و افسرست \\
بگودرز داد آن زمان اصفهان & کلاه بزرگی و تخت مهان \\
باندازه اندر خور کارشان & بیاراست خلعت سزاوارشان \\
از آنها که بودند مانده بجای & که پیرانشان بد سرو کد خدای \\
فرستاده آمد بنزدیک شاه & خردمند مردی ز توران سپاه \\
که ما شاه را بنده و چاکریم & زمین جز بفرمان او نسپریم \\
کس از خواست یزدان نیابد رها & اگر چه شود در دم اژدها \\
جهاندار داند که ما خود کییم & میان تنگ بسته ز بهر چییم \\
نبدمان بکار سیاوش گناه & ببرد اهرمن شاه را دل ز راه \\
که توران ز ایران همه پر غمست & زن و کودک خرد در ماتمست \\
نه بر آرزو کینه خواه آمدیم & ز بهر بر و بوم و گاه آمدیم \\
ازین جنگ ما را بد آمد بسر & پسر بی پدر شد پدر بی پسر \\
بجان گر دهد شاهمان زینهار & ببندیم پیشش میان بنده‌وار \\
بدین لشکر اندر بس مهترست & کجا بندگی شاه را در خورست \\
گنهکار اوییم و او پادشاست & ازو هرچ آید بما بر رواست \\
سران سربسر نزد شاه آوریم & بسی پوزش اندر گناه آوریم \\
گر از ما بدلش اندرون کین بود & بریدن سر دشمن آیین بود \\
ور ایدونک بخشایش آرد رواست & همان کرد باید که او را هواست \\
چو بشنید گفتار ایشان بدرد & ببخشودشان شاه آزاد مرد \\
بفرمود تا پیش او آمدند & بران آرزو چاره‌جو آمدند \\
همه بر نهادند سر بر زمین & پر از خون دل و دیده پر آب کین \\
سپهبد سوی آسمان کرد سر & که ای دادگر داور چاره‌گر \\
همان لشکرست این که سر پر ز کین & همی خاک جستند ز ایران زمین \\
چنین کردشان ایزد دادگر & نه رای و نه دانش نه پای و نه پر \\
بدو دست یازم که او یار بس & ز گیتی نخواهیم فریادرس \\
بدین داستان زد یکی نیک رای & که از کین بزین اندر آورد پای \\
که این باره رخشنده تخت منست & کنون کار بیدار بخت منست \\
بدین کینه گر تخت و تاج آوریم & و گر رسم تابوت ساج آوریم \\
و گرنه بچنگ پلنگ اندرم & خور کرگسانست مغز سرم \\
کنون بر شما گشت کردار بد & شناسد هر آنکس که دارد خرد \\
نیم من بخون شما شسته چنگ & که گیرم چنین کار دشوار تنگ \\
همه یکسره در پناه منید & و گر چند بدخواه گاه منید \\
هر آنکس که خواهد نباشد رواست & بدین گفته افزایش آمد نه کاست \\
هر آنکس که خواهد سوی شاه خویش & گذارد نگیرم برو راه پیش \\
ز کمی و بیشی و از رنج و آز & بنیروی یزدان شدم بی نیاز \\
چو ترکان شنیدند گفتار شاه & ز سر بر گرفتند یکسر کلاه \\
بپیروزی شاه خستو شدند & پلنگان جنگی چو آهو شدند \\
بفرمود شاه جهان تا سلیح & بیارند تیغ و سنان و رمیح \\
ز بر گستوان و ز رومی کلاه & یکی توده کردند نزدیک شاه \\
بگرد اندرش سرخ و زرد و بنفش & زدند آن سرافراز ترکان درفش \\
بخوردند سوگندهای گران & که تا زنده‌ایم از کران تا کران \\
همه شاه را چاکر و بنده‌ایم & همه دل بمهر وی آگنده‌ایم \\
چو این کرده بودند بیدار شاه & ببخشید یکسر همه بر سپاه \\
ز همشان پس آنگه پراگنده کرد & همه بومش از مردم آگنده کرد \\
ازان پس خروش آمد از دیده‌گاه & که گرد سواران برآمد ز راه \\
سه اسب و دو کشته برو بسته زار & همی بینم از دور با یک سوار \\
همه نامداران ایران سپاه & نهادند چشم از شگفتی براه \\
که تا کیست از مرز توران زمین & که یارد گذشتن برین دشت کین \\
هم اندر زمان بیژن آمد دمان & ببازو بزه بر فگنده کمان \\
بر اسبان چو لهاک و فرشیدورد & فگنده نگونسار پرخون و گرد \\
بر اسبی دگر بر پر از درد و غم & بغوش ترک اندرون گستهم \\
چو بیژن بنزدیک خسرو رسید & سر تاج و تخت بلندش بدید \\
ببوسید و بر خاک بنهاد روی & بشد شاد خسرو بدیدار اوی \\
بپرسید و گفتش که ای شیر مرد & کجا رفته بودی ز دشت نبرد \\
ز گستهم بیژن سخن یاد کرد & ز لهاک وز گرد فرشیدورد \\
وزان خسته و زاری گستهم & ز جنگ سواران وز بیش و کم \\
کنون آرزو گستهم را یکیست & که آن کار بر شاه دشوار نیست \\
بدیدار شاه آمدستش هوا & وزان پس اگر میرد او را روا \\
بفرمود پس شاه آزرم جوی & که بردند گستهم را پیش اوی \\
چنان نیک دل شد ازو شهریار & که از گریه مژگانش آمد ببار \\
چنان بد ز بس خستگی گستهم & که گفتی همی برنیامدش دم \\
یکی بوی مهر شهنشاه یافت & بپیچید و دیده سوی او شتافت \\
ببارید از دیدگان آب مهر & سپهبد پر از آب و خون کرد چهر \\
بزرگان برو زار و گریان شدند & چو بر آتش تیز بریان شدند \\
دریغ آمد او را سپهبد بمرگ & که سندان کین بد سرش زیر ترگ \\
ز هوشنگ و طهمورث و جمشید & یکی مهره بد خستگان را امید \\
رسیده بمیراث نزدیک شاه & ببازوش برداشتی سال و ماه \\
چو مهر دلش گستهم را بخواست & گشاد آن گرانمایه از دست راست \\
ابر بازوی گستهم برببست & بمالید بر خستگیهاش دست \\
پزشکان که از روم و ز هند وچین & چه از شهر یونان و ایران زمین \\
ببالین گستهمشان بر نشاند & ز هر گونه افسون بر و بر بخواند \\
وز آنجا بیامد بجای نماز & بسی با جهان آفرین گفت راز \\
دو هفته برآمد بران خسته مرد & سر آمد همه رنج و سختی و درد \\
بر اسبش ببردند نزدیک شاه & چو شاه اندرو کرد لختی نگاه \\
بایرانیان گفت کز کردگار & بود هر کسی شاد و به روزگار \\
ولیکن شگفتست این کار من & بدین راستی بر شده یار من \\
بپیروزی اندر غم گستهم & نکرد این دل شادمان را دژم \\
بخواند آن زمان بیژن گیو را & بدو داد دست گو نیو را \\
که تو نیک‌بختی و یزدان شناس & مدار از تن خویش هرگز هراس \\
همه مهر پروردگارست و بس & ندانم بگیتی جز او هیچ کس \\
که اویست جاوید فریادرس & بسختی نگیرد جز او دست کس \\
اگر زنده گردد تن مرده مرد & جهاندار گستهم را زنده کرد \\
بدآنگه بدو گفت تیمار دار & چو بیژن نبیند کس از روزگار \\
کزو رنج بر مهر بگزیده‌ای & ستایش بدین گونه بشنیده‌ای \\
بزیبد ببد شاه یک هفته نیز & درم داد و دینار و هر گونه چیز \\
فرستاد هر سو فرستادگان & بنزد بزرگان و آزادگان \\
چو از جنگ پیران شدی بی‌نیاز & یکی رزم کیخسرو اکنون بساز 
\end{traditionalpoem}
\end{document}